\section{Дифракционный предел разрешения оптических и спектральных приборов. Критерий Рэлея.}

Вычисление интеграла Френеля для случая дифракции Фраунгофера показывает, что первый минимум интенсивности находится под углом $\lambda / D$, где $D$ -- ширина прямоугольной щели ($1.22 \lambda / D$ для круглой). В пределах этого угла находится основная часть излучения. В таком случае изображением точечного источника при прохождении света через линзу размера $D$ будет не точка, а пятно, причем угловой размер этого порядка $\lambda / D$. Если источников несколько, то при малом расстоянии между источниками их пятна Эйри будут перекрываться и для наблюдателя выглядеть как одно пятно.

\textit{Критерий Рэлея} устанавливает условие различимости источников: расстояние центрами (максимумами) дифракционных пятен должно быть не меньше радиуса одного из пятен (то есть максимум одного совпадает с минимумом другого). В соответствие с этим критерием говорят о \textit{разрешающей способности оптических приборов} -- минимальном угловом или линейном расстоянии, при котором два точечных источника воспринимаются раздельными.

Основным элементом оптических приборов является линза, рассмотрим применение критерия Рэлея для неё: пусть два источника находятся на расстоянии $L$ от линзы и $l$ друг от друга. Тогда расстояние между их изображениями в фокальной плоскости $\delta x = f \theta = f \frac{l}{L}$ (при малых углах). Для разрешимости необходимо

\begin{equation}
    f \theta > f \frac{\lambda}{D}
\end{equation}

\noindent
Для углового и линейного размера (в случае круглого отверстия добавляется множитель 1.22):

\begin{align}
    \theta &> \frac{\lambda}{D}, & l &> L \frac{\lambda}{D}
\end{align}

В случае спектральных приборов дифракционный предел играет ту же роль, однако ставится задача различимости не точечных источников, а спектральных линий. В большинстве случаев для определения разрешимости линий так-же используют критерий Рэлея.
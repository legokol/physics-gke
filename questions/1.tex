\section{Законы Ньютона. Движение тел в инерциальных и неинерциальных системах отсчета.}

\begin{definition}
    Масса (количество материи) есть характеристика тела, являющаяся мерой инерции.
\end{definition}

\begin{definition}
    Импульс (количество движения) есть произведение массы тела на его скорость: $\b p = m \b v$.
\end{definition}

\begin{definition}
    Взаимодействие сил определяется силой $\b F$, которая необходима для изменения состояния покоя или равномерного движения тела.
\end{definition}

\section*{Первый закон Ньютона}

\textit{Существуют системы отсчёта, называемые инерциальными, в которых тела (материальные точки), на которые внешние силы не действуют, или их действие скомпенсировано, покоятся или движутся равномерно прямолинейно.}

\begin{note}
    Инерциальные системы отсчёта движутся равномерно прямолинейно друг относительно друга.
\end{note}

\section*{Второй закон Ньютона}

\textit{В инерциальных системах отсчёта изменение импульса тела прямо пропорционально величине приложенной силы.}

\begin{equation} \label{eq:второй-закон-ньютона-в-импульсной-форме}
    \dot{\b p} = \b F
\end{equation}

Если же движение тела не является релятивистским, то \eqref{eq:второй-закон-ньютона-в-импульсной-форме} можно представить в виде:

\begin{equation} \label{eq:второй-закон-ньютона}
    m \b a = m \ddot{\b r} = \b F
\end{equation}

\section*{Третий закон Ньютона}

\textit{Силы взаимодействия двух материальных точек равны по величине, противоположно направлены и действуют вдоль прямой, соединяющей эти материальные точки.}

\subsection*{Движение тел в инерциальных системах отсчёта}

В инерциальных системах отсчёта движение тел (материальных точек) описывается законами Ньютона и получается при решении дифференциального уравнения \eqref{eq:второй-закон-ньютона-в-импульсной-форме}.

\subsection*{Движение тел в неинерциальных системах отсчёта}

Как было указано ранее, инерциальные системы отсчёта движутся относительно друга равномерно прямолинейно. Пусть в некоторой ИСО $(O_1)$ уравнение движения тела имеет вид

\begin{equation} \label{eq:уравнение-движения-в-исо}
    m  \b a_\text{абс} = \b F
\end{equation}

Пусть $\b R$ -- радиус-вектор точки $M$ в ИСО $O_1$, $\b R_0$ -- вектор $\b{O_1 O}$, описывающий положение НСО относительно ИСО, а $\b r$ -- радиус вектор точки относительно НСО \ref{fig:нсо}. В таком случае будут верны равенства

\begin{align}
    \b R &= \b R_0 + \b r \\
    \dot{\b R} &= \dot{\b R_0} + \dot{\b r} \\
    \ddot{\b R} &= \ddot{\b R_0} + \ddot{\b r}
\end{align}

\begin{figure}[htbp]
    \centering
    \input{images/1_НСО.pdf_tex}
    \caption{Неинерциальная система отсчёта}
    \label{fig:нсо}
\end{figure}

Рассмотрим сначала поступательно движение НСО. Тогда имеем

\begin{align}
    \b v_\text{абс} &= \b v_\text{отн} + \b v_\text{пер} \\
    \b a_\text{абс} &= \b a_\text{отн} + \b a_\text{пер}
\end{align}

Здесь относительная скорость и ускорение описывают движение точки в НСО, а переносные -- движение НСО относительно ИСО. Подставляя результаты в \eqref{eq:уравнение-движения-в-исо}, получим

\begin{equation}
    m  \b a_\text{отн} = \b F - m \b a_\text{пер}
\end{equation}

Слагаемое $- m \b a_\text{пер}$ называется (поступательной) силой инерции.

Если НСО вращается относительно ИСО со скоростью $\boldsymbol \omega$, то эти выражения усложняются:

\begin{align}
    \b{\dot r} &= \left[ \boldsymbol \omega, \b r \right] + \b{v_\text{отн}} \\
    \b{\ddot r} &= \left[\boldsymbol \epsilon, \b r \right] + \left[ \boldsymbol \omega, \left[ \boldsymbol \omega, \b r \right] \right] + 2 \left[ \boldsymbol \omega, \b{v_\text{отн}} \right] + \b a_\text{отн}
\end{align}

Первые два слагаемых относятся к переносному ускорению, а третье называется Кориолисовым. Соответственно они объединятся в силу инерции и Кориолисову силу.
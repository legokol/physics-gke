\section{Электромагнитная индукция в движущихся и неподвижных проводниках. ЭДС индукции. Само- и взаимоиндукция. Теорема взаимности.}

\textit{Электромагнитная индукция} -- явление возникновения тока (ЭДС индукции) в проводниках при наличии меняющегося магнитного поля. Так, например, если к катушке приближать постоянный магнит, в ней будет возникать индукционный ток.

Явление электромагнитной индукции в движущихся проводниках объясняется силой Лоренца. Рассмотрим простейший случай, когда два параллельных провода помещены в магнитное поле $\b B$, а проводящий мост $AB$ может скользить вдоль этих проводов (\ref{fig:эдс индукции}). Пусть мост движется со скоростью $\b v$. Тогда на свободные электроны в этом мосте действует сила Лоренца $\b F = \left( e / c \right) \left[ \b v, \b B \right]$, направленная вдоль проводника. На рисунке она действует вниз на положительный ион и вверх на отрицательный. В результате под действием этой силы в проводнике возникнет электрический ток. Сила Лоренца является сторонней силой, вызывающей ток, и соответственно стороннее электрическое напряжение равно $\b E^\text{стор} = \frac{\b F}{e}= \frac{1}{c} \left[ \b v, \b B \right]$. Возникающая электродвижущая сила называется \textit{ЭДС индукции} и обозначается $\mathcal{E}^\text{инд}$. В рассматриваемом случае $\mathcal{E}^\text{инд} = - \frac{v}{c} B l$, где $l$ -- длина мостика. Минус здесь поставлен потому, что сила направлена против положительного обхода контура (магнитное поле, вызванное индукционным током, направлено противоположно внешнему).

\begin{figure}[htbp]
    \centering
    \input{images/26_ЭДС.pdf_tex}
    \caption{Явление электромагнитной индукции в движущемся проводнике}
    \label{fig:эдс индукции}
\end{figure}

Величина $l v$ есть приращение площади контура в единицу времени, так что $v B l = d \Phi / dt$, где $\Phi$ -- магнитный поток, пронизывающий контур. Таким образом

\begin{equation}
    \mathcal{E}^\text{инд} = - \frac{1}{c} \frac{d \Phi}{dt}
\end{equation}

\noindent
Эта формула выражает закон индукции Фарадея и остаётся верной если, например, внешнее поле направлено под углом к контуру с током.

Индукционные токи могут возникать и в неподвижных проводниках. Опыт показывает, что изменение магнитного потока, пронизывающего контур, всегда приводит возникновению ЭДС индукции, причём величина этой ЭДС определяется законом Фарадея. Имеет место правило Ленца: \textit{индукционный ток всегда имеет такое направление, что он ослабляет действие причины, возбуждающей этот ток}.

Токи являются источниками магнитного поля. Таким образом, например, замкнутый контур с током пронизывает магнитное поле, создаваемое этим током. Если внутри контура провести произвольную поверхность $S$, магнитная индукция и поток через эту поверхность будут пропорциональные току, и можно написать

\begin{equation}
    \Phi = \frac{1}{c} L I
\end{equation}

\noindent
где коэффициент $L$ называется \textit{индуктивностью} проводника и определяется только его геометрическими параметрами. При переменном токе в проводнике поток, пронизывающий его, меняется, и возникает явление самоиндукции. В соответствие с законом Фарадея

\begin{equation}
    \mathcal{E}^\text{инд} = - \frac{L}{c} \frac{dI}{dt}
\end{equation}

Если рассмотреть два витка или катушки с током, то помимо вклада собственного тока в магнитный поток, имеет место вклад другого проводника. Для потоков можно записать

\begin{equation}
    \begin{aligned}
        \Phi_1 &= \frac{1}{c} L_{11} I_1 + \frac{1}{c} L_{12} I_2 \\
        \Phi_2 &= \frac{1}{c} L_{21} I_2 + \frac{1}{c} L_{22} I_2 \\
    \end{aligned}
\end{equation}

\noindent
Коэффициенты $L_{11}$ и $L_{22}$ -- это собственные индуктивности витков, а $L_{21}$ и $L_{12}$ -- взаимные индуктивности.

Электрический ток обладает запасом магнитной энергии. Получить выражение для неё просто. Если мы будем наращивать ток в цепи, то будет увеличиваться и магнитный поток, а значит источник будет совершать работу против ЭДС самоиндукции. Тогда

\begin{equation}
    \delta A^\text{внеш} = - \mathcal{E}^\text{инд} I dt = \frac{1}{c} I d \Phi
\end{equation}

\noindent
В случае немагнитной среды энергия пойдёт только на увеличение энергии, тогда после интегрирования имеем

\begin{equation}
    W = \frac{L}{2} \left( \frac{I}{c} \right)^2 = \frac{1}{2 c} I \Phi = \frac{\Phi^2}{2 L}
\end{equation}

\noindent
Для системы витков будем иметь:

\begin{equation*}
    dW = \frac{1}{c} \sum_i I_i d \Phi_i
\end{equation*}

\noindent
С учётом определения коэффициентов взаимной индукции, мы имеем

\begin{equation}
    dW = \frac{1}{c^2} \sum_i \sum_j L_{i j} I_i dI_j
\end{equation}

\noindent
Наконец, воспользуемся свойством равенства смешанных производных. Тогда

\begin{equation}
    \frac{1}{c^2} L_{i j} = \frac{\partial^2 W}{\partial I_i \partial I_j} = \frac{\partial^2 W}{\partial I_j \partial I_i} = \frac{1}{c^2} L_{j i}
\end{equation}

\noindent
что и доказывает \textit{теорему взаимности}.
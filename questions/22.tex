\section{Электростатическое поле в веществе. Вектор поляризации, электрическая индукция. Граничные условия для векторов \textbf{E} и \textbf{D}.}

Различные вещества по-разному реагируют на внешнее электрическое поле. В проводниках есть свободные электроны. Под действием внешнего поля происходит перераспределение этих электронов до тех пор, пока поле внутри проводника не станет равно нулю (действительно, иначе бы заряды продолжали движение под действием внешнего поля). Таким образом, с одной стороны проводники являются эквипотенциальными, а с другой плотность заряда в них равна нулю.

Совсем иначе ведут себя во внешнем поле диэлектрики. Они также реагируют на внешнее электрическое поле, но менее активно, поскольку не имеют свободных носителей заряда, заряды в диэлектрике могу смещаться лишь на малые расстояния, порядка атомных (например поворот полярных молекул). Под действием внешнего поля центр тяжести электронов в молекуле немного смещается относительно центра тяжести ядра, и молекулы становятся диполями, ориентированными положительными концами в направлении поля $\b E$.

В качестве количественной характеристики поляризации диэлектрика используется \textit{вектор поляризации} $\b P$. Он представляет дипольный момент единицы объёма:

\begin{equation}
    \b P = \frac{\sum \b p}{V}
\end{equation}

Рассмотрим однородный изотропный диэлектрик, имеющий форму косого параллелепипеда, и поместим его во внешнее поле, направленное параллельно боковым рёбрам. Тогда на основаниях параллелепипеда образуются поляризационные заряды с поверхностной плотностью $\sigma_\text{пол}$. На боковых гранях зарядов не будет, поскольку поле направлено параллельно им. Пусть $S$ -- площадь основания параллелепипеда, а $\b l$ -- вектор, направленный от отрицательного основания к положительному параллельно боковым рёбрам. Тогда дипольный момент будет равен $\sigma_\text{пол} S \b l$ а вектор поляризации

\begin{equation} \label{eq:поляризация параллелепипеда}
    \b P = \frac{\sigma_\text{пол} S \b l}{V}
\end{equation}

\noindent
Если $\b n$ -- вектор нормали к основанию (заряженному положительно), то $V = S \left( \b l, \b n \right)$. Подставляя это выражение в \eqref{eq:поляризация параллелепипеда} и домножив скалярно на $\b n$ получим

\begin{equation}
    \sigma_\text{пол} = \left( \b P, \b n \right) = P_n
\end{equation}

\noindent
Эта формула оказывается справедлива в общем случае (нормаль к отрицательному основанию направлена в противоположную сторону, а вектор поляризации параллелен боковым рёбрам).

При неоднородной поляризации могут появляться не только поверхностные, но и объёмные заряды. Выделим объём $V$ в диэлектрике, ограниченный поверхностью $S$. При поляризации через площадку $dS$ в отрицательном направлении нормали $\b n$ к этой площадке смещается заряд $dq = - P_n dS$. Таким образом заряд во всём объёме:

\begin{equation}
    q_\text{пол} = - \oint P_n dS = - \oint \b P d \b S
\end{equation}

\noindent
Возникающие поляризационные (связанные) заряды необходимо учесть в теореме Гаусса:

\begin{equation}
    \oint \b E d \b S = 4 \pi \left( q + q_\text{пол} \right)
\end{equation}

\noindent
Подставляя сюда выражение для поляризационных зарядов, имеем

\begin{equation}
    \oint \left( \b E + 4 \pi \b P \right) d \b S = 4 \pi q
\end{equation}

\noindent
Введём \textit{вектор электрической индукции}:

\begin{equation}
    \b D = \b E + 4 \pi \b P
\end{equation}

\noindent
Тогда Теорема Гаусса принимает вид

\begin{equation}
    \oint \left( \b D, d \b S \right) = 4 \pi q
\end{equation}

\noindent
где $q$ -- свободные заряды. В дифференциальной форме это соотношение имеет вид

\begin{equation}
    \nabla \cdot \b D = 4 \pi \rho
\end{equation}

Получим теперь граничные условия для полей $\b E$ и $\b D$. Рассмотрим малый элемент поверхности границы раздела двух сред. Выберем поверхность, охватывающую этот элемент: цилиндр, площадь которого равна $dS$ а высота пренебрежимо мала. Пусть $\sigma$ -- плотность свободных зарядов на поверхности. Тогда по теореме Гаусса имеем:

\begin{equation*}
    \Phi = \b D_1 d \b S_1 + \b D_2 d \b S_2 = 4 \pi \sigma dS
\end{equation*}

\noindent
Пусть $\b n$ -- нормаль по направлению из среды один к среде два. Тогда $d \b S_1 = \b n dS$, а $d \b S_2 = - \b n dS$. Сокращая $dS$ и учитывая, что скалярное произведение на вектор нормали есть нормальная составляющая, имеем

\begin{equation}
    D_{1n} - D_{2n} = 4 \pi \sigma
\end{equation}

Таким образом при прохождении через границу раздела сред вектор электрической индукции может испытывать скачок. Это и есть граничное условие на вектор электрической индукции. В случае вектора электрической напряжённости стоит также учесть плотность связанных (поляризационных) зарядов.

Теперь построим рядом с поверхностью контур в форме параллелограмма. Его можно сделать настолько малым, что две его грани будут параллельны поверхности, а вклад двух других в циркуляцию электрического поля будет пренебрежимо мал. Тогда с учётом теоремы о циркуляции

\begin{equation*}
    \b E_1 d \b l_1 + \b E_2 d \b l_2 = 0
\end{equation*}

\noindent
Пусть $\boldsymbol \tau$ -- единичный касательный к поверхности вектор, параллельный $d \b l_1$. Тогда $d \b l_1 = \boldsymbol \tau dl$, а $d \b l_2 = - \boldsymbol \tau dl$. Подставляя эти выражения в выражение для циркуляции и учитывая, что скалярное произведение на касательный вектор есть тангенциальная составляющая, имеем

\begin{equation}
    E_{1 \tau} - E_{2 \tau} = 0
\end{equation}

\noindent
То есть тангенциальные составляющие вектора напряжённости электрического поля совпадают.
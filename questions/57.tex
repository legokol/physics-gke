\section{Туннелирование частиц сквозь потенциальный барьер. Альфа-распад. Закон Гейгера-Нэттола и его
объяснение.}

Рассмотрим задачу о прохождении частицы через прямоугольный потенциальный барьер с высотой, большей энергии частицы. Пусть $\mathcal{E}$ -- энергия частицы, $U_0$ -- высота барьера, $a$ -- его ширина. $\Psi_1 = e^{i k x} + A e^{- i k x}$ и $\Psi_3 = C e^{i k x}$ -- решения стационарного уравнения Шрёдингера перед и после барьера. Здесь $k = \sqrt{\frac{2 m \mathcal{E}}{\h^2}}$. Единичный коэффициент перед первой экспонентной в $\Psi_1$ выбран исключительно для удобства, поскольку нас интересует только отношение коэффициентов.

Уравнение Шрёдингера над барьером будет иметь вид $\frac{\h^2}{2 m} \Psi'' + (\mathcal{E} - U_0) \Psi = 0$, его решение имеет вид $\Psi_2 = B_1 e^{\kappa x} + B_2 e^{- \kappa x}$, где $\kappa = \sqrt{\frac{2 m (U_0 - \mathcal{E})}{\h^2}}$. Для определения постоянных интегрирования необходимо <<сшить>> значения функций и первых производных на границах. Решая систему линейных уравнений, найдём величину $T = \left| C e^{i k a} \right|^2$ -- коэффициент прохождения. Получим

\begin{equation}
    D = \frac{4}{4 + \left( \frac{\kappa}{k} + \frac{k}{\kappa} \right)^2 \sh^2 \kappa a}
\end{equation}

\noindent
Коэффициент прохождения не равен нулю, то есть часть частиц проходит через потенциальный барьер. Это явление называется \textit{туннелированием}. Для барьера произвольной формы имеет место приближённая формула $D \approx \exp \left( - 2 \int_0^a \sqrt{\frac{2 m \left( U (x) - \mathcal{E} \right)}{\h^2}} \right)$

При альфа-распаде ядро выпускает ядро гелия, соответственно его массовое число уменьшается на 4, а заряд -- на 2. Для распада необходимо, чтобы энергия связи исходного ядра была меньше, чем сумма энергий связи альфа-частица и полученного ядра. Это условие выполняется для тяжёлый частиц.

Испускаемая альфа-частица оказывается в кулоновском потенциале, причём кулоновский потенциал значительно превышает энергию наблюдаемых в эксперименте частиц. Таким образом, при распаде происходит туннелирование альфа-частицы через кулоновский барьер.

Кулоновский потенциал описывается формулой $\frac{2 Z e^2}{r}$, где $Z$ -- заряд ядра. С учётом приближённой формулы для коэффициента прохождения

\begin{equation}
    D \approx \exp \left( - 2 \int_r^R \sqrt{\frac{2 m}{\h^2} \left( \frac{2 Z e^2}{r} - \mathcal{E}_\alpha \right)} dr \right) \approx \exp \left( - 4 \sqrt{\frac{4 m Z e^2}{\h^2}} \sqrt{R} \right)
\end{equation}

\noindent
где $r$ -- радиус ядра, порядка ферми, а $R = \frac{2 Z e^2}{\mathcal{E}_\theta}$. Отсюда $\ln D \propto \frac{Z}{\sqrt{\mathcal{E}_\alpha}}$. С другой стороны, период полураспада обратно пропорционален коэффициенту пропускания. Данное утверждение есть \textit{закон Гейгера-Нэттола}:

\begin{equation}
    \ln T_{1 / 2} = a \frac{Z}{\sqrt{\mathcal{E}_\alpha}} + b
\end{equation}
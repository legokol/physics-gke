\section{Тождественность частиц. Симметрия волновой функции относительно перестановки частиц. Бо-
зоны и фермионы. Принцип Паули. Электронная структура атомов. Таблица Менделеева.}

Поскольку в квантовой механике движение частиц описывается не траекторией, но волновой функцией, характеризующей лишь плотность вероятности, микрочастицы \textit{принципиально неразличимы}.

Рассмотрим два электрона в потенциальной яме в состояниях $a$ и $b$, с волновыми функциями $\Psi_a (x_1)$ и $\Psi_b (x_2)$ и поставим задачу о нахождении волновой функции $\Psi_{a b} (x_1, x_2)$, описывающей оба электрона. Поскольку частицы неразличимы, то квадрат модуля $\left| \Psi_{a b} \right|^2$, соответствующий плотности вероятности, не должен измениться при перестановке $x_1$ и $x_2$ (не должны измениться наблюдаемые эффекты). В частности, произведение функций $\Psi_a \Psi_b$ не удовлетворяет этому условию. Введём оператор перестановки $\hat{T}$. Поскольку плотность вероятности сохраняется, то $\hat{T} = e^{i \phi}$ -- не более чем фазовый множитель. Более того, $\hat{T}^2 = 1$ -- тождественный оператор, значит $\hat{T} = \pm 1$. Волновая функция пары неразличимых частиц имеет вид

\begin{equation}
    \Psi_{a b}^{(\pm)} (\b r_1, \b r_2) = \frac{1}{\sqrt{2}} \left[ \Psi_a (\b r_1) \Psi_b (\b r_2) \pm \Psi_a (\b r_2) \Psi_b (\b r_1) \right]
\end{equation}

\noindent
множитель $1 / \sqrt{2}$ необходим для сохранения нормировки. Волновая функция со знаком ''$+$'' чётна относительно перестановки, со знаком ''$-$'' -- нечётна. Чётность определяется типом частиц. \textit{Бозонами} или \textit{бозе-частицами} называются частицы, чётные относительно перестановки. Они обладают целым спином. Нечётные к перестановке частицы называются \textit{фермионами} или \textit{ферми-частицами}. Они обладают нецелым спином.

Заметим, что если волновая функция нечётна к перестановке, то для одинаковых состояний $\Psi_{a a} \equiv 0$, то есть фермионы не могут находиться в одном состоянии (плотность вероятности есть тождественный ноль). Это свойство называется \textit{принципом запрета Паули}.

С учётом принципа запрета формируется электронная структура атомов: невозможно существование двух электронов с одинаковыми квантовыми числами. В основе систематики таблицы элементов Менделеева лежит заряд ядра, причём свойства элемента зависят только от электронов внешней оболочки, с чем связана их периодичность.
\section{Квазистационарные токи. Свободные и вынужденные колебания в электрических цепях. Явление
резонанса. Добротность колебательного контура, ее энергетический смысл.}

Будем рассматривать колебательный контур в предположении \textit{квазистационарности} тока, то есть когда сила тока постоянна во всём контуре. Колебательным контуром называется схема, состоящая из последовательно соединенных конденсатора $C$, катушки индуктивности $L$ и сопротивления $R$. Возможно также подключение к контуру внешнего источника тока с переменной ЭДС.

Пусть $q$ -- заряд конденсатора. Тогда записав закон Ома для колебательного контура, получим:

\begin{equation}
    L \frac{d^2 q}{d t^2} + R \frac{d q}{d t} + \frac{q}{C} = \mathcal{E}
\end{equation}

\noindent
Это дифференциальное уравнение второго порядка. Если внешней ЭДС нет, то уравнения линейны и однородны относительно $q$. Такие уравнения описывают свободные колебания. Введём обозначения:

\begin{align*}
    \omega_0^2 &= \frac{1}{L C}, & 2 \gamma &= \frac{R}{L}, & \frac{\mathcal{E}}{C} &= X
\end{align*}

\noindent
Тогда уравнение можно переписать в виде

\begin{equation} \label{eq:уравнение колебательного контура}
    \ddot q + 2 \gamma \dot q + \omega_0^2 q = X
\end{equation}

Рассмотрим сначала свободные колебания: $X = 0$. В случае отсутствия сопротивления, уравнение \eqref{eq:уравнение колебательного контура} примет вид уравнения гармонического осциллятора, решением которого являются гармонические колебания без затуханий:

\begin{align}
    \ddot q + \omega_0^2 q = 0 \\
    q = q_0 \cos \left( \omega_0 t + \psi \right)
\end{align}

Теперь рассмотрим затухающие колебания. Уравнение \eqref{eq:уравнение колебательного контура} примет вид:

\begin{equation}
    \ddot q + 2 \gamma \dot q + \omega_0^2 q = 0
\end{equation}

Корни характеристического уравнения $\lambda = - \gamma \pm \sqrt{\gamma^2 - \omega_0^2}$. В зависимости от знака выражения под корнем, возможны три различных случая:

\begin{itemize}
    \item $\omega_0^2 - \gamma^2 > 0$

    В таком случае выражение под корнем отрицательно и решением будут затухающие колебания:

    \begin{equation}
        q = q_0 e^{- \gamma t} \cos \left( \omega t + \psi \right)
    \end{equation}

    \noindent
    где $\omega = \sqrt{\omega_0^2 - \gamma^2}$. При малых затуханиях $\gamma \ll \omega_0$ и частота практически совпадает с частотой незатухающих колебаний.

    \item $\omega_0^2 - \gamma^2 = 0$

    В этом случае решение уравнения имеет вид $q = (a + b t) e^{- \gamma t}$. Такой режим называется апериодическим и достигается при критическом сопротивлении $R_\text{кр} = 2 \frac{L}{C}$. В зависимости от начальных параметров величина $q$ (или другие параметры контура в силу одинаковых уравнений) может достигать или не достигать максимума.

    \item $\omega_0^2 - \gamma^2 < 0$

    В этом случае решение есть просто сумма двух экспоненциально затухающих решений: $q = C_1 e^{- \lambda_1 t} + C_2 e^{- \lambda_2 t}$, где $\lambda_{1, 2} = - \gamma \pm \sqrt{\gamma^2 - \omega_0^2}$.
\end{itemize}

Наибольший интерес представляет случай затухающих колебаний. Их амплитуда экспоненциально затухает, время, при котором амплитуда убывает в $e$ раз, называется временем затухания: $\tau = 1 / \gamma$. Отношение амплитуд в моменты колебаний, отличающиеся на период, определяется по формуле $A_1 / A_2 = e^{\gamma T}$. Величина

\begin{equation}
    d = \ln \frac{A_1}{A_2} = \gamma T
\end{equation}

\noindent
называется \textit{логарифмическим декрементом затухания}. Величина

\begin{equation}
    Q = \frac{\pi}{d} = \frac{\pi}{\gamma T} = \frac{\omega}{2 \gamma}
\end{equation}

\noindent
называется \textit{добротностью} колебательного контура. Она характеризует способность контура сохранять запасённую его энергию. Энергетический смысл добротности есть отношение энергии, запасённой в осцилляторе, к потерям её за период:

\begin{equation}
    Q = 2 \pi \frac{W_0}{\Delta W}
\end{equation}

Рассмотрим, наконец, случай вынужденных колебаний, когда к контуру подключено внешнее гармоническое напряжение $X = X_0 \cos \omega t$. Поскольку общее решение однородного уравнения, связанного с собственными колебаниями, затухает, мы будем интересоваться только частными решениями. Для нахождения частного решения удобно воспользоваться переходом к комплексным числам: $X = X_0 e^{i \omega t}$, а частное решение искать в виде $q = q_0 e^{i \omega t}$. Тогда для уравнение \eqref{eq:уравнение колебательного контура} имеем:

\begin{align}
    \left( - \omega^2 + 2 i \omega \gamma + \omega_0^2 \right) e^{i \omega t} = X_0 e^{i \omega t} \\
    q = \frac{X_0}{\omega_0^2 - \omega^2 + 2 i \omega \gamma} e^{i \omega t}
\end{align}

\noindent
Это решение комплексное, причём мнимая часть амплитуды влияет на сдвиг фазы. Выделяя вещественную часть, получим для решения

\begin{align}
    q = q_0 \cos \left( \omega t - \delta \right) \\
    q_0 = \frac{X_0}{\sqrt{\left( \omega^2 - \omega_0^2 \right)^2 + 4 \omega^2 \gamma^2}} \\
    \tg \delta = \frac{2 \omega \gamma}{\omega_0^2 - \omega^2}
\end{align}

\noindent
Таким образом, максимум амплитуды достигается при совпадении частоты внешней ЭДС с собственной частотой контура. Это явление называется резонансом. Оно соответствует  Кривая, изображающая зависимость амплитуды от частоты вблизи собственной частоты контура, называется резонансной кривой. Помимо амплитуды в максимуме часто важна \textit{ширина резонансной кривой} -- разность между частотами, на которых энергия колебаний вдвое меньше энергии в максимуме (амплитуда меньше в $\sqrt{2}$ раз соответственно):

\begin{equation*}
    \left( \omega_{1, 2}^2 - \omega_0^2 \right)^2 = 4 \omega_{1, 2}^2 \gamma^2
\end{equation*}

\noindent
Если частоты лежат близко к резонансной, то приближённо имеем:

\begin{equation}
    \Delta \omega = 2 \gamma = \omega_0 / Q
\end{equation}
\section{Принцип относительности Галилея и принцип относительности Эйнштейна. Преобразования Лоренца. Инвариантность интервала.}

Рассмотрим задачу о равномерном движении ИСО $S'$ относительно ИСО $S$ (рис. \ref{fig:исо}).

\begin{figure}[htbp]
    \centering
    \input{images/2_ИСО.pdf_tex}
    \caption{Относительно движение ИСО}
    \label{fig:исо}
\end{figure}

Пусть для простоты оси систем отсчёта $X, Y, Z,$ и $X', Y', Z'$ соответственно параллельны друг другу, а $X$ и $X'$ совпадают. Пусть также в начальный момент времени $t = 0$ точки $O$ и $O'$ совпадают, а $\b V$ -- относительная скорость движения $S'$ направлена вдоль оси $X$. Переход к более общему случаю осуществляется с помощью переноса начала отсчёта и поворота координат.

Пусть некоторая точка в момент времени $t$ находится в положении $M$. За это время точка $O'$ сдвинется на $\b V t'$ Для положения имеем

\begin{align}
    \b r &= \b r' + \b V t', & t &= t'
\end{align}

\noindent
В проекции на оси это соотношение запишется следующим образом:

\begin{align} \label{eq:преобразования-галилея}
    x &= x' + V t', & y &= y', & z &= z', & t &= t'
\end{align}

\noindent
Формулы обратного преобразования имеют вид

\begin{align} \label{eq:обратные преобразования галилея}
    x' &= x - V t', & y' &= y, & z' &= z, & t' &= t
\end{align}

\noindent
\eqref{eq:преобразования-галилея} и \eqref{eq:обратные преобразования галилея} представляют собой решение задачи о преобразовании координат при движении одной из систем отсчёта. Они называются преобразованиями Галилея.

\begin{note}
    Расстояние $l = \sqrt{\Delta x^2 + \Delta y^2 + \Delta z^2}$ является инвариантом преобразования Галилея.
\end{note}

\noindent
Продифференцировав, получим

\begin{equation}
    \b{\dot r} = \b{\dot r'} + \b V \Leftrightarrow \b v = \b v' + \b V
\end{equation}

\noindent
Это соотношение представляет собой нерелятивистский закон сложения скоростей. При повторном дифференцировании получим

\begin{equation}
    \b a = \b a'
\end{equation}

Поскольку силы не зависят от системы отсчёта, то уравнения движения системы (второй закон Ньютона) инвариантны относительно преобразований Галилея. Это и есть принцип относительности: уравнения механики инвариантны относительно преобразований Галилея (отличие траекторий есть следствие отличия начальных условиях в разных СО). Иначе его можно сформулировать следующий образом: \textit{Законы природы, определяющие изменения состояния движения механический систем, не зависят от того, к какой из двух инерциальных систем отсчёта, движущихся одна относительно другой прямолинейно и равномерно, они относятся}.

Всё физические явления нельзя разделить на чисто механические и немеханические, поэтому имеет место принцип относительности Эйнштейна: \textit{Законы природы, определяющие изменения состояний физических систем, не зависят от того, к какой из двух инерциальных систем отсчёта, движущихся одна относительно другой прямолинейно и равномерно, они относятся}.

Полученные результаты имеют место в предположении об абсолютности времени и расстояния в разных системах отсчёта, но при возрастании скоростей эти предположения нарушаются. Оказывается, например, что скорость света абсолютна и не зависит от системы отсчёта, что противоречит нерелятивистскому закону сложения скоростей. Вернёмся к рассмотрению задачи о двух ИСО (рис. \ref{fig:исо}). В релятивистском случае имеют место преобразования Лоренца:

\begin{align} \label{eq:преобразования лоренца}
    x &= \frac{x' + V t'}{\sqrt{1 - \frac{V^2}{c^2}}}, & y &= y', & z &= z', & t &= \frac{t' + \frac{V}{c^2}x'}{\sqrt{1 - \frac{V^2}{c^2}}}
\end{align}

\noindent
Обратное преобразование:

\begin{align} \label{eq:обратные преобразования лоренца}
    x' &= \frac{x - V t'}{\sqrt{1 - \frac{V^2}{c^2}}} - V t', & y' &= y, & z' &= z, & t' &= \frac{t - \frac{V}{c^2}x}{\sqrt{1 - \frac{V^2}{c^2}}}
\end{align}

\noindent
Дифференцируя \eqref{eq:преобразования лоренца}, получим

\begin{align}
    dx &= \frac{dx' + V dt'}{\sqrt{1 - \frac{V^2}{c^2}}}, & dy &= dy', & dz &= dz', & dt &= \frac{dt' + \frac{V}{c^2}dx'}{\sqrt{1-\frac{V^2}{c^2}}}
\end{align}

\noindent
Отсюда, во-первых, можно получить формулу преобразования скорости

\begin{align} \label{eq:дифференциал преобразований лоренца}
    v_x &= \frac{dx}{dt} = \frac{v_x' + V}{1 + \frac{v_x'V}{c^2}}, & v_y &= v_y' \frac{\sqrt{1 - \frac{V^2}{c^2}}}{1 + \frac{v_x' V}{c^2}}, & v_z &= v_z' \frac{\sqrt{1 - \frac{V^2}{c^2}}}{1 + \frac{v_x' V}{c^2}}
\end{align}

\noindent
Во-вторых, из этих выражений следует инвариантность интервала -- расстояния между точкам в 4-мерном пространстве Минковского $(t, x, y, z)$, $\Delta S^2 = c^2 \Delta t^2 - \Delta x^2 - \Delta y^2 - \Delta z^2$. Рассмотрим приращение интервала между двумя малыми точками:

\begin{equation}
    ds^2 = c^2dt^2 - dx^2 - dy^2 - dz^2 = [\eqref{eq:дифференциал преобразований лоренца}] = ds'^2
\end{equation}

\noindent
Таким образом, интервал инвариантен относительно преобразований Лоренца.
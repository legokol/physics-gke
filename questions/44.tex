\section{Квантовая природа света. Внешний фотоэффект. Уравнение Эйнштейна. Эффект Комптона.}

\textit{Фотоэффектом} называется явление взаимодействия вещества со светом или другим излучением, при котором из вещества выбиваются электроны. Фотоэффект называется \textit{внешним}, если электроны покидают вещество. Имеют место законы фотоэффекта:

\begin{enumerate}
    \item \textit{Максимальный фототок пропорционален интенсивности}
    \item \textit{Максимальная энергия фотоэлектронов зависит только от частоты падающего света}
    \item \textit{Существует красная граница фотоэффекта: при частоте меньшей некоторой критической фотоэффект не происходит}
\end{enumerate}

Так же фотоэффект \textit{безинерционен}: он начинается сразу же при освещении вещества и прекращается при исчезновении излучения. Описать фотоэффект с помощью волновой теории света невозможно. Но если представить свет пучком квантов (фотонов), то каждый квант имеет энергию $\h \omega$ и фотоэффект происходит при поглощении электроном фотона. Красная граница объясняется некоторым минимальным количеством энергии, которое нужно сообщить электрону, чтобы выбить его из вещества. Этот процесс описывается уравнением Эйнштейна:

\begin{equation}
    \h \omega = A + \frac{m v^2}{2}
\end{equation}

\noindent
Где величина $A$ называется работой выхода.

Другим физическим эффектом, демонстрирующим квантовую природу света, является \textit{Эффект Комптона}, заключающийся в изменении длины волны света при рассеянии на свободном электроне. Пусть фотон с длиной волны $\lambda_0$ налетает на покоящийся свободный электро, а после они разлетаются под некоторыми углами. Пусть $\theta$ -- угол отклонения фотона. Воспользуемся законом сохранения 4-импульса, где 4-импульс это

\begin{equation*}
    \b P = \begin{pmatrix}
        E \\
        p_x c \\
        p_y c \\
        p_z c
    \end{pmatrix}
\end{equation*}

Будем рассматривать только две оси. Тогда для эффекта Комптона:

\begin{equation}
    \begin{pmatrix}
        \h \omega_0 \\
        \h \omega_0  \\
        0
    \end{pmatrix}
    +
    \begin{pmatrix}
        m_e c^2 \\
        0 \\
        0
    \end{pmatrix}
    =
    \begin{pmatrix}
        \h \omega \\
        \h \omega \cos \theta \\
        \h \omega \sin \theta
    \end{pmatrix}
    + \b P_e
\end{equation}

\noindent
В полученном выражении перенесём новый импульс фотона в левую часть и возведём в квадрат. Учитывая, что квадрат 4-импульса есть энергия покоя в квадрате, получим

\begin{equation*}
    \left( m c^2 \right)^2 = \left( m c^2 \right)^2 + 2 \h \omega_0 m c^2 - 2 \h \omega m c^2 - \left( 2 \h \omega_0 \h \omega - 2 \h \omega_0 \h \omega \cos \theta \right)
\end{equation*}

\noindent
Учитывая, что длина волны $\lambda = 2 \pi c / \omega$, получим

\begin{equation}
    \lambda - \lambda_0 = \Lambda_e \left( 1 - \cos \theta \right)
\end{equation}

\noindent
где $\Lambda = \frac{h}{m_e c}$ -- комптоновская длина волны электрона.

Заметим, что поглощение фотона свободным электроном невозможно, тогда из законов сохранения следовало бы, что $\h \omega = 0$. Поглощение в случае фотоэффекта происходит за счёт того, что часть импульса передаётся ядру, с которым связан электрон.
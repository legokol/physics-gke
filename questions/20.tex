\section{Броуновское движение, закон Эйнштейна-Смолуховского. Связь диффузии и подвижности (соотношение Эйнштейна).}

\begin{definition}
    Броуновское движение -- хаотическое движение макроскопических частиц в среде (газе или жидкости), вызванное случайными ударами молекул среды.
\end{definition}

Рассмотрим движение шарообразной частицы в жидкости. При равномерном движении со скоростью $V$ на неё будет действовать сила сопротивления $F$, пропорциональная этой скорости. Коэффициент пропорциональности в формуле

\begin{equation}
    V = B F
\end{equation}

\noindent
называется \textit{подвижностью частицы}. В частности, для шарообразной частицы радиуса $a$

\begin{equation}
    B = \frac{1}{6 \pi \eta a}
\end{equation}

\noindent
где $\eta$ -- вязкость жидкости.

Уравнение движения Броуновской частицы в направлении оси $X$ имеет вид

\begin{equation*}
    M \ddot x = - \frac{\dot x}{B} + X
\end{equation*}

\noindent
Здесь первое слагаемое -- трение со стороны жидкости, а второе -- сила взаимодействия с молекулами жидкости, хаотично сталкивающимися с броуновской частицей. Ввиду хаотичности $\overline{X} = 0$. Домножим обе части этого уравнения на $x$ и учтём, что

\begin{align*}
    \frac{d}{dt} x^2 &= 2 x \dot x, & \frac{d^2}{dt^2} = 2 \dot x^2 + 2 x \ddot x
\end{align*}

\noindent
Получим

\begin{equation*}
    M \frac{d^2}{dt^2} x^2 + \frac{1}{B} \frac{d}{dt} x^2 - 2 M \dot x^2 = 2 X x
\end{equation*}

\noindent
Будем отсчитывать координату от начального положения частицы при $t = 0$. Усредним это уравнение. Ввиду хаотичности движения имеем $\langle X x \rangle = 0$. В тепловом равновесии $\langle M \dot x^2 \rangle = k T$. Тогда

\begin{equation}
    M \frac{d^2}{dt^2} \langle x^2 \rangle + \frac{1}{B} \frac{d}{dt} \langle x^2 \rangle - 2 k T = 0
\end{equation}

\noindent
Решение этого дифференциального уравнения имеет вид:

\begin{equation*}
    \overline{x^2} = x_0^2 + 2 k T B t + C \exp \left( - \frac{t}{M B} \right)
\end{equation*}

\noindent
Или, при больших временах,

\begin{equation}
    \overline{x^2} = x_0^2 + 2 k T B t
\end{equation}

\noindent
При обобщении на трёхмерный случай кинетическая энергия будет в три раза больше, тогда

\begin{equation}
    \overline{\b{r}^2} = r_0^2 + 6 k T B t
\end{equation}

Рассмотрим множество частиц, находящихся в постоянном и однородном силовом поле, где мы пренебрегаем силами взаимодействия между этими частицами (множество броуновских частиц в жидкости, идеальный газ). Пусть $\b F$ -- сила, действующая на частицу в силовом поле. Тогда потенциальная энергия $\epsilon_p = - F x$ (ось $X$ направлена вдоль действия силы). Если состояние стационарно, а температура постоянна, то концентрация меняется в соответствии с распределением Больцмана:

\begin{equation}
    n = n_0 \exp \left( - \frac{\epsilon_p}{k T} \right) = n_0 \exp \left( \frac{F x}{k T} \right)
\end{equation}

Поскольку концентрация не постоянна, возникает диффузия диффузия, причём $j_D = - D \frac{dn}{dx}$. С другой стороны есть поток частиц, вызванный силовым полем. Поскольку $\b u = B \b F$, то $j_F = B F n$. В состоянии равновесия эти потоки равны друг другу:

\begin{equation*}
    -D \frac{dn}{dx} + B F n = 0
\end{equation*}

\noindent
С учётом распределения Больцмана получим

\begin{equation}
    D = k T B
\end{equation}

\noindent
Это соотношение между диффузией и подвижностью частицы называется \textit{соотношением Эйнштейна}.
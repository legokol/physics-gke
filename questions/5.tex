\section{Закон всемирного тяготения и законы Кеплера. Движение тел в поле тяготения.}

Кеплер эмпирически установил три закона движения планет, основываясь на результатах многолетних анстрономических наблюдений. Их формулировки выглядят следующим образом:

\begin{enumerate}
    \item \textit{Каждая планета движется по эллипсу, в одном из фокусов которого находится Солнце.}
    \item \textit{Радиус-вектор планеты за равные промежутки времени заметает равные площади.}
    \item \textit{Квадраты времён обращений планет относятся как кубы больших полуосей эллиптических орбит, которые они описывают вокруг солнца.}
\end{enumerate}

На основе этих законов Ньютон получил закон всемирного тяготения:

\begin{equation} \label{eq:закон всемирного тяготения}
    \b F = - G \frac{m_1 m_2}{r^3} \b r
\end{equation}

Покажем вывод второго и третьего законов Кеплера из закона всемирного тяготения. Поскольку сила притяжения центральна, то на тело не действует никаких моментов сил, а значит имеет место закон сохранения момента импульса:

\begin{equation*}
    \b L = \left[ \b r, m \b v \right] = \text{const}
\end{equation*}

\begin{equation*}
    \frac{dS}{dt} = \frac{r v_\phi}{2} = \frac{|\left[ \b r, \b v_\phi \right]|}{2} = \text{const}
\end{equation*}

Введём потенциальную энергию $U = - G \frac{M m}{r}$. Для движения в поле тяжести имеет место закон сохранения энергии:

\begin{equation*}
    \frac{m v^2}{2} - G \frac{M m}{r} = \epsilon = \text{const}
\end{equation*}

\noindent
Рассмотрим крайние точки эллипса. В них скорость имеет только поперечную составляющую. С учётом сохранения момента импульса имеем для них

\begin{equation*}
    \frac{L^2}{2 m r^2} - G \frac{M m}{r} = \epsilon
\end{equation*}

\noindent
Это квадратное уравнение и по теореме Виета сумма его корней равна

\begin{equation*}
    r_1 + r_2 = - G \frac{M m}{\epsilon} = 2a \Rightarrow \epsilon = - G \frac{M m}{2 a}
\end{equation*}

\begin{figure}[htbp]
    \centering
    \input{images/3_эллипс.pdf_tex}
    \caption{Траектория движения тела в поле тяготения}
    \label{fig:эллипс}
\end{figure}

Заметим теперь, что сумма расстояний от фокусов до точки на эллипсе постоянна. Таким образом двойное расстояния от фокуса $2 CB = 2a \Rightarrow CB = a$. Отсюда получим скорость в точке $B$ (рис. \ref{fig:эллипс}):

\begin{equation*}
    \frac{m v_B^2}{2} - G \frac{M m}{a} = \epsilon = - G \frac{M m}{2 a} \Rightarrow v_B = \sqrt{\frac{G M}{a}}
\end{equation*}

\noindent
Наконец, рассчитаем период обращения:

\begin{equation*}
    S = \pi a b = \frac{L}{2 m} T \Rightarrow T = \frac{2 \pi a b m}{L} = \frac{2 \pi a}{v_B} \Rightarrow T^2 \propto a^3
\end{equation*}

\noindent
Что и доказывает третий закон Кеплера.
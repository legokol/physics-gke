\section{Пространственное фурье-преобразование в оптике. Дифракция на синусоидальных решетках. Теория Аббе формирования изображения.}

Одной из задач волновой оптики является решения задач дифракции -- расчёт волновых полей за различным препятствиями. Одним из способов решения этих задач является метод Гюйгенса-Френеля, где поле представляется как суперпозиция сферических волн. Альтернативным является \textit{метод Рэлея}, при котором волновое поле рассматривается как суперпозиция плоских волн.

Если рассматривать волновое поле вида $E(\b r, t) = E(\b r) e^{-i \omega t}$, то его амплитуда должна удовлетворять уравнению Гельмгольца.

\begin{equation}
    \Delta E + k^2 E = 0, \; k^2 = \frac{\omega^2}{c^2}
\end{equation}

\noindent
Частным решением этого уравнения является плоская волна $A = A_0 e^{i \b k \b r}$, $\b k^2 = k^2$.

Пусть поле задано в плоскости $z = 0$:

\begin{equation}
    E |_{z = 0} = E_0 e^{i k_x x + i k_y y}
\end{equation}

\noindent
Необходимо найти поле при $z > 0$. Уравнению Гельмгольца с таким граничным условиям удовлетворяет функция

\begin{equation}
    E = E_0 e^{i \left( k_x x + k_y y + k_z z \right)}
\end{equation}

\noindent
где $k_z = \sqrt{k^2 - K_x^2 - k_y^2}$. Это и есть решение поставленной задачи.

В более общем виде функцию $E$ можно разложить в ряд Фурье или интеграл Фурье, то есть

\begin{equation}
    E |_{z = 0} = \sum_n E_n e^{i \left( k_{x n} x + k_{y n} y \right)}
\end{equation}

\noindent
В таком случае решение уравнения Гельмгольца имеет аналогичный вид для каждой составляющей, то есть

\begin{equation}
    E = \sum_n E_n e^{i \left( k_{x n} x + k_{y n} y + k_{z n} z \right)}
\end{equation}

\noindent
где $k_{z n} = \sqrt{k^2 - k_{x n}^2 - k_{y n}^2}$. Задача сводится к разложению в ряд или интеграл Фурье исходного волнового поля.

Метод Рэлея хорошо подходит для решения задач дифракции на периодических структурах. В качестве примера рассмотрим амплитудную синусоидальную решётку с функцией пропускания $T = T_0 \left( 1 + m \cos \Omega x \right)$, где $b = 2 \pi / \Omega$ -- период решётки. Пусть на неё нормально падает волна $E_0$. Тогда в плоскости решётки поле имеет вид

\begin{equation}
    E(x) = E_0 T_0 \left( 1 + m \cos \Omega x \right) = E_0 T_0 \left( 1 + m \frac{e^{i \Omega x} + e^{- i \Omega x}}{2} \right)
\end{equation}

\noindent
Для поля за решёткой ($z > 0$) имеем:

\begin{equation}
    E(x, z) = E_0 T_0 e^{i k z} + E_0 T_0 m e^{i \sqrt{k^2 - \Omega^2} z} \cos \Omega x
\end{equation}

\noindent
где $k$ -- волновое число падающей волны.

Как было сказано ранее, в методе Рэлея волновое поле представляется суперпозицией плоских волн, то есть осуществляется пространственное преобразование Фурье. Пространственное разделение фурье-гармоник можно осуществить, в частности, с помощью линзы: волны, направленные под одним углом, собираются в одну точку в фокальной плоскости линзы. В связи с этим её называют \textit{фурье-плоскостью}.

Рассмотрим теорию Аббе формирований изображений. Изображение с помощью одиночной линзы формируется в два этапа. Сначала линза осуществляет пространственное преобразование Фурье и в фокальной плоскости линзы формируется \textit{первичное изображение}.  Затем свет распространяется до сопряжённой плоскости, в которой формируется изображение предмета (\textit{вторичное изображение}). Распространение от фокальной до сопряжённой плоскости соответствует обратному преобразованию Фурье.
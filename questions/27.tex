\section{Система уравнений Максвелла в интегральной и дифференциальной формах. Ток смещения. Материальные уравнения.}

Основные уравнения, описывающие электромагнитное поле в неподвижных средах, были установлены Максвеллом. К общим законам электродинамики отнесём теоремы Гаусса для электрического и магнитного поля а так же теоремы о циркуляции. Однако, теорему о циркуляции магнитного поля необходимо дополнить. Рассмотрим закон сохранения электрического заряда (уравнение непрерывности):

\begin{equation}
    \frac{\partial \rho}{\partial t} + \div \b j = 0
\end{equation}

\noindent
Теорема о циркуляции магнитного поля в дифференциальной форме имеет вид

\begin{equation}
    \rot \b H = \frac{4 \pi}{c} \b j
\end{equation}

\noindent
Поскольку дивергенция ротора всегда равна нулю, то это уравнение не может быть верным (оно верно только для стационарных токов). Если продифференцировать по времени теорему Гаусса для электрической индукции, получим

\begin{equation*}
    \frac{\partial \rho}{\partial t} = \frac{1}{4 \pi} \div \b{\dot D}
\end{equation*}

\noindent
или с учётом уравнения непрерывности

\begin{equation}
    \div \cdot \left( \b j + \frac{1}{4 \pi} \b{\dot D} \right) = 0
\end{equation}

\noindent
Где величина

\begin{equation}
    \b j_\text{см} = \frac{1}{4 \pi} \b{\dot D}
\end{equation}

\noindent
называется \textit{током смещения}, который и необходимо учесть в теореме о циркуляции напряжённости магнитного поля. Таким образом, получим систему уравнений Максвелла в интегральной форме:

\begin{align}
    \oint_S \b D d \b S &= 4 \pi q \\
    \oint_S \b B d \b S &= 0 \\
    \oint_L \b E d \b l &= - \frac{1}{c} \int_S \frac{\partial \b B}{\partial t} d \b S \\
    \oint_L \b H d \b l &= \frac{4 \pi}{c} \int_S \left( \b j + \frac{1}{4 \pi} \frac{\partial \b D}{\partial t} \right) d \b S
\end{align}

\noindent
и в дифференциальной форме:

\begin{align}
    \div \b D &= 4 \pi \rho \\
    \div \b B &= 0 \\
    \rot \b E &= - \frac{1}{c} \frac{\partial \b B}{\partial t} \\
    \rot \b H &= \frac{4 \pi}{c} \b j + \frac{1}{c} \frac{\partial \b D}{\partial t}
\end{align}

Однако эти уравнения не описывают электромагнитное поле полностью, система не доопределена. Их необходимо дополнить соотношениями, характеризующими свойства среды. Эти соотношения называются \textit{материальными уравнениями}. Они определяются из теорий поляризации, намагничивания и проводимости среды. В простейшем случае изотропных сред эти уравнения будут иметь вид

\begin{align}
    \b D &= \epsilon \b E \\
    \b B &= \mu \b H \\
    \b j &= \lambda \b E
\end{align}

\noindent
где $\epsilon, \mu, \lambda$ -- диэлектрическая проницаемость, магнитная проницаемость и электрическая проводимость среды.
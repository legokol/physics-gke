\section{Корпускулярно-волновой дуализм. Волны де Бройля. Опыты Девиссона-Джермера и Томсона по
дифракции электронов.}

В таких явлениях, как фотоэффект или эффект Комптона, свет проявляет квантовую природы, их невозможно объяснить с помощью волновой теории. Интерференцию или дифракцию, напротив, можно описать только с использованием волновой оптики. В этом заключается \textit{корпускулярно-волновой дуализм}: в некоторых явлениях свет ведёт себя как волна, а в некоторых -- как поток частиц.

\textit{Гипотеза де Бройля} заключается в том, что корпускулярно-волновой дуализм характерен не только для света, но и для других материальных частиц. Длина волны описывается формулой, аналогичной формуле для фотона:

\begin{equation}
    \lambda = \frac{h}{p}
\end{equation}

Обычные макроскопические объекты имеют слишком малую длину волны, чтобы как-то их наблюдать. Однако если рассматривать, например, электроны, то их длина волны де Бройля может быть порядка нескольких \AA. Для таких размеров можно попытаться наблюдать дифракцию на кристаллических решётках, аналогично дифракции рентгеновских лучей.

В \textit{опытах Девиссона-Джермера} параллельный пучок электронов, разгоняемых до одинаковой скорости, направлялся на монокристалл никеля. Рассеянные электроны затем улавливались коллектором, положение которого в пространстве можно было изменять. Таким образом была получена зависимость интенсивности от угла рассеяния, имеющий характерный максимум. При использовании же поликристаллической пластинки никеля, содержащей беспорядочно расположенные кристаллы, никакого преимущественного направления не возникало, что свидетельствовало о дифракции электронов (дифракция рентгеновский лучей носит интерференционный характер, максимум определяется условием Брэгга-Вульфа).

Другим опытом, подтверждающим гипотезу де Бройля, стал \textit{опыт Томпсона}. Пучок высокоскоростных электронов пропускался через поликристаллическую фольгу толщиной $\sim 10^{-5}$ см. После прохождения на фотопластинке за фольгой возникало пятно, окружённое дифракционными кольцами. Чтобы подтвердить, что наблюдалась дифракция именно электронов, а не возникающего рентгеновского излучения, такой же опыт проводился при наличии магнитного поля, что приводило к смещению и искажению дифракционной картины.

Таким образом, гипотеза де Бройля была подтверждена экспериментально и корпускулярно-волновой дуализм проявляется не только в поведении света.
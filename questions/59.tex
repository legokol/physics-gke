\section{Ядерные реакции. Составное ядро. Сечение нерезонансных реакций. Закон Бете.}

\textit{Ядерные реакции} -- процессы, возникающие в результате взаимодействия сложных атомных ядер или элементарных частиц. Примером ядерных реакции являются упругие рассеяния частиц, при которых не происходит возбуждения или образования новых частиц, а осуществляется лишь передача импульса и энергии.

Ключевым понятием является \textit{сечение реакции}. Рассмотрим поток частиц $j$, попадающих на мишень. Мишень содержит $N = n S dx$ частиц, где $n$ -- концентрация, $S$ -- площадь, а $dx$ -- толщина слоя. Тогда количество реакций в единицу времени будет пропорционально потоку налетающих частиц и числу частиц в мишени:

\begin{equation}
    \frac{dN}{dt} = j \left( n S dx \right) \sigma
\end{equation}

\noindent
где коэффициент $\sigma$ имеет размерность площади и называется сечением реакции.

Описывать ядерные реакции можно с помощью \textit{модели составного ядра}. При столкновении частицы с ядром-мишенью сначала образуется нестабильное составное ядро, которое через некоторое время распадается на продукты реакции.

\textit{Примечание: здесь и далее под $\lambda$ следует понимать длину волны де Бройля, делённую на $2 \pi$, то есть $\lambda = \dfrac{\hbar}{p}$.}

В рамках этой модели можно оценить сечение реакции: $\sigma = \pi (\lambda + R)^2 D(\mathcal{E})$, где $\pi (\lambda + R)^2$ -- геометрическая площадь, а $D$ -- коэффициент пропускания, характеризующий вероятность проникновения частицы в ядро. Представляя ядро как потенциальную яму с энергией $- U_0$, значительно превышающей энергию частицы $\mathcal{E}$, имеем для коэффициента пропускания $D = \frac{4 k k'}{\left( k + k' \right)^2} \approx 4 \frac{k}{k'} = 4 \frac{\mathcal{E}}{U_0}$. Поскольку при малых энергиях так же $\lambda \gg R$, имеем для сечения реакции

\begin{equation}
    \sigma \approx 4 \pi \lambda^2 \sqrt{\frac{\mathcal{E}}{U_0}} = \frac{2 \pi h^2}{m} \frac{1}{\sqrt{\mathcal{E} U_0}}
\end{equation}

\noindent
Из такой оценки следует, что $\sigma \propto \frac{1}{v}$. Обратная пропорциональность сечения реакции и скорости называется \textit{законом Бете}.
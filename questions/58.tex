\section{Виды бета-распадов. Объяснение непрерывности энергетического спектра электронов распада.
Нейтрино.}

Бета-распад -- это процесс изменения зарядового числа ядра при сохранении массового (превращение протона в нейтрон или наоборот). Возможны несколько типов бета-распада::

\begin{itemize}
    \item Электронный $\beta$-распад: ядро испускает электрон, заряд увеличивается на единицу.
    \item Позитронный $\beta$-распад: ядро испускает позитрон, заряд уменьшается на единицу.
    \item $K$-захват: ядро поглощает один из электронов своей оболочки, заряд уменьшается на единицу. Обычно поглощается электрон ближайшей оболочки, то есть $K$-оболочки, с чем и связано его название.
\end{itemize}

Для протекания распадов необходимо, чтобы масса исходного элемента была меньше массы возникающих. В частности, масса нейтрона несколько больше массы электрона, в связи с чем свободные нейтроны имеют среднее время жизни около 15 минут, а затем происходит электронный $\beta$-распад.

Особенностью $\beta$-распада является непрерывный энергетический спектр наблюдаемых электронов или позитронов. В случае, например, $\alpha$-распада, частицы имеют вполне определённое значение, полностью определяемое законами сохранения энергии и импульса. Если бы при $\beta$-распаде испускался только электрон или позитрон, то его состояние тоже должно быть полностью определено. Но непрерывность спектра позволила предположить, что вместе с ними выпускается ещё одна частица, нейтральная и слабо взаимодействующая с веществом. Тогда распад удовлетворяет закону сохранения энергии, а случайность распределения энергии между электроном (или позитроном) и этой частицей объясняет непрерывный спектр распад. При этом частица не должна иметь магнитного момента, поскольку она не ионизирует вещества, и обладать полуцелым спином, поскольку спин электрона или позитрона равен $1 / 2$. Частица была названа \textit{нейтрино} (электронное нейтрино в в позитронном распаде и электронное антинейтрино в электронном).
\section{Волновая функция. Операторы координаты и импульса. Средние значения физических величин.
Соотношение неопределенности для координаты и импульса. Уравнение Шредингера.}

Одним из определяющих принципов квантовой механики является \textit{принцип неопределённости Гейзенбера}: неопределённости в значении координаты и импульса должны удовлетворять соотношению $\Delta p \Delta x \geq \h$. В классической механике сами величины достаточно велики, что позволяет описывать движение траекторий, задавая положение в пространстве и скорость в каждый момент времени, однако в случае микромира такой описание становится невозможным. В квантовой механике состояние частицы описывается \textit{волновой функцией} $\Psi (x, t)$, физический смысл которой определяется следующим образом:

\begin{equation}
    \int_a^b \Psi^* (x) \Psi (x) d x = \int_a^b \left| \Psi (x) \right|^2 d x = w (a < x < b)
\end{equation}

\noindent
То есть \textit{квадрат модуля волновой функции есть плотность вероятности нахождения частицы в данной точке пространства}. Важным является принципе суперпозиции волновых функций: если $\Psi_a$ и $\Psi_b$ -- волновые функции состояний $a$ и $b$, то для суперпозиции этих состояний $\Psi = \alpha \Psi_a + \beta \Psi_b$.

В случае плоской монохроматической волны волновая функция имеет вид

\begin{equation}
    \Psi (x, t) = A e^{i \left( k x - \omega t \right)}
\end{equation}

Определение физических величин в квантовой механике реализуется с помощью \textit{операторов физический величин}. Всякое измерение влияет на состояние системы, так что действие такого оператора имеет вид

\begin{equation}
    \hat{A} \Psi = C \Psi'
\end{equation}

\noindent
При этом существуют, разумеется, \textit{собственные функции} таких операторов, для которых верно

\begin{equation}
    \hat{A} \Psi_n = A_n \Psi_n
\end{equation}

\noindent
Постоянная $C$ или $A_n$ при этом соответствуют измеренной величине. Средние значения определяется при этом как

\begin{equation}
    \left< A \right> = \int \Psi^* \hat{A} \Psi dx
\end{equation}

Рассмотрим операторы координаты и импульса. Для плоской волны импульс равен $\b p = \h \b k$. Учитывая, что волновая функция есть $A e^{i \left( \b k \b r - \omega t \right)}$, определим оператор импульса

\begin{equation}
    \hat{\b p} = - i \h \nabla
\end{equation}

\noindent
Оператор координаты есть просто

\begin{equation}
    \hat{x} = x
\end{equation}

Если по определению дисперсии случайной величины рассчитать дисперсии импульса и координаты, то для них так же будет получено соотношение неопределённостей:

\begin{equation}
    \Delta p_x^2 \Delta x^2 \geq \frac{\h^2}{4}
\end{equation}

Задачей квантовой механики является нахождение зависимости волновой функции от координат и времени -- аналог нахождения траектории в классической механике. Для этого необходимо решение \textit{уравнения Шрёдингера}:

\begin{equation}
    i \h \frac{\partial \Psi}{\partial t} = \hat{H} \Psi
\end{equation}

\noindent
где $\hat H$ -- оператор Гамильтона (оператор энергии). Учитывая, что кинетическая энергия есть $\frac{p^2}{2 m}$, то оператор кинетической энергии -- $\frac{\hat{\b p}^2}{2 m}$. Если потенциальная энергия равна $U (\b r)$, то

\begin{equation}
    \hat{H} = - \frac{\h^2}{2 m} \Delta + U(\b r)
\end{equation}

\noindent
Заметим, что если $\Psi$ -- собственная функция оператора Гамильтона, то плотность вероятности получаемого решения не зависит от времени. Тогда стационарное уравнение Шрёдингера имеет вид

\begin{equation}
    \hat{H} \Psi = \mathcal{E} \Psi
\end{equation}
\section{Спонтанное и вынужденное излучение. Инверсная заселенность уровней. Принцип работы лазера.}

Рассмотрим двухуровневую систему, где энергия второго уровня больше, чем первая. \textit{Спонтанным излучением} называется произвольный переход системы в состояние с меньшей энергией, при котором излучается фотон. При наличии внешнего излучения, энергия фотонов которого равна разнице энергий между уровнями, возможно также \textit{вынужденное или индуцированное излучение}, когда под действием фотона испускается такой же фотон.

Благодаря индуцированным переходам можно добиться усиления падающего излучения, но для этого нужно чтобы число электронов на уровне с большей энергией $n_2$ было больше числа электронов в основном состоянии $n_1$. Такое состояние называется \textit{инверсной заселённостью}.

Пусть $n_0 = n_1 + n_2$ -- общее число электронов, $N$ -- число падающих фотонов, $w_\text{инд}$ -- вероятность индуцированного перехода а $w_\text{сп}$ -- спонтанного. Если $n_2^0$ -- число частиц на верхнем уровне в равновесии при отсутствии внешнего излучения, то имеем

\begin{equation}
    \frac{d n_2}{d t} = n_1 N w_\text{инд} - n_2 N w_\text{инд} - \left( n_2 - n_2^0 \right) w_\text{сп}
\end{equation}

\noindent
Учтём, что в равновесии $n_2^0 = n_0 e^{- \h \omega / k T} / \left( 1 + e^{- h \omega / k T} \right)$. Тогда стационарное решение

\begin{equation}
    n_2 = n_0 \frac{N w_\text{инд} + \dfrac{e^{-\h \omega / k T}}{1 + e^{- \h \omega / k T}} w_\text{сп}}{2 N w_\text{инд} + w_\text{сп}} < \frac{n_0}{2}
\end{equation}

\noindent
Таким образом, в создать инверсную заселённость в двухуровневой системе невозможно.

Принцип работы лазера основан на создании системы с инверсной заселённостью. Простейшая схема -- резонатор (два зеркала), между которыми находится активная среда с тремя уровнями энергии. Активная среда обладает следующими свойствами:

\begin{itemize}
    \item Малое время жизни на третьем уровне, вероятность перехода $3 \to 2$ велика
    \item Большое время жизни на втором уровне
\end{itemize}

\noindent
Резонатор при этом настроен на длину волны соответствующую переходу $2 \to 1$. Таким образом, при накачке лазера частотой перехода $1 \to 3$ возникает инверсная заселённость первых двух уровней и излучение на частоте перехода $1 \to 2$ лавинообразно усиливается. С помощью такой схемы возможна генерация или усиление излучение на соответствующей частоте.
\section{Принцип Гюйгенса-Френеля. Зоны Френеля. Дифракция Френеля и Фраунгофера. Границы применимости геометрической оптики.}

\begin{definition}
    Дифракцией называется отклонение распространения света от законов геометрической оптики.
\end{definition}

Примером дифракции является проникновение света в зону геометрической тени. Ключевым в описании дифракционных явлений является \textit{принцип Гюйгенса-Френеля}: каждая точка волнового фронта является источником вторичной сферической волны, при том источники вторичного волнового фронта когерентны и интерферируют. Количественным описанием этого принципа является интеграл Френеля:

\begin{equation}
    E = \int_\Sigma K(\phi) E (\b r) \frac{e^{i k \rho}}{\rho} d\sigma
\end{equation}

\noindent
Где интеграл берётся по поверхности волнового фронта $\Sigma$, а $K(\phi)$ - множитель, зависящий от угла наблюдения, стремящийся к нулю при углах близких к $90^\circ$.

Для вычисления этого интеграла удобно разбить волновой фронт на зоны Френеля. Если, например, рассмотреть фронт сферической волны, то первая зона Френеля будет строиться таким образом, чтобы расстояние от точки наблюдения до этой зоны будет $r + \lambda / 2$, где $r$ -- расстояние от точки наблюдения до сферы. Таким образом, соседние зоны Френеля находятся в противофазе друг к другу. Так, например, если открыты первые две зоны Френеля, интенсивность в точке наблюдения будет минимальной.

Рассмотрим, например, дифракцию на круглом отверстии. Точные расчёты показывают, что для этой задачи радиус $m$-ой зоны Френеля равен $r_m = \sqrt{\lambda z m}$ где $z$ -- расстояние до точки наблюдения. В зависимости от числа открытых зон Френеля можно выделить несколько случаев:

\begin{itemize}
    \item $m \gg 1$ -- открыто очень большое число зон Френеля. Это случай геометрической оптики. Для размеров имеем $R \gg \lambda$, то есть характерные размеры значительно превышают лину волны.
    \item $m \sim 1$ -- открыто несколько зон Френеля. Этот случай называется дифракцией Френеля.
    \item $m < 1$ -- открыто меньше одной зоны Френеля. Реализуется на больших расстояниях (на бесконечности). Особенность дифракции Фраунгофера в том, что она \textit{подобна}, то есть световое поле есть функция угла. В связи с этим для наблюдения дифракции Фраунгофера используют линзы (бесконечность <<переносится>> в фокальную плоскость линзы).
\end{itemize}
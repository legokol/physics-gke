\section{Закон вязкого течения жидкости. Формула Пуазейля. Число Рейнольдса, его физический смысл.}

В реальных (неидеальных) жидкостях помимо сил давления действуют касательные силы вязкости. Для получения количественных законов вязкости рассмотрим следующий пример: пусть слой жидкости заключён между двумя бесконечно длинными параллельными пластинками. Нижняя пластинка пусть неподвижна, а верхняя движется относительно неё со скоростью $\b v_0$. Оказывается, что для поддержания равномерного движения верхней пластинки к ней надо будет приложить силу $\b F$ направленную в сторону движения. Такая же сила, противоположная по направлению, должна быть приложена к нижней пластинке для сохранения неподвижности. Ньютоном было установлено, что эта сила пропорциональна площади пластинки $S$, скорости $v_0$, и обратно пропорциональна расстоянию между пластинками $h$:

\begin{equation} \label{eq:закон вязкости ньютона}
    F = \eta S \frac{v_0}{h}
\end{equation}

Постоянная $\eta$ называется вязкостью жидкости и зависит только от жидкости, но не от материала пластинок. Обобщение этой формулы можно получить, разбивая движущуюся жидкость на тонкие слои. Результатом будет формула для касательного напряжения:

\begin{equation}
    \tau_{yx} = \eta \frac{\partial v_x}{\partial y}
\end{equation}

Рассмотрим стационарное течение вязкой несжимаемой жидкости по цилиндрической прямолинейной трубе радиуса $R$. Если выделить бесконечно маленькую трубку тока, то из несжимаемости жидкости будет следовать постоянство скорости на этой линии тока, так что скорость течения $v$ является только функцией радиуса $r$. Рассмотрим теперь тонкий цилиндр жидкости толщины $dx$ и радиуса $r$. На него действует касательная сила вязкости

\begin{equation*}
    2 \pi r \eta \frac{dv}{dr} dx
\end{equation*}

\noindent
Так же на него действует сила разности давлений

\begin{equation*}
    \left( P(x) - P(x + dx) \right) S = - \pi r^2 \frac{dP}{dx} dx
\end{equation*}

\noindent
При стационарном течении скорость постоянна, а значит сумма этих сил должна быть равна нулю:

\begin{equation*}
    2 \eta \frac{dv}{dr} = r \frac{dP}{dx}
\end{equation*}

\noindent
Скорость $v(r)$, а значит и её производная не зависят от $x$, а значит производная $\frac{dP}{dx}$ является постоянной и будет равна

\begin{equation*}
    \frac{dP}{dx} = \frac{P_2 - P_1}{l}
\end{equation*}

\noindent
Где $l$ -- длина трубы, а $P_1$ и $P_2$ -- давления на входе и выходе соответственно. Таким образом получаем формулу

\begin{equation}
    \frac{dv}{dr} = -  \frac{P_1 - P_2}{2 \eta l} r
\end{equation}
\noindent
При интегрировании получим

\begin{equation*}
    v = C - \frac{P_1 - P_2}{4 \eta l} r^2
\end{equation*}

\noindent
Постоянная интегрирования определяется из условия равенства нулю скорости на стенках трубы. Таким образом, скорость вязкой жидкости при удалении от оси меняется по параболическому закону:

\begin{equation}
    v = \frac{P_1 - P_2}{4 \eta l} \left( R^2 - r^2 \right)
\end{equation}

Рассчитаем теперь расход жидкости при таком течении. Масса жидкости, протекающая через кольцо с внутренним радиусом $r$ и внешним $r + dr$ равна $dQ = 2 \pi r dr \cdot \rho v$. Тогда

\begin{equation*}
    Q = \pi \rho \frac{P_1 - P_2}{2 \eta l} \int_0^R \left( R^2 - r^2 \right) r dr
\end{equation*}

\begin{equation} \label{eq:формула Пуазейля}
    Q = \pi \rho \frac{P_1 - P_2}{8 \eta l} R^4
\end{equation}

\noindent
Последняя формула и называется формулой Пуазейля. Заметим, что она верна только для \textit{ламинарных} течений, то есть тех, в которых слои жидкости не перемешиваются между собой и трубки тока параллельны трубе. При больших скоростях ламинарное течение становится турбулентным и формула \eqref{eq:формула Пуазейля} перестаёт быть верной. Тип течения характеризуется числом Рейнольдса

\begin{equation}
    Re = \frac{\rho l v_0}{\eta}
\end{equation}

Число Рейнольдса пропорционально отношению кинетической энергии жидкости к энергии, теряемой из-за вязкости на характерной длине. Оно определяет относительную роль инерции и вязкости жидкости. При больших числах преимущественна инерция, при малых -- вязкость. При больших числах Рейнольдса течение обычно турбулентное, при малых -- ламинарное, однако возможны исключения.
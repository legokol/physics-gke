\section{Интерференция волн. Временная и пространственная когерентность. Соотношение неопределенностей.}

Рассмотрим точку пространства, в которой пересекаются два пучка света. Напряжённость электрического поля в этой точке равна их сумме: $\b E = \b E_1 + \b E_2$. Рассчитаем интенсивность:

\begin{equation*}
    I = \overline{\b E^2} = \overline{\b E_1^2} + \overline{\b E_2^2} + 2 \overline{\left( \b E_1, \b E_2 \right)} = I_1 + I_2 + I_{12}
\end{equation*}

\noindent
Таким образом в общем случае суммарная интенсивность не равна сумме интенсивностей падающих волн. Слагаемое $I_{12}$ учитывает взаимодействие пучков света. Если источники света независимы, то оно обратится в нуль. Такие источники света называются \textit{некогерентными}.

Пусть падающие волны имеют в этой точке вид $\b E_1 = \b E_{0 1} \cos (\omega_1 t + \phi_1)$, $\b E_2 = \b E_{0 2} \cos (\omega_2 t + \phi_2)$. Тогда для части скалярного произведения, зависящей от времени, имеем

\begin{equation*}
    \cos (\omega_1 t + \phi_1) \cos (\omega_2 t + \phi_2) = \frac{1}{2} \left( \cos ((\omega_1 + \omega_2)t + \phi_1 + \phi_2) + \cos ((\omega_1 - \omega_2)t + \phi_1 - \phi_2) \right)
\end{equation*}

\noindent
Видно, что если частоты волн отличаются, то при усреднении этот член обратится в ноль. Таким образом, интерференция возможна только в случае одинаковых частот. \textit{Когерентность} означает, что разность фаз остаётся постоянной во времени. Одной из характеристик интерференционной картины является \textit{видность} $V = \frac{I_{max} - I_{min}}{I_{max} + I_{min}}$, характеризующая контрастность.

Одной из наиболее классических интерференционных схем является схема Юнга (рис. \ref{fig:схема Юнга}): два когерентных источника $S_1$, и $S_2$ создаются источником $S$ с помощью щелей. Разность фаз в точке наблюдения будет равна $k \Delta$, где $\Delta$ -- оптическая разность хода (с учётом показателя преломления). Если расстояние между щелями много меньше расстояния до экрана, то $\Delta \approx \frac{D x}{L}$. Для схемы Юнга интенсивность равна

\begin{equation}
    I = I_0 \left( 1 + \cos k \frac{D x}{L} \right)
\end{equation}

\noindent
где $I_0$ -- интенсивность источников $S_1$ и $S_2$. Видность интерференционной картины в схеме Юнга равна 1.

\begin{figure}[htbp]
    \centering
    \input{images/35_схема_Юнга.pdf_tex}
    \caption{Схема Юнга}
    \label{fig:схема Юнга}
\end{figure}

Рассмотрим случай немонохроматичного источника. Пусть, сначала, он излучает на двух длинах волн $\lambda_1$ и $\lambda_2$ с одинаковой интенсивностью. Интерференционные картины для каждой длины волны возникнут независимо друг от друга в силу некогерентности, соответственно для суммарной интенсивности имеем:

\begin{equation*}
    I = 2 I_0 (1 + \cos k_1 \Delta) + 2 I_0 (1 + \cos k_2 \Delta)
\end{equation*}

\noindent
Пользуясь тригонометрической формулой для суммы косинусов, получим

\begin{equation}
    I = 4 I_0 ( 1 + \cos \frac{k_1 + k_2}{2} \Delta + \cos \frac{k_1 - k_2}{\Delta})
\end{equation}

\noindent
Второй косинус осциллирует намного медленнее первого и определяет видность $V = \left| \cos \frac{k_1 - k_2}{2} \Delta \right|$.

Пусть теперь интенсивность распределена в некотором интервале $\delta k$: $I_0 = J_0 \delta k$. Тогда для интенсивности на экране имеем $dI = 2 d I_0 \left( 1 + \cos k \Delta \right)$, $d I_0 = J_0 d k$. Проинтегрируем:

\begin{equation*}
    I = \int_{k_0 - \delta k / 2}^{k_0 + \delta k / 2} 2 J_0 \left( 1 + \cos k \Delta \right) d k = \left. 2 J_0 \left( k + \frac{\sin k \Delta}{\Delta} \right) \right|_{k_0 - \delta k / 2}^{k_0 + \delta k / 2}
\end{equation*}
\noindent
С учётом тригонометрической формулы для разности синусов получим

\begin{equation}
    I = 2 I_0 \left( 1 + \frac{\sin \frac{\delta k}{2} \Delta \cos k_0 \Delta}{\frac{\delta k}{2} \Delta} \right)
\end{equation}

\noindent
В этом случае видность равна

\begin{equation}
    V = \left| \frac{\sin \frac{\delta k}{2} \Delta}{\frac{\delta k}{2} \Delta} \right|
\end{equation}

Теперь рассмотрим случае протяжённого источника в случае схемы Юнга (рис. \ref{fig:пространственная когерентность}).

\begin{figure}[htbp]
    \centering
    \input{images/35_пространственная_когерентность.pdf_tex}
    \caption{Схема Юнга с протяжённым источником}
    \label{fig:пространственная когерентность}
\end{figure}

Каждая точка источника излучает независимо, на неё приходится интенсивность $d I_0 = \frac{I_0}{a} d \xi$. Разность хода складывается из двух частей: разности хода до щелей и после. Тогда

\begin{equation*}
    d U = 2 d I_0 \left( 1 + \cos k \left( \frac{\xi D}{l} + \frac{x D}{L} \right) \right)
\end{equation*}

\noindent
Суммарная интенсивность

\begin{equation*}
    I = \int_{- a / 2}^{a / 2} 2 \frac{I_0}{a} \left[ 1 + \cos k \left( \frac{\xi D}{l} + \frac{x D}{L} \right) \right] d \xi = \left. \frac{2 I_0}{a} \left( \xi + \frac{\sin k \left( \frac{\xi D}{l} + \frac{x D}{L} \right) }{\frac{k D}{l}} \right) \right|_{- a / 2}^{a / 2}
\end{equation*}

\noindent
С учётом выражения для разности синусов

\begin{equation}
    I = 2 I_0 \left( 1 + \frac{\sin \frac{k a D}{2 l} \cos \frac{k x D}{l}}{\frac{k a D}{2 l}} \right)
\end{equation}

\noindent
Видность:

\begin{equation}
    V = \left| \frac{\sin \frac{k a D}{2 l}}{\frac{k a D}{2 l }} \right|
\end{equation}

\noindent
Источники, размеры и расположение которых позволяют наблюдать интерференционную картину, называются \textit{пространственно когерентными.}

Рассмотрим прямоугольный импульс амплитуды $A$ длительностью $\tau$ и найдём его спектр. Преобразование Фурье:

\begin{equation*}
    a(\omega) = \int_{- \tau / 2}^{\tau / 2} A e^{-i \omega t} d t = A \frac{\sin \frac{\omega \tau}{2}}{\omega / 2}
\end{equation*}

\noindent
Эта величина убывает с ростом $\omega$ и первый раз обращается в ноль при $\omega \tau = 2 \pi$. Таким образом характерная спектральная ширина волнового пакета

\begin{equation}
    \Delta \omega \tau = 2 \pi
\end{equation}

\noindent
Полученное соотношение называется \textit{соотношением неопределённостей}.
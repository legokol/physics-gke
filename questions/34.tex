\section{Плазма. Плазменная частота. Диэлектрическая проницаемость плазмы. Дебаевский радиус.}

\begin{definition}
    Плазмой называется частично или полностью ионизированный квазинейтральный газ. Квазинейтральность означает, что количества положительных и отрицательных зарядов почти одинаковы в любом выделенном объёме.
\end{definition}

Как поведение плазмы во внешних электромагнитных полях (на которые обычный газ не реагирует) так и её механические свойства значительно отличаются от свойств газа из-за электромагнитного взаимодействия между частицами. Свойства плазмы в целом в значительной степени определяются именно электродинамикой.

Выделем в плазме объём в форме параллелепипеда, оси координат направим вдоль его сторон. Будем считать, что заряд ионов равен единицы. В силу квазинейтральности концентрации ионов и электронов в этом объёме совпадают: $n_e = n_i$. Предположим, что ионы сместились на расстояние $x$ вдоль оси $O x$, а ионы остались неподвижными, поскольку их масса много больше массы электронов. В результате на краях параллелепипеда образуется поверхностный заряд

\begin{align}
    \sigma &= \pm \frac{q}{S} = \pm \frac{n_e e S x}{S} = \pm n_e e x \\
    E  &= 4 \pi \sigma = 4 \pi n_e e x
\end{align}

\noindent
Запишем уравнения движения электронов:

\begin{equation}
    m_e \ddot x = -e E = - 4 \pi n_e e^2 x = - \omega_p^2 m_e x
\end{equation}

\noindent
Мы получили уравнение колебаний. Частота этих колебаний $\omega_p = \sqrt{4 \pi n_e e^2 / m_e}$ называется \textit{плазменной частотой}. Колебания, совершаемые электронами в плазме относительно ионов называются \textit{плазменными колебаниями}.

Оценим амплитуду плазменных колебаний. Средняя скорость движения электронов есть средняя скорость теплового движения $\overline{v_e} \sim \sqrt{k T_e / m_e}$, амплитуда имеет порядок $r \sim \overline{v_e} / \omega_p$. Получим

\begin{equation}
    r_D = \frac{k T_e}{4 \pi n_e e^2}
\end{equation}

\noindent
Полученная величина называется \textit{дебаевским радиусом}. Дебаевский радиус это одна из основных характеристик плазмы: он характеризует характерный масштаб, на котором может существенно нарушаться квазинейтральность.

Если внести в плазму точечный заряд $q$, то из-за кулоновского взаимодействия произойдёт пространственное перераспределение частиц плазмы и внесённый заряд будет окружён зарядами противоположного знака. Потенциал внесённого заряда с учётом окружающих его быстро затухает на расстояниях порядка дебаевского радиуса, поэтому его также называют \textit{радиусом экранирования}. На расстояниях больших дебаевского радиуса внесённый заряд оказывают слабое влияние на плазму.

Рассмотрим падение электромагнитной волны на плазму. Ввиду тяжести ионов под действием излучения двигаются в основном электроны. Более того, скорость электронов обычно мала по сравнению со скоростью света $v \ll c$, поэтому можно пренебречь взаимодействием с магнитным полем по сравнению со взаимодействием с электрическим. Наконец, смещение электронов мало, поэтому мы учтём только зависимость электрического поля от времени. Уравнение движение электрона во внешнем поле имеет вид

\begin{equation*}
    m \ddot x = -e E = - e E_0 \cos \omega t
\end{equation*}

\noindent
Решение этого уравнения имеет вид

\begin{equation*}
    x = \frac{e E_0}{m \omega^2} \cos \omega t = \frac{e E}{m \omega^2}
\end{equation*}

Электроны слабо смещаются относительно положения равновесия и можно уподобить их связанным зарядам в диэлектрике. Тогда возникающий дипольный момент $p = -e x = - e^2 E / m \omega^2$. В векторном виде $\b p = - e^2 \b E / m \omega^2$. пусть $N$ -- концентрация электронов. Тогда для вектора поляризации имеем

\begin{equation}
    \b P = - N \b p = - \frac{N e^2}{m \omega^2} \b E = \alpha \b E
\end{equation}

\noindent
Учитывая, что диэлектрическая проницаемость плазмы равна $\epsilon = 1 + 4 \pi \alpha$, получим

\begin{equation}
    \epsilon = 1 - \frac{4 \pi N e^2}{m \omega^2} = 1 - \frac{\omega_p^2}{\omega^2}
\end{equation}

В отличие от диэлектриков диэлектрическая проницаемость плазмы меньше единицы, причем при частотах меньших плазменной она отрицательна: такие волны не проходят в плазму, они экспоненциально затухают при распространении в ней.
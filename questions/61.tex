\section{Деление ядер под действием нейтронов. Принцип работы ядерного реактора на тепловых нейтронах.}

\textit{Деление ядер} -- процесс расщепления атомного ядра на несколько (обычно два, реже три) осколков близкой массы. Возможно выделение других продуктов реакции, таких как $\gamma$-кванты, лёгкие ядра и нейтроны. Возможно спонтанное деление ядер и вынужденное, в результате взаимодействия с другими частицами, обычно нейтронами.

Ядро тяжёлого элемента при захвате нейтрона может быть стабильным, однако оно оказывается в сильно возбуждённом состоянии, что и приводит к делению на несколько частей. Деление тяжёлых ядер связано с выделением большого числа энергии, поскольку средняя энергия связи в тяжёлых ядрах меньше.

Другим важным свойством деления ядра является процесс образования вторичных нейтронов. Отношение числа нейтронов к числу протонов тем больше, чем тяжелее ядро. При делении ядра это отношение сохраняется осколках (либо в одном из них оно больше, чем в ядре), то есть хотя бы один из осколков оказывается нейтроноизбыточным и испускает их.

На основе деления ядер под действием нейтронов основан \textit{ядерный реактор}. В качестве примера будем рассматривать ядерный реактор на \textit{тепловых нейтронах}, использующий в качестве топлива $^{235}$U. Пусть в некоторый момент в реакторе появился нейтрон. Рано или поздно он будет захвачен одним из ядер урана-235, что приведёт к делению этого ядра и возникновением ещё двух нейтронов, нейтронов \textit{первого поколения}. Рано или поздно они так же будут захвачены ядрами топлива, что приведёт к появлению уже четырёх нейтронов \textit{второго поколения}. Таким образом в ходе реакции число нейтронов будет экспоненциально возрастать, а реакции деления ядер урана будут непрерывно идти с выделением большой энергии. Поскольку сечение реакции обратно пропорционально скорости (закон Бете), необходимы медленные нейтроны (тепловые имеют энергии порядка $k T \approx 0.025$ эВ при комнатной температуре). Поскольку нейтроны, возникающие в реакциях деления, оказываются быстрыми, их требуется замедлять. При этом замедляющая среда, с одной стороны, не должна захватывать нейтроны, а с другой должна иметь близкую к ним массу для наиболее эффективного обмена энергией. На практике обычно используется тяжёлая вода D$_2$O или графитовые блоки из углерода.

\noindent
Важнейшей характеристикой активной зоны ядерного реактора является \textit{коэффициент размножения} нейтронов $k$, равный отношение нейтронов в каком-либо поколении к породившему их числу нейтронов в предыдущем поколении. При $k = 1$ реакция деления стационарна и поддерживает сама себя, но если $k > 1$ то число нейтронов и реакций быстро экспоненциально возрастает, что может привести к неконтролируемой реакции и взрыву. При $k < 1$ реакции затухает.

Контролировать число $k$ можно, используя среду, хорошо поглощающую нейтроны (для этого в активную зону вводятся стержни из поглощающего материала). Время жизни вторичных нейтронов составляет $\sim 10$ мкс. Очевидно, что на таком промежутке времени технически невозможно контролировать реакцию, и при малейшем отклонении значения $k$ от единицы реакция либо затухнет, либо произойдёт взрыв реактора. Реализация ядерных реакторов оказывается возможной благодаря \textit{запаздывающим нейтронам}. Осколки деления ядер радиоактивны и при их распаде могут возникнуть нейтроноизбыточные состояние, порождающие запаздывающие нейтроны. Их число очень мало по отношению к общему числу вторичных нейтронов ($\sim 0.5 \%$). Если обеспечить число вторичных нейтронов, близкое к необходимому для поддержания реакции, но немного меньше, то регулировать скорость можно используя вторичные нейтроны, поскольку изменение их числа значительно медленнее меняет $k$.

При поддержании цепной реакции в активной зоне постоянно выделяется большое количество энергии, идущее на нагревание теплоносителя, передающего теплоту непосредственна в установку, вырабатывающую электричество.
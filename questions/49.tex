\section{Строение водородоподобного атома. Уровни энергии и кратность их вырождения. Спектр излучения атома водорода.}

В случаях центрального поля (например, кулоновского) оператор Гамильтона удобно расписать в сферических координатах. Стационарное уравнение Шрёдингера примет вид

\begin{equation}
    - \frac{\h^2}{2 m} \frac{1}{r^2} \frac{\partial}{\partial r} \left( r^2 \frac{\partial \Psi}{\partial r} \right) + U(r) \Psi + \frac{\h^2}{2 m r^2} \hat{l}^2 \Psi = \mathcal{E} \Psi
\end{equation}

\noindent
здесь $\hat{l}$ -- оператор момента импульса. В этом уравнении разделяются части, действующие на радиус и углы. Решение представимо в виде

\begin{equation}
    \Psi = \frac{\xi (r)}{r} Y_{l m} (\theta, \phi)
\end{equation}

\noindent
где $Y_{l m}$ -- сферическая функция, с помощью которых решается уравнение вида $\hat{l}^2 \Psi = A \Psi$. Энергия в таком зависит от двух \textit{квантовых чисел} $\mathcal{E} = \mathcal{E} (n_r, l)$, а решение (состояние) определяется числами $n_r$, $l$ и $m$.

В случае атома водорода потенциал имеет вид $U (r) = - \frac{e^2}{r}$. Точное решение уравнения Шрёдингера для атома водорода обладает следующими свойствами:

\begin{enumerate}
    \item Энергия определяется главным квантовым числом $n = n_r + l + 1$
    \item Случайное вырождение по орбитальному квантовому числу
    \item $l$ принимает целые значения от $0$ до $n - 1$,  $m$ -- от $- l$ до $l$
\end{enumerate}

Энергия состояний при этом определяется как 

\begin{equation}
    \mathcal{E}_n = - \frac{m e^4}{2 \h^2} \frac{1}{n^2}
\end{equation}

\noindent
где $Ry = \frac{m e^4}{2 \h^2} = 13.6$ эВ. Спектр излучения атома водорода определяется именно этой дискретностью уровней энергии, возможно только поглощение и излучение фотонов на частотах, удовлетворяющих

\begin{equation}
    \h \omega = Ry \left( \frac{1}{n^2} - \frac{1}{m^2} \right)
\end{equation}

\noindent
где $m$ и $n$ -- натуральные числа.

В случае, если рассматривается система двух тел, то можно перейти к таким же уравнение для относительного движение, заменив массу $m$ на приведённую. Для водородоподобных атомов решение аналогично с тем лишь отличием, что потенциал будет равен уже $ - \frac{Z e^2}{r^2}$, где $Z$ -- заряд. Имеем

\begin{equation}
    \mathcal{E}_n = - \frac{\mu^2 Z^2 e^4}{2 \h^2} \frac{1}{n^2}
\end{equation}
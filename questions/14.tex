\section{Распределения Максвелла и Больцмана.}

Поставим задачу о распределении частиц газа по скоростям. Будем рассматривать пространство скоростей. Пусть $d \omega$ -- вероятность вектора скорости в этом пространстве попасть в параллелепипед $dv_x dv_y dv_z$. Вероятность должна быть пропорционально объёму, то есть

\begin{equation*}
        d \omega = f(v) dv_x dv_y dv_z = \phi (v_x) \phi (v_y) \phi (v_z) dv_x dv_y dv_z
\end{equation*}

\noindent
Где мы считаем независимыми друг от друга события, выражающие попадания в соответствующие интервалы по разным осям. Пользуясь этим предположением и изотропностью пространства, можно получить \textit{распределение Максвелла}:

\begin{enumerate}
    \item Одномерный случай:
    \begin{equation}
        f_1 (v) = \sqrt{\frac{m}{2 \pi k T}} \exp \left( -\frac{m v^2}{2 k T} \right)
    \end{equation}
    \item Двумерный случай:
    \begin{equation}
        f_2 (v) = \frac{m}{2 \pi k T} \exp \left( -\frac{m v^2}{2 k T} \right)
    \end{equation}
    \item Трёхмерный случай:
    \begin{equation}
        f_3 (v) = \left( \frac{m}{2 \pi k T} \right)^{3/2} \exp \left( -\frac{m v^2}{2 k T} \right)
    \end{equation}
\end{enumerate}

Рассмотрим теперь распределение не по направлениям, а по величине скорости $F(v)$. С учётом полученных ранее результатов,

\begin{equation}
    F(v) dv = f_3(v) \cdot 4 \pi v^2 dv = 4 \pi \left( \frac{m}{2 \pi k T} \right)^{3/2} v^2 \exp \left( -\frac{m v^2}{2 k T} \right)
\end{equation}

Поставим теперь вопрос распределения частиц в пространстве. В отсутствие внешних сил в равновесии концентрация молекул газа будет всюду одинакова, однако ситуация изменится при добавлении внешних полей. Рассмотрим идеальный газ в поле тяжести. Выделим столб газа толщины $dz$ с основанием площадью $S$. Сила тяжести этого столба должна уравновешиваться разностью давлений:

\begin{equation*}
    \left( P_1 - P_2 \right) S = - \frac{dP}{dz} S dz = n m g S dz
\end{equation*}

\noindent
Отсюда

\begin{equation}
    \frac{dP}{dz} = - n m g
\end{equation}

\noindent
В идеальном газе $P = n k T$. В предположении, что температура газа всюду одинакова, получим

\begin{equation*}
    k T \frac{dn}{dz} = - n m g \Rightarrow k T d \ln n = - m g dz
\end{equation*}

\noindent
Таким образом, получаем \textit{барометрическую формулу}

\begin{equation} \label{eq:барометрическая формула}
    n = n_0 \exp \left( - \frac{m g h}{k T} \right)
\end{equation}

\noindent
\eqref{eq:барометрическая формула} обобщается на общий случай потенциальных сил. Результатом является \textit{распределение Больцмана}:

\begin{equation}
    n = n_0 \exp \left( - \frac{\epsilon_p}{k T} \right)
\end{equation}
\section{Эффект Зеемана в слабых магнитных полях.}

В отсутствие магнитного поля уровни энергии атома зависят только от главного квантового числа $n$, наличие спин-орбитального взаимодействия приводит к тонкой структуре. Однако в отсутствие внешних полей в пространстве нет выделенного направления и магнитное квантовое число не влияет уровни энергии. Однако при наличии внешнего магнитного поля $\b B$ появляется энергия взаимодействия с магнитным полем $- \left( \mathfrak{m}, \b B \right)$, приводящая к расщеплению уровней в зависимости от проекции магнитного момента. Это явление называется \textit{эффектом Зеемана}.

В случае слабых магнитных полей спиновое и орбитальное квантовые числа не является независимыми из-за спин-орбитального взаимодействия, а атом характеризуется полным моментом

\begin{equation}
    \hat{\b J} = \hat{\b L} + \hat{\b S}
\end{equation}

\noindent
имеющим постоянную длину и направление в пространстве. Магнитный момент в свою очередь складывается из орбитального и спинового магнитных моментов:

\begin{equation}
    \mathfrak{m} = \mu_\text{Б} \left( \b L + 2 \b S \right)
\end{equation}

\noindent
С другой стороны он определяется полным моментом

\begin{equation}
    \mathfrak{m} = g \mu_\text{Б} \b J
\end{equation}

Учитывая, что $\left( \b L, \b S \right) = \frac{J (J + 1) - L (L + 1) - S (S + 1)}{2}$ и $g \b J^2 = (\b L + 2 \b S) ( \b L + \b S)$, получим

\begin{equation}
    g = \frac{3}{2} + \frac{S (S + 1) - L (L + 1)}{2 J (J + 1)}
\end{equation}

\noindent
Величина $g$ называется \textit{множителем Ланде} и в слабых магнитных полях расщепление уровней определяется с учётом $(\mathfrak{m}, \b B) = \mathfrak{m}_z B$, где 

\begin{equation}
    \mathfrak{m_z} = g \mu_\text{Б} J_z
\end{equation}

Таким образом уровень энергии расщепляется на $2 J + 1$ равностоящих подуровней.
\section{Электрические флуктуации. Дробовой и тепловой шумы. Предел чувствительности электроизмерительных приборов.}

В электрических цепах существуют флуктуации тока и напряжения, проявляющиеся в различных измерениях. Эти флуктуации создают ложные сигналы на выходе усилителей и ограничивают чувствительность электрических приборов. Возникновение флуктуаций связано с дискретностью зарядов, осуществляющих перенос тока, и случайностью характера движения этих зарядов.

\subsection*{Дробовой шум}

\textit{Дробовым шумом} называются флуктуации, связанные с дискретностью зарядов и случайным характером их вылета и прибытия. Ток образуется электронами, покидающими катод. Импульсы тока, производимые отдельными зарядами, достигающими анода, можно считать одинаковыми. Но эти импульсы следуют случайным образом, то есть наблюдаемый ток есть последовательность затухающих импульсов. Если $i(t)$ -- форма импульса, то ток имеет вид:

\begin{equation}
    I(t) = \sum_j i \left( t - t_j \right)
\end{equation}

Из соотношения неопределённостей имеем: $\tau \Delta \nu \approx 1$, а для дисперсии числа зарядов $\overline{\Delta N^2} = N$. Учтём, что $N = q / e = I \tau /e$, где $I$ -- средний ток. Тогда для шумового тока имеем:

\begin{equation*}
    \overline{i^2} = \frac{e^2 \overline{\Delta N^2}}{\tau} = \frac{e I}{\tau} \approx I e \Delta \nu
\end{equation*}

\noindent
Это соотношение называется формулой Шотки.

\subsection*{Тепловой шум}

\textit{Тепловой шум} связан с тепловым движением зарядов. Он проявляется во флуктуациях разности потенциалов на концах проводника. Рассмотрим высокодобротный резонансный $LCR$-контур. Энергия, запасённая в контуре, состоит из электрической энергии в конденсаторе и магнитной в катушке:

\begin{equation*}
    W = \frac{L I^2}{2} + \frac{C U^2}{2}
\end{equation*}

\noindent
Если система имеет температура $T$, то по теореме о равнораспределении имеем:

\begin{align*}
    \overline{W_L} &= \frac{\overline{L I^2}}{2} = \frac{k T}{2}, & \overline{W_C} &= \frac{\overline{C U^2}}{2} = \frac{k T}{2}, & \overline{W} &= k T
\end{align*}

\noindent
Учтём теперь, что добротность контура выражается, с одной стороны, как $Q = 2 \pi W / \Delta W$, то есть $\Delta W = 2 \pi W / Q$. С другой стороны $W = k T$ и $Q = \omega_0 / \Delta \omega$, где $\omega_0$ -- резонансная частота, а $\Delta \omega$ -- ширина резонанса. Тогда для потери энергии за период имеем:

\begin{equation*}
    \Delta W = 2 \pi \frac{k T}{\omega_0} \Delta \omega
\end{equation*}

\noindent
Учтём, что $2 \pi / \omega_0 = T_0$ -- период колебаний системы, а $\Delta W / T_0$ -- мощность потерь, которая, с другой стороны, равна $P = I^2 R = \mathcal{E}^2 / R$. Тогда для флуктуаций имеем:

\begin{align}
    \overline{I^2} &= \frac{k T}{R} \Delta \omega, & \overline{\mathcal{E}^2} &= k T R \Delta \omega
\end{align}
\section{Магнитное поле постоянных токов в вакууме. Основные уравнения магнитостатики в вакууме. Закон Био-Савара. Сила Ампера. Сила Лоренца.}

Экспериментально установлено, что при наличии магнитного поля на движущийся заряд действует сила, которая определяется формулой

\begin{equation}
    \b F_m = \frac{q}{c} \left[ \b v, \b B \right]
\end{equation}

Здесь $\b B$ -- величина не зависящая заряда, а характеризующая лишь магнитное поле. Она называется \textit{магнитной индукцией}.

В электрическом поле на заряд также действует сила $\b F_e = \b E e$. Если предположить, что электрическое и магнитное поля действуют на заряд независимо друг от друга, то результирующая сила будет равна $\b F = \b F_e + \b F_m$ или

\begin{equation}
    \b F = q \left( \b E + \frac{1}{c} \left[ \b v, \b B \right] \right)
\end{equation}

\noindent
Суммарная сила называется \textit{силой Лоренца}.

Можно рассмотреть действие магнитного поля не на отдельную частицу, а на поток частиц. Пусть ток создается движением одинаковых зарядов $e$ с концентрацией $n$. Тогда $\b j = n e \b v$ -- плотность тока. Число частиц в объёме $dV$ будет $dN = n dV$ и тогда на этот объём действует сила

\begin{equation*}
    d \b F = \frac{e}{c} \left[ \b v, \b B \right] dN = \frac{n e}{c} \left[ \b v, \b B \right] dV = \frac{1}{c} \left[ \b j, \b B \right] dV
\end{equation*}

\noindent
Если рассмотреть участок бесконечно тонкого провода, по которому течёт ток $I$, а площадь которого равна $S$, то $dV = S dl$, причём вектор $d \b l$ совпадает по направлению с током, тогда имеем

\begin{equation}
    d \b F = \frac{I}{c} \left[ d \b l, \b B \right]
\end{equation}

\noindent
Сила, действующая на ток в магнитном поле, называется \textit{силой Ампера}.

Источником магнитного поля являются движущиеся заряды. Опытным путём был установлен закон, определяющий магнитное поле заряда, скорость которого много меньше скорости света:

\begin{equation}
    \b B = \frac{q}{c r^3} \left[ \b v, \b r \right]
\end{equation}

\noindent
Причём постоянная $c$ есть ни что иное, как скорость света. Стоит отметить, что как и электрическое поле, магнитное поле обратно пропорционально квадрату расстояния. Рассмотрим теперь поле, создаваемое не отдельным зарядом, а элементом тока. Аналогично предыдущим выкладкам, получим

\begin{equation}
    d \b B = \frac{1}{c} \frac{\left[ \b j, \b r \right]}{r^3} dV
\end{equation}

\noindent
Или для линейного элемента тока

\begin{equation}
    d \b B = \frac{I}{c} \frac{\left[ d \b l, \b r \right]}{r^3}
\end{equation}

\noindent
Эти формулы выражают закон Био-Савара. Суммарное поле же находится интегрированием по всем токам.

Прямым следствием закона Био-Савара является теорема Гаусса для магнитного поля:

\begin{align}
    \oint \b B d \b S &= 0, & \div \b B &= 0
\end{align}

Так же с помощью него можно доказать теорему о циркуляции магнитного поля:

\begin{equation}
    \oint \b B d \b s = \frac{4 \pi}{c} I = \frac{4 \pi}{c} \int \b j d \b S
\end{equation}

\noindent
где $I$ -- токи, обхватываемые контуром (в случае интеграла рассматривается распределение тока). Используя формулу Стокса можно получить дифференциальную форму теоремы о циркуляции:

\begin{equation}
    \rot \b B = \frac{4 \pi}{c} \b j
\end{equation}
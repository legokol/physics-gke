\section{Магнитное поле в веществе. Основные уравнения магнитостатики в веществе. Граничные условия для векторов B и H.}

Как было сказано ранее, источником магнитного поля являются токи, а магнитных зарядов не существует. Простейший виток с током можно рассматривать как магнитный диполь $\mathfrak{m}$.

В веществе магнитное поле может возбуждаться не только токами, но и движением заряженных частиц внутри самих атомов и молекул. Магнетизм веществ в основном обусловлен вращением электронов вокруг атомных ядер, собственным вращением (спином) электронов, или собственным вращением атомных ядер. По этим причинам возникают микроскопические поля, а их результирующее поле -- макроскопическое. В силу отмеченной ранее эквивалентности магнитного диполя витку с током, собственные магнитные диполи вещества эквивалентны \textit{молекулярным токам} $\b j_m$. Обычные токи, связанные с переносом зарядов, называются \textit{токами проводимости} $\b j$. В таком случае результирующее магнитное поле $\b B$ является суммой полей токов проводимости и молекулярных токов. Так, например, для теоремы о циркуляции

\begin{align}
    \oint \b B d \b s &= \frac{4 \pi}{c} \left( I + I_m \right), & \rot \b B = \frac{4 \pi}{c} \left( \b j + \b j_m \right)
\end{align}

Реакцию среды на внешнее магнитное поле характеризует \textit{вектор намагниченности} $\b I$. По аналогии с вектором поляризации

\begin{equation}
    \b I = n \mathfrak{m}
\end{equation}

\noindent
где $\mathfrak{m}$ -- средний дипольный момент молекулы вещества. Если рассмотреть цилиндр объёмом $V$, то, с одной стороны, его момент есть $\mathfrak{m} = \b I V$, а с другой $\mathfrak{m} = I_m \b S / c$, где $I_m$ -- молекулярный ток, текущий по поверхности. Если учесть, что $V = S L$ (для прямого цилиндра), для линейной плотности молекулярного тока получим

\begin{equation}
    i_m = c \b I
\end{equation}

\noindent
В общем случае необходимо рассматривать здесь проекцию на ось цилиндра $I_l = \left( \b I, \b l \right)$.

Теорема Гаусса для магнитного поля не меняет свой вид в веществе. Рассмотрим теорему о циркуляции. Произвольный контур окружим бесконечно тонкой трубкой. По её поверхности будет циркулировать ток с плотностью $i_m = c I_l$. Он будет пересекать произвольную поверхность, натянутую на контур, только один раз. На единицу длины трубки приходится ток $dI_m = i_m dl = c \left( \b I, d \b l \right)$. Для полного тока намагничивания имеем

\begin{equation}
    I_m = c \oint \b I d \b l
\end{equation}

\noindent
Подставляя это выражение в теорему о циркуляции, имеем

\begin{equation}
    \oint \left( \b B - 4 \pi \b I \right) d \b l = \frac{4 \pi}{c} I
\end{equation}

\noindent
А в дифференциальной форме

\begin{equation}
    \rot \b B = \frac{4 \pi}{c} \left( \b j + c \rot \b I \right)
\end{equation}

Эти формулы упрощаются, если ввести \textit{вектор напряжённости магнитного поля}:

\begin{equation}
    \b H = \b B - 4 \pi \b I
\end{equation}

\noindent
Тогда теорема о циркуляции принимает вид:

\begin{align}
    \oint \b H d \b l &= \frac{4 \pi}{c} I, & \rot H = \frac{4 \pi}{c} \b j
\end{align}

Аналогично полям $\b D$ и $\b E$ из этих уравнения можно получить граничные условия. Так, в силу теоремы Гаусса, нормальная компонента вектора магнитной индукции должна быть непрерывной:

\begin{equation}
    B_{1n} = B_{2n}
\end{equation}

А тангенциальная компонента вектора напряжённости может испытывать скачок:

\begin{equation}
    H_{2 \tau} - H_{1 \tau} = \frac{4 \pi}{c} i
\end{equation}
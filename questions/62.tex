\section{Соотношение неопределенностей для энергии и времени. Оценка времени жизни виртуальных частиц, радиусов сильного и слабого взаимодействий.}

Соотношение неопределённостей в волновой теории

\begin{equation}
    \Delta t \delta \omega \geq 2 \pi
\end{equation}

\noindent
Его смысл состоит в том, что локализованный во времени волновой процесс не является монохроматическим, а представим в виде волнового пакета с характерным разбросом частот $\Delta \omega$. Если домножить это формулу на постоянную Планка $\h$, то получим соответствующее \textit{соотношение неопределённостей для энергии и времени}:

\begin{equation}
    \delta t \Delta \mathcal{E} \geq h
\end{equation}

\noindent
Полученное соотношение связывает время существования какого-то состояния с его энергией.

Взаимодействие между двумя частицами может осуществляться через обмен другой частицей, передающей импульс и энергию. Если $\Delta \mathcal{E}$ -- энергия, переданная в результате взаимодействия, то можно определить радиус такого действия. Согласно соотношению неопределённостей, время жизни такой частицы имеет порядок $\tau \approx \frac{h}{\Delta E}$. В таком случае оценим радиус:

\begin{equation}
    L \sim c \tau \sim \frac{h c}{\Delta \mathcal{E}}
\end{equation}

Межнуклонное взаимодействие в ядре осуществляется $\pi$-мезонами, имеющими массу 140 МэВ. Рассчитав расстояние, получим величину порядка $10^{-15}$ м, или 1 фм -- характерные размеры ядер.

Слабое взаимодействие осуществляется $W$- и $Z$-бозонами, имеющими энергию порядка 90 ГэВ. Радиус слабого взаимодействия -- порядка $10^{-18}$ м.
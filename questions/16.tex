\section{Фазовые переходы. Уравнение Клапейрона-Клаузиуса. Диаграммы состояний.}

Ранее было получено условие равновесия двух фаз вещества -- равенство химических потенциалов

\begin{equation}
    \phi_1 = \phi_2
\end{equation}

Будем рассматривать процессы испарения и конденсации (переход между жидкой и газообразной фазами). Каждое состояние вещества можно изобразить точкой на плоскости $T, P$. Условие равенства химических потенциалов будет определять некоторую кривую на этой плоскости -- \textit{кривую равновесия фаз}. Можно изобразить этот график для трёх фаз: жидкой, твёрдой и газообразной. Этот график называется диаграммой состояний. Точка пересечения кривых равновесия называется \textit{тройной точкой}: в ней все три фазы могут существовать одновременно. Области, на которые кривые делят плоскость, соответствуют различным фазам.

Найдём наклон кривой равновесия. Имеем $d \phi_1 = d \phi_2$, или, если расписать дифференциал

\begin{equation*}
    v_1 dP - s_1 dT = v_2 dP - s_2 dT
\end{equation*}

\noindent
или

\begin{equation}
    \frac{dP}{dT} = \frac{s_1 - s_2}{v_1 - v_2}
\end{equation}

\noindent
Если считать процесс квазистатическим, то имеет место выражение $q = T ds$, где $q$ -- удельная теплота испарения, которая поглощается или выделяется системой при фазовом переходе. С учётом этого выражения имеем

\begin{equation} \label{eq:уравнение клапейрона-клаузиуса}
    \frac{dP}{dT} = \frac{q}{T \left( v_1 - v_2 \right)}
\end{equation}

\noindent
Уравнение \eqref{eq:уравнение клапейрона-клаузиуса} называется уравнением Клапейрона-Клаузиуса.
\section{Поверхностное натяжение. Формула Лапласа. Свободная энергия и внутренняя энергия поверхности.}

Известно, что на молекулу жидкости действуют силы притяжения со стороны окружающих её молекул. Если она удалена от поверхности и находится внутри жидкости, то эти силы в среднем уравновешиваются. Однако если молекула близка к поверхности, то на неё действует сила, направленная внутрь. С этим связано явление поверхностного натяжения: чтобы извлечь молекулу на поверхность, необходимо совершить работу.

\begin{definition}
    Работа, необходимая для увеличения поверхности жидкости на единицу площади в изотермическом квазистатическом процессе при сохранении объёма, называется поверхностным натяжением.
\end{definition}

Изотермическая работа равна убыли свободной энергии. Свободную энергию жидкости можно представить в виде

\begin{equation}
    \Psi = \Psi_\text{об} + \Psi_\text{пов}
\end{equation}

\noindent
Здесь $\Psi_\text{об}$ -- объёмная составляющая свободной энергии, а $\Psi_\text{пов}$ -- поверхностная. С учётом определения поверхностного натяжения имеем

\begin{equation}
    \Psi_\text{пов} = \sigma F
\end{equation}

\noindent
где $\sigma$ -- поверхностное натяжение, а $F$ -- площадь поверхности. Таким образом можно определить поверхностное натяжение \textit{ как свободную энергию, приходящуюся на единицу поверхности.}

Понятие поверхностного натяжения также связано с прикладываемой силой. Пусть сила $f$ действует на элемент поверхности длины $l$, смещая его на $\Delta x$. Тогда эта сила совершает работу $A = f \Delta x$, увеличивая площадь поверхности на $S = l \Delta x$. С учётом определения поверхностного натяжения получим

\begin{equation}
    \sigma = \frac{f}{l}
\end{equation}

Получим выражение для внутренней энергии поверхности. Поскольку при увеличении поверхности внешние силы совершают работу $\delta A = \sigma dF$, то сама плёнка совершает работу $- \sigma dF$. Таким образом

\begin{equation*}
    dU = T dS + \sigma dF
\end{equation*}

\noindent
Для свободной энергии

\begin{equation*}
    d \Psi = - S dT + \sigma dF
\end{equation*}

\noindent
отсюда $S = - \left( \partial \Psi / \partial T \right)_F$ а значит

\begin{equation}
    \Psi = U + T \left( \frac{\partial \Psi}{\partial T} \right)_F
\end{equation}

\noindent
Учтём, что $\Psi = \sigma F$. При этом поверхностное натяжение не зависит от площади поверхности, но зависит от температуры. Имеем

\begin{equation}
    U = \left( \sigma - T \frac{d \sigma}{dT} \right) F
\end{equation}

Если поверхность жидкости -- кривая, то из-за поверхностного натяжения давление по разные стороны поверхности будет различным. Рассмотрим случай, когда боковая поверхность -- цилиндр радиуса $R$ и длины $b$ (рис. \ref{fig:формула лапласа}).

\begin{figure}[htbp]
    \centering
    \input{images/18_Лаплас.pdf_tex}
    \caption{К выводу формулы Лапласа}
    \label{fig:формула лапласа}
\end{figure}

Выберем на его поверхности участок $AB$, стягиваемый центральным углом $\phi$. На его боковые стороны действуют касательные силы $b \sigma$. Их равнодействующая направлена к центру цилиндра и по величине равна $2 b \sigma \sin (\phi / 2)$. В силу малости угла $F = b \sigma \phi$.

Пусть $a$ -- длина дуги. Тогда $AB$. Тогда $\phi = a / R$, а $S = a b$ -- площадь элемента поверхности. Таким образом

\begin{equation*}
    F = \frac{\sigma}{R} S
\end{equation*}

\noindent
Наконец, разделив силу на площадь, найдём возникающую разность давлений:

\begin{equation}
    \Delta P = \frac{\sigma}{R}
\end{equation}

\noindent
Это и есть формула Лапласа. В случае поверхности двойной кривизны она имеет вид

\begin{equation}
    \Delta P = \sigma \left( \frac{1}{R_1} + \frac{1}{R_2} \right)
\end{equation}

\noindent
Так, например, для сферической поверхности

\begin{equation}
    \Delta P = \frac{2 \sigma}{R}
\end{equation}

\noindent
Если поверхности две (случай плёнки), то разность давлений увеличивается ещё в два раза.
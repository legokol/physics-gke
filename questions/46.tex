\section{Излучение абсолютно черного тела. Формула Планка, законы Вина и Стефана-Больцмана.}

\begin{definition}
    Абсолютно чёрным телом называется тело, поглощающее всё падающее на него излучение.
\end{definition}

Пусть $a(\omega) = \frac{W_\text{погл}}{W_\text{АЧТ, погл}}$, $b(\omega) = \frac{W_\text{изл}}{W_\text{АЧТ, изл}}$ -- спектральные поглощательная и испускательная способности тела, характеризующие насколько хорошо оно поглощает и излучает энергию в сравнении с абсолютно чёрным телом. Поместим АЧТ и некоторое произвольное тело в зеркальную оболочку, нагрев их до одинаковой температуры. Если предположить, что для второго тела $b(\omega) < a (\omega)$, то возникнет передача тепла между телами одинаковой температуры (а затем от менее нагретого к более нагретому) без совершения работы, что противоречит второму началу термодинамики. Таким образом $a(\omega) = b(\omega)$.

Поставим задачу о нахождении излучения АЧТ. Хорошей моделью АЧТ служит плоскость с маленьким отверстием и почти зеркальными стенками: излучение, попадающее внутрь, практически не выходит наружу, со временем поглощаясь. Через некоторое время излучение в полости придёт в равновесии со стенками. $u(\omega) = \frac{d E}{d V d \omega}$ -- спектральная плотность энергии внутри полости, $\rho (\omega) = \frac{d E}{d S d t d \omega}$ -- спектральная плотность излучения, то есть энергия, выделяемая элементом поверхности (отверстия) в единицу времени в интервале частот. Связь между величинами:

\begin{equation}
    \frac{d E}{d \omega} (\theta) = \frac{2 \pi \sin \theta d \theta}{4 \pi} d S \cos \theta c dt \cdot u (\omega)
\end{equation}

\noindent
Смысл этого выражения: часть волн, лежащих в телесном угле и цилиндрическом объёме, которая пройдёт через элемент поверхности. Проинтегрировав по всем углам, получим

\begin{equation}
    \rho (\omega) = \frac{c u (\omega)}{4}
\end{equation}

Будем рассматривать плоскость как большой прямоугольный резонатор, в котором возникает система стоячих волн. $d N / d \omega$ -- число мод в интервале частот. Предполагая, что размеры резонатора $L_x = L_y = L_z = L$ и с учётом условий существования волн $k_x = \frac{K \pi}{L_x}$, $k_y = \frac{K \pi}{L_y}$ и $k_z = \frac{K \pi}{L_z}$ получим

\begin{equation*}
    \omega^2 = c^2 \left( k_x^2 + k_y^2 + k_z^2 \right) = \frac{c^2 \pi^2}{L^2} \left( K^2 + M^2 + P^2 \right)
\end{equation*}

\noindent
Или

\begin{equation}
    K^2 + M^2 + P^2 = 4 L^2 \frac{\nu^2}{c^2}
\end{equation}

\noindent
Это уравнение сферы в координатах $K M P$. Каждая точка в этом пространстве (дискретном) соответствует одной из мод колебаний. Тогда число мод в сферическом слое равно его объёму (объём элемента пространства равен единице!) с учётом того, что нужно взять $1 / 8$ этого объёма, то есть только положительные числа, и домножить на два, поскольку на каждую моду приходится две поляризации. Получаем

\begin{equation}
    d N = 2 \cdot \frac{1}{8} 4 \pi R^2 dR = \left( \frac{2 L}{c} \right)^3 \pi \nu^3 d \nu \Rightarrow \frac{d N}{d \omega} = V \frac{\omega^2}{\pi^2 c^3}
\end{equation}

\noindent
Заметим, что если считать среднюю энергию на моду колебаний $E = k T$, то мы получим $u(\omega) \propto \omega^2$ -- закон Рэлея-Джинса, экспериментально верный при малых частотах, но приводящий к расходящемуся интегралу для полной энергии (ультрафиолетовой катастрофе). Согласно \textit{Гипотезе Планка} энергия квантуется, причём $E = \h \omega n$, где $n$ -- целое неотрицательное число, определяющееся распределением Больцмана. Тогда

\begin{equation}
    \overline{E} = \frac{\sum_{n = 0}^\infty n \h \omega e^{- n \h \omega / k T}}{\sum_{n = 0}^\infty e^{- \h \omega / k T}}
\end{equation}

\noindent
Интеграл в знаменателе считается как сумма геометрическая прогрессии, а числитель -- его производная по величине $- 1 / k T$. Тогда

\begin{equation}
    \overline{E ( \omega)} = \frac{\h \omega}{e^{\h \omega / k T} - 1}
\end{equation}

\noindent
С учётом выражения для числа мод, средней энергии и равенства $\rho = c u / 4$, получим \textit{формулу Планка}:

\begin{equation} \label{eq:формула-Планка}
    \rho (\nu) = \frac{2 \pi}{c^2} \frac{h \nu^3}{e^{h \nu / k T} - 1}
\end{equation}

Переход к зависимости спектральной плотности излучения от длины волны делается с помощью формулы $\rho (\lambda) = \rho (\omega) \frac{d \omega}{d \lambda}$. Плотности излучения имеет максимум на длине волны $\lambda_{max} = \frac{b}{T}$, где $b = 2.9$ мм$\cdot$К, эта зависимость от температуры называется \textit{законом смещения Вина}.

Наконец, интегрируя \eqref{eq:формула-Планка} по всем частотам, получим \textit{закон Стефана-Больцмана}:

\begin{equation}
    \rho = \frac{d E}{d S dt} = \sigma T^4
\end{equation}

\noindent
где $\sigma = 5.67 \cdot 10^{-8}$ Вт$/$(м$^2 \cdot$К).
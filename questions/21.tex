\section{Закон Кулона. Теорема Гаусса в дифференциальной и интегральной формах. Теорема о циркуляции для статического электрического поля. Потенциал. Уравнение Пуассона.}

Основным количественным законом электростатики является закон Кулона: \textit{сила взаимодействия точечных зарядов в вакууме прямо пропорциональна величинам их зарядов и обратно пропорциональна квадрату расстояния между ними. Это сила притяжения, если заряды различного знака, и отталкивания, если одного}. Математически это утверждение выглядит следующим образом:

\begin{equation}
    F = C \frac{q_1 q_2}{r_{12}^2}
\end{equation}

\noindent
Где $C$ -- некоторый числовой коэффициент. Его значение и размерность можно выбрать произвольно, получая таким образом различные системы единиц. Мы будем рассматривать систему единиц СГСЭ или Гауссову систему единиц, где $C = 1$ и не имеет размерности. В таком случае размерность заряда получается из закона Кулона, который выглядит следующим образом:

\begin{equation}
    F = \frac{q_1 q_2}{r_{12}^2}
\end{equation}

\noindent
Или в векторном виде

\begin{equation}
    \b F = \frac{q_1 q_2}{r_{12}^3} \b r_{12}
\end{equation}

Так же вводится понятие напряжённости электрического поля: \textit{сила, действующая на единичный положительный заряд, помещённый в электрическое поле}.

\begin{equation}
    E = \frac{F}{q}
\end{equation}

Например, напряжённость поля точечного заряда

\begin{equation}
    E = \frac{q}{r^2}
\end{equation}

\noindent
В векторном виде

\begin{equation}
    \b E = \frac{q}{r^3} \b r
\end{equation}

Для напряжённости электрического поля имеет место \textit{принцип суперпозиции}: напряжённость электрического поля нескольких неподвижных зарядов есть векторная сумма напряжённости полей каждого из них, взятого по отдельности

\begin{equation}
    \b E = \sum_i \frac{q_i}{r_i^3} \b r_i
\end{equation}

В физике часто применяется понятие \textit{потока вектора через площадку}. По определению это скалярное произведение вида $\b a \cdot \b S$, где $\b a$ -- векторное поле, а $\b S$ -- вектор нормали, длина которого равна площади площадки $S$. Если векторное поле не постоянно, то можно посчитать поток разбив площадку на малые элементы: $\int \left( \b a, d \b S \right)$. Далее будем рассматривать поток вектора напряжённости. Если поле $\b E$ есть суперпозиция (векторная сумма) нескольких полей, то в силу линейности скалярного произведения окажется, что поток суммарного поля есть алгебраическая сумма потоков каждого из элементарных полей.

Докажем основную теорему электростатики -- \textit{теорему Гаусса}. Она определяет поток вектора напряжённости электрического поля через произвольную замкнутую поверхность $S$. Положительной нормалью будем считать внешнюю. Рассмотрим для начала случай, когда поле создается одним точечным зарядом. Пусть поверхность $S$ -- сфера, а заряд помещён в центр. Тогда

\begin{equation*}
    d \Phi = \left( \b E, \b n \right) dS = \frac{q dS}{r^2}
\end{equation*}

\noindent
Интегрируя по поверхности сферы получим

\begin{equation} \label{eq:поток E}
    \Phi = 4 \pi q
\end{equation}

\noindent
Покажем, что этот результат не зависит от формы поверхности. Рассмотрим произвольную элементарную площадку $dS$ с нормалью $\b n$. Поток через неё

\begin{equation*}
    d \Phi = \left( \b E, \b n \right) dS = E dS \cos \alpha = E dS_r
\end{equation*}

Здесь $dS_r$ -- проекция площадки на плоскость, перпендикулярную радиусу $\b r$. Тогда $d \Phi = \frac{q dS_r}{r^2}$. По определению телесного угла имеем $dS_r = r^2 d \Omega$, тогда $d \Phi = q d \Omega$. Интегрируя по всей поверхности имеем $\Phi = q \Omega$, где $\Omega$ -- телесный угол, под которым видна поверхность из точки расположения заряда. Здесь стоит рассмотреть два случая

\begin{itemize}
    \item Если заряд лежит внутри поверхности, то телесный угол охватывает все направления и равен $4 \pi$. Получаем формулу \eqref{eq:поток E}.
    \item Если заряд лежит вне поверхности, то телесный угол равен нулю, поскольку прямая, направленная от заряда, будет пересекать поверхность чётное число раз, причём нормали будут иметь противоположные направления. Получим, что поток равен нулю.
\end{itemize}

В случае, если рассматривается система зарядов, с учётом принципа суперпозиции стоит рассматривать алгебраическую сумму зарядов. Таким образом получаем электростатическую теорему Гаусса в интегральной форме

\begin{equation}
    \Phi = \oint \b E d \b S = 4 \pi q
\end{equation}

\noindent
где $q$ -- алгебраическая сумма зарядов, охваченных поверхностью.

Используя теорему Гаусса-Остроградского из математического анализа, можно получить теорему Гаусса в дифференциальной форме:

\begin{equation*}
    \oint_S \b E d \b S = \iiint_V \div \b E dV = 4 \pi \iiint_V \rho dV
\end{equation*}

\noindent
где $\rho$ -- объёмная плотность заряда. Устремляя объём к нулю, получим

\begin{equation}
    \div \b E = 4 \pi \rho
\end{equation}

Рассмотрим теперь работу, совершаемую электрическим полем точечного заряда $Q$. Напряжённость электрического поля этого заряда равна $\b E = \frac{Q}{r^3} \b r$. Таким образом работа, совершаемая силами поля при перемещении из некоторого положения 1 в положение 2 равна

\begin{equation*}
    A_{12} = \int_{12} q \b E d \b r = q Q \int_{12} \frac{\b r d \b r}{r^3}
\end{equation*}

\noindent
Поскольку $\b r d \b r = r dr$, то криволинейный интеграл сводится к определённому

\begin{equation}
    A_{12} = q Q\int_{r_1}^{r_2} \frac{dr}{r^2} = q Q \left( \frac{1}{r_1} - \frac{1}{r_2} \right)
\end{equation}

\noindent
Таким образом, при любом перемещении заряда работа электрического поля не зависит от пути, а только от начальной и конечной точки, то есть поле точечного заряда потенциально. Из принципа суперпозиции следует потенциальность поля любой системы точечных зарядов, а поскольку любую систему зарядов можно аппроксимировать точечной, то электрическое поле всегда потенциально. В силу потенциальности имеет место теорема о циркуляции статического электрического поля:

\begin{equation}
    \oint \b E d \b s = 0
\end{equation}

\noindent
для любого замкнутого контура. Для потенциальных полей можно ввести понятие потенциала, как работы по перемещению единичного заряда из некоторой начальной точки. Например, для точечных зарядов удобно выбирать ноль потенциала на бесконечности. Тогда

\begin{equation}
    \phi = Q \int_{\infty}^r \frac{dr}{r^2} = \frac{Q}{r}
\end{equation}

\noindent
Работа при перемещении из точки 1 в точку 2 будет равна $A = q \left( \phi_1 - \phi_2 \right)$.

Найдём связь между потенциалом и напряжённостью электрического поля. Работа при бесконечно малом перемещении единичного заряда $d \b l = \left( dx, dy, dz \right)^T$ будет, с одной стороны, равна $- d \phi$, а с другой $\b E \cdot d \b l = E_x dx + E_y dy + E_z dz$. Приравнивая эти выражения, получим

\begin{equation}
    \b E = - \nabla \phi
\end{equation}

\noindent
Подставив это выражение в теорему Гаусса в дифференциальной форме, получим

\begin{equation}
    - \nabla \cdot \nabla \phi = - \nabla^2 \phi = - \Delta \phi = 4 \pi \rho
\end{equation}

\noindent
Это уравнение называется уравнением Пуассона. Здесь $\Delta = \frac{\partial}{\partial x} + \frac{\partial}{\partial y} + \frac{\partial}{\partial z}$ -- оператор Лапласа.
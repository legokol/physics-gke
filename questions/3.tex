\section{Законы сохранения энергии и импульса в классической механике. Упругие и неупругие столкновения.}

Рассмотрим систему материальных точек. На $j$-ую точки системы действуют силы, которые можно разбить на внешние и внутренние: $\b{\dot p}_i = \b F_i^{(e)} + \b F_i^{(i)}$, где внутренние силы -- силы взаимодействия с другими точками этой системы. В силу третьего закона Ньютона для силы взаимодействия между $j$ и $k$ точками $\b F_{kj} + \b F_{jk} = \b 0$. Таким образом $\b{\dot P} = \sum \b{\dot p}_i = \sum \b F_i^{(e)} = \b F^{(e)}$. Если сумма внешних сил равна нулю (в частности, если система замкнута), то имеет место закон сохранения импульса: \textit{Если сумма сил, действующих на систему, равна нулю, то импульс системы сохраняется}.

Рассмотрим теперь понятия работы и энергии.

\begin{definition}
    Элементарной работой называется скалярное произведение силы $\b F$ на элементарном перемещении $d \b s$:
    \begin{equation} \label{eq:элементарная работа}
        \delta A = \left( \b F, d \b s \right)
    \end{equation}
    Работа на пути -- криволинейный интеграл второго рода 
    \begin{equation} \label{eq:работа на пути}
        A = \int_L \left( \b F, d \b s \right)
    \end{equation}
\end{definition}

\noindent
С учётом второго закона Ньютона, получим

\begin{equation}
    \delta A = \b F \cdot d \b s = \frac{d \b p}{dt} \cdot \b v dt = \b v \cdot d \b p
\end{equation}

\noindent
Вспоминая определение импульса, имеем

\begin{equation}
    \delta A = m (\b v, d \b v) = m v dv \Rightarrow A_{12} = m \int_{v_1}^{v_2} vdv = \frac{m v_2^2}{2} - \frac{m v_1^2}{2}
\end{equation}

\begin{definition}
    Величина $K = \dfrac{m v^2}{2}$ называется кинетической энергией.
\end{definition}

Таким образом, работа равна изменению кинетической энергии. Этот факт обобщается на систему материальных точек с необходимостью учитывать работы всех сил, включая внутренние.

\begin{definition}
    Сила называется консервативной, если её работа на замкнутой траектории равна нулю (если работа не зависит от траектории).
\end{definition}

\begin{definition}
    Сила называет гироскопической, если зависит от скорости точки и действует перпендикулярно ей.
\end{definition}

\begin{note}
    Гироскопические силы не совершают работу.
\end{note}

Если на систему сил действуют только гироскопические и консервативные силы, то можно ввести понятие потенциальной энергии. Выберем <<нулевое>> положение системы, а работу, совершаемую консервативными силами при перемещении в это положение, назовём потенциальной энергией. Потенциальная энергия $U$ является функцией только координат, а консервативные силы $F = - \nabla U$. Работа этих сил при переходе из состояния 1 в состояние 2 равна $A_{12} = U_1 - U_2 = - \Delta U$.

Вспомним, что работа сил равна приращению кинетической энергии. Тогда в консервативной системе имеем

\begin{equation}
    K_2 - K_1 = U_1 - U_2 \Rightarrow K_1 + U_1 = K_2 + U_2 = E = \text{const}
\end{equation}

Величина $E$ называется полной энергией системы. Таким образом, мы получаем закон сохранения энергии: \textit{Полная энергия консервативной системы (с гироскопическими силами) сохраняется}.

Законы сохранения и импульса часто применяются для решения задач о соударениях частиц. Удар называется абсолютно неупругим, если после него два тела <<слипаются>> и продолжают движение вместе. В таком случае закон сохранения импульса даёт

\begin{equation}
    m_1 \b v_1 + m_2 \b v_2 = \left( m_1 + m_2 \right) \b u
\end{equation}

\noindent
а полная механическая энергия системы уменьшается.

Удар называется абсолютно упругим, если механическая энергия системы сохраняется. Для решения задач об упругих ударах необходимо рассматривать системы из уравнений ЗСИ и ЗСЭ.
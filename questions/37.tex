\section{Спектральные приборы (призма, дифракционная решетка, интерферометр Фабри-Перо) и их основные характеристики.}

\textit{Спектральными приборами} называются оптические инструменты, предназначенные для пространственного разделения спектральных компонент излучения. Они характеризуются несколькими основными параметрами:

\begin{enumerate}
    \item \textit{Угловая дисперсия} $D_\theta = \frac{d \theta}{d \lambda}$ характеризует угловое расстояние между спектральными линиями.
    \item \textit{Линейная дисперсия} $D_l = \frac{d l}{d \lambda}$ характеризует линейное расстояние между спектральными линиями.
    \item \textit{Разрешающая способность} $R = \frac{\lambda}{\delta \lambda}$, где $\delta \lambda$ -- минимальная разность разрешимых спектральных линий.
    \item \textit{Дисперсионная область} $\Delta \lambda$ -- область, в которой прибор применим (в больших областях возможно, например, перекрывание соседних линий).
\end{enumerate}

Рассмотрим некоторые примеры спектральных приборов.

\subsection{Призма}

Действие призмы как спектрального прибора основано на зависимости показателя преломления вещества от длины волны: $\frac{d n }{d \lambda}$. Таким образом разные спектральные линии отклоняются на разные углы.

\begin{figure}[htbp]
    \centering
    \input{images/37_призма.pdf_tex}
    \caption{Призма}
    \label{fig:призма}
\end{figure}

Для определения спектральных характеристик призмы воспользуемся принципом таутохронизма: лучи одного волнового фронта должны проходить одинаковый оптический путь. С учётом рис. \ref{fig:призма}:

\begin{equation*}
    \left.
        \begin{aligned}
            l n = 2 b \sin \alpha \\
            \pi = \theta + 2 \left( \frac{\pi}{2} - \alpha \right)
        \end{aligned}
    \right| \Rightarrow l d n = 2 b \cos \alpha d \alpha = a d \theta
\end{equation*}

\noindent
Отсюда для угловой дисперсии имеем

\begin{equation}
    D_\theta = \frac{l}{a} \frac{d n }{d \lambda}
\end{equation}

\noindent
Пусть $a$ -- ширина волнового фронта, прошедшего через диафрагму. Тогда имеет место дифракционное расширение $\delta \theta = \frac{\lambda}{a}$. Для разрешения спектральных линий нужно, чтобы угол между ними составлял больше $\delta \theta$: $\frac{l}{a} dn > \frac{\lambda}{a}$. Отсюда получим разрешающую способность призмы

\begin{equation}
    \frac{\lambda}{d \lambda} < l \frac{d n}{d \lambda} = R
\end{equation}

\subsection{Дифракционная решётка}

В случае дифракционной решётки направление на максимумы имеет вид

\begin{equation}
    \lambda m = b \sin \theta_m
\end{equation}

\noindent
А для ближайших к нему минимумов

\begin{equation}
    \lambda \left( m + \frac{1}{N} \right) = b \sin \theta_m
\end{equation}

\noindent
где $b$ -- период решётки, $m$ -- порядок максимума, $N$ -- число штрихов. Получим спектральные характеристики дифракционной решётки.

Дифференцируя выражение для максимума, получим угловую дисперсию:

\begin{equation}
    \frac{d \theta}{d \lambda} = \frac{m}{b \cos \theta_m}
\end{equation}

Для корректного разрешения необходимо, чтобы спектры соседних порядков не перекрывались. Тогда для дисперсионной области $m (\lambda + \Delta \lambda) = (m + 1) \lambda$

\begin{equation}
    \Delta \lambda = \frac{\lambda}{m}
\end{equation}

Наконец, условие разрешения двух спектральных линий $\lambda$ и $\lambda'$:

\begin{align*}
    b \sin \theta_m &= \left( m + \frac{1}{N} \right) \lambda \\
    b \sin \theta_m &= m \lambda'
\end{align*}

\noindent
Тогда $(m + 1 / N) \lambda = m \lambda'$ и $\delta \lambda = \lambda / (N m)$. Разрешающая способность дифракционной решётки:

\begin{equation}
    R = m N
\end{equation}

\subsection{Интерферометр Фабри-Перо}

Интерферометр Фабри-Перо представляет собой систему из двух параллельных пластик с высокими коэффициентами отражения. При таких условиях интерференционная картина представляет собой последовательность резких высоких максимумов. Если $h$ -- расстояние между пластинками, то для направления на максимум: $m \lambda = 2 h \cos \theta_m$. Разрешающая способность интерферометра Фабри-Перо:

\begin{equation}
    R = \frac{\pi m \sqrt{R}}{1 - R}
\end{equation}

\noindent
где $R$ -- коэффициент отражения по энергии, а $m$ -- номер максимума. Дифференцируя выражение для направления на максимум, получим угловую дисперсию:

\begin{equation}
    \frac{d \theta}{d \lambda} = \frac{m}{2 h \sin \theta_m}
\end{equation}
\section{Резонансные ядерные реакции, формула Брейта-Вигнера. Упругий и неупругие каналы реакции.}

При некоторых значениях энергии налетающих частиц сечение реакции (и, соответственно, её вероятность) резко возрастают и имеют резонансный характер. Такие реакции называются \textit{резонансными}. В отличие от модели составного ядра, резонансные реакции обычно связаны с действительным образованием реальную промежуточную частицу, а не <<виртуальную>>, имеющую короткое время жизни. Вблизи максимума сечение реакции описывается формулой Брейта-Вигнера:

\begin{equation}
    \sigma_{a b} = \pi \lambda^2 \frac{\Gamma_a \Gamma_b}{(\mathcal{E} - \mathcal{E}_0)^2 + \Gamma^2 / 4}
\end{equation}

\noindent
где $\lambda = \frac{\hbar}{p}$. Индексы $a$ и $b$ характеризуют возможные состояние. $\Gamma$ -- ширина резонансной кривой, она равна сумме ширин уровней по всем возможным каналам реакции $\Gamma = \Gamma_a + \Gamma_b + \dots$. Величина $\Gamma$ связана со временем жизни составной частицы через соотношение неопределённостей: $\Gamma \approx \frac{\h}{\tau}$.

Если в результате составная частица распадается на исходные элементы, то есть происходит реакция вида $a + A = \dots = a + A$, то говорят об \textit{упругом канале реакции}. В противном случае -- о \textit{неупругом}.
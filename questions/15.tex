\section{Теплоемкость. Закон равномерного распределения энергии по степеням свободы. Зависимость теплоемкости газов от температуры.}

\begin{definition}
    Теплоёмкостью $C$ называется величина
    \begin{equation*}
        C = \frac{\delta Q}{dT}
    \end{equation*}
    Теплоёмкость на единицу массы называется удельной, на один моль -- молярной. В термодинамике наиболее часто используется именно последняя.
\end{definition}

В частности, из первого начала термодинамики следуют выражения для теплоёмкостей при постоянном объёме и давлении:

\begin{gather}
    C_V = \left( \frac{\partial U}{\partial T} \right)_V \\
    C_P = \left( \frac{\partial U}{\partial T} \right)_V + \left[ \left( \frac{\partial U}{\partial V} \right)_T + P \right] \left( \frac{\partial V}{\partial T} \right)_P
\end{gather}

Из определения энтальпии для теплоёмкости при постоянном давлении так же имеет место выражение

\begin{equation}
    C_P = \left( \frac{\partial H}{\partial T} \right)_P
\end{equation}

В состоянии теплового равновесия, например, газа и поршня, оказывается, что средняя кинетическая энергия молекул газа должна быть равна средней кинетической энергии поршня, откуда следует связь температуры с кинетической энергией. В зависимости от строения молекул эта кинетическая энергия может иметь различную структуру: только поступательные степени свободы, вращательные степени свободы и наконец даже колебательные степени свободы.

Ввиду хаотичности теплового движения все направления скорости молекулы равновероятны, а значит одинакова и средняя энергия, приходящаяся на каждое направление (кинетическая энергия движения вдоль осей $X, Y, Z$). Учитывая, что всего на поступательные степени свободы приходится $\frac{3}{2} k T$, то на каждую степень свободы приходится $k T / 2$ энергии. При наличии вращательных степеней свободы и колебательных степеней свободы имеет место \textit{теорема о равномерном распределении кинетической энергии по степеням свободы}. Её формулировка состоит в следующем:

\textit{В состоянии теплового равновесия средняя кинетическая энергия, приходящаяся на каждую степень свободы, одинакова.} В случае газа эта энергия равна $k T / 2$.

Классическая теория теплоёмкости идеальных газов основана на этой теореме. Для идеальных газов имеем:

\begin{align}
    C_V &= \frac{R}{\gamma - 1}, & C_P &= \gamma \frac{R}{\gamma - 1}
\end{align}

Внутренняя энергия состоит из кинетической энергии поступательного, вращательного и колебательного движения молекул. Потенциальная энергия взаимодействия в идеальном газе не учитывается.

Молекулы одноатомного газа будем рассматривать как материальные точки. Они имеют только поступательные степени свободы, таким образом молярные теплоёмкости одноатомного газа:

\begin{align}
    C_V &= \frac{3}{2} R, & C_P &= \frac{5}{2} R
\end{align}

Двухатомные газы имеют уже 5 степеней свободы: 3 поступательных и 2 вращательных. Таким образом

\begin{align}
    C_V &= \frac{5}{2} R, & C_P &= \frac{7}{2} R
\end{align}

Молекулу многоатомного газа можно рассматривать как твёрдое тело, имеющее 6 степеней свободы. Таким образом его теплоёмкость

\begin{align}
    C_V &= 3 R, & C_P &= 4 R
\end{align}

Теплоёмкость, однако, не является постоянной величиной и может меняться с изменением температуры. Это связано с тем, что некоторые степени свободы не всегда возбуждены. Например, колебательные степени свободы начинают оказывать влияние на теплоёмкость только при высоких температурах. Объяснение зависимости теплоёмкости от температуры формируется на основе квантовой физики. При достаточной температуре <<возбуждаются>> вращательные и колебательные степени свободы у большого числа молекул, что и оказывает влияние на теплоёмкость газа.
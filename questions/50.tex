\section{Опыты Штерна и Герлаха. Спин электрона. Орбитальный и спиновый магнитные моменты электрона.}

Рассмотрим вращение заряженной частицы с постоянной скоростью $v$. Её момент импульса равен $L = m v r$, где $r$ -- радиус окружности. С другой стороны, поскольку она имеет заряд, при вращении возникает ток и, соответственно, магнитный момент $\mathfrak{m} = \frac{1}{c} \pi r^2 \frac{q}{2 \pi r / v} = \frac{q v r}{2 c}$. Таким образом, отношение магнитного момента к моменту импульса есть постоянная величина, называемая \textit{гиромагнитным соотношением}:

\begin{equation}
    \frac{\mathfrak{m}}{L} = \gamma = \frac{q}{2 m c}
\end{equation}

\noindent
Из этой связи следует, например, что при перемагничивании вещества должен измениться момент импульса.

Электрон обладает собственным моментом импульса, не связанным с орбитальным вращением, что продемонстрировал \textit{эксперимент Штерна-Герлаха}: пучок атомов серебра в $1S$ состоянии пропускался через создаваемое постоянным магнитом сильно неоднородное магнитное поле. В классической механике распределение проекций момента импульса было бы непрерывным, что привело бы к непрерывной полосе на пластинке, фиксировавшей попадание атомов. На практике, однако на пластинке наблюдались две чёткие полосы, что свидетельствует о квантовании момента импульса.

\noindent
Заметим, что подобное квантование не может быть связано с орбитальным моментом импульса: он принимает неотрицательные целые значения $l$ и число возможных проекций на выделенную ось равно $2 l + 1$. Таким образом причиной расщепления является собственный момент импульса электрона, значение которого равно $1 / 2$ в единицах $\h$, называемый \textit{спином электрона}.
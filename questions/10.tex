\section{Уравнение состояния идеального газа, его объяснение на основе молекулярно-кинетической теории. Уравнение неидеального газа Ван-дер-Ваальса.}

Системы, рассматриваемые в термодинамике, как например газы, содержат в себе очень большое число частиц, поэтому их описание путем описания, например, траекторий и скоростей всех частиц не представляется возможным. Для описания состояния таких систем используются макроскопические параметры: давление, объём, температура и т.д. Оказывается, что если две макроскопические величины заданы, то они однозначно определяют третью и для таким систем можно записать \textit{уравнение состояния:}

\begin{equation*}
    f(P, V, T) = 0
\end{equation*}

\begin{definition}
    Газ, подчиняющийся законам Бойля-Мариотта ($PV = f(T)$) и Гей-Люссака ($P/T = g(P)$) будем называть идеальным. Эквивалентное определение идеального газа -- газ, в котором не учитывается взаимодействие молекул за исключением соударения.
\end{definition}

Следствием этих экспериментальных законов является уравнение состояния идеального газа, называемое уравнением Менделеева-Клапейрона:

\begin{equation}
    P V = \nu R T
\end{equation}

Получим формулу для давления идеального газа с точки зрения молекулярно-кинетической теории. Разделим молекулы на группы так, что в одной и той же группы молекулы имеют примерно одинаковую по величине и направлению скорость. Скорость молекул этой группы -- $v_i$, а концентрация -- $n_i$. Рассмотрим площадку $\sigma$ на стенке сосуда. Пусть $i$-ые молекулы движутся к ней. За время $dt$ с ней столкнётся $n_i v_{i n} \sigma dt$ молекул, где $v_{in}$ -- нормальная составляющая скорости к площадке.

Рассчитаем силу, с которой эти молекулы давят на площадку. Предположим, что удары абсолютно упругие. Тогда для одной молекулы $dF_i = dp / dt = 2 m v_{in} / dt$, а для всех попадающих на стенку за время $dt$ $F_i = 2 m n_i v_{in}^2 \sigma$, или $P_i = 2 m n_i v_{in}^2$. С учётом равномерного распределения на стенку давит примерно $1/6$ всех молекул, таким образом

\begin{equation*}
    P = \frac{1}{3} m n \overline{v^2}
\end{equation*}

\noindent
или

\begin{equation}
    P V = N m \overline{v^2} = \frac{2}{3} N \overline{E_\text{пост}} = \nu R T
\end{equation}

Однако, модель идеального газа не всегда даёт верные результаты. Из-за взаимодействия между частицами поведение реального газа может значительно отличаться. Одним из уравнений, учитывающих это взаимодействие, является уравнение Ван-дер-Ваальса. Учтём влияние молекулярных сил в уравнении Менделеева-Клапейрона. Во-первых на малых расстояниях молекулы отталкиваются друг от друга, уменьшая доступный объём:

\begin{equation*}
    V \to V - b
\end{equation*}

\noindent
Во-вторых, на больших расстояниях они притягиваются, уменьшая давление:

\begin{equation*}
    P \to P + \frac{a}{V^2}
\end{equation*}

\noindent
Результатом этих предположений является уравнение Ван-дер-Ваальса для одного моля:

\begin{equation}
    \left( P + \frac{a}{V^2} \right) \left( V - b \right) = R T
\end{equation}

\noindent
Или для, нескольких молей:

\begin{equation}
    \left( P + \frac{a \nu^2}{V^2} \right) \left( V - b \nu \right) = \nu R T
\end{equation}
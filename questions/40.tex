\section{Принципы голографии. Голограмма Габора. Голограмма с наклонным опорным пучком. Объемные
голограммы.}

\begin{definition}
    Голография -- метод точной записи волновых полей с учётом амплитуды и фазы.
\end{definition}

Основным отличием от фотографии является именно запись фазы, а не только интенсивности. Для записи голограммы некоторого предмета используется две волны: первая -- \textit{предметная волна}, вторая -- \textit{опорная волна}. результат их интерференции записывается на фотоэмульсии, в результате чего записывается и информация о фазе предметной волны относительно опорной. Результатом является фотопластинка с некоторой функцией пропускания $T(x)$. При освещении голограммы волной, идентичной опорной, будет восстановлено изображение предмета.

\textit{Схемой Габора} называется голографическая схема, при которой опорная волна падает на фотопластинку нормально в том же направлении, что и предметная волна.

В качестве примера рассмотрим запись плоской волны, падающей под углом $\theta$ к нормали в схеме Габора. Волновые векторы падающей и опорной волны $\b k_2$ и $\b k_1$ соответственно, причём $k_1 = k_2 = k$. Имеем

\begin{equation}
    E = E_1 e^{i k_1 z} + E_2 e^{i \b k_2 \b r}
\end{equation}

\noindent
Для интенсивности интерферирующих волн в плоскости голограммы:

\begin{equation}
    I = E_1^2 + E_2^2 + 2 E_1 E_2 \cos \left( k \sin \theta x \right)
\end{equation}

\noindent
Пропускающая функция фотопластинки будет иметь вид $T = T_0 (1 + m \cos \left( k \sin \theta x \right))$. В соответствии с принципами фурье-оптики, при освещении пластинки волной, аналогичной опорной, мы получим:

\begin{equation}
    E = E_0 T_0 e^{i k z} + \frac{m T_0}{2} e^{i \left( k \sin \theta x + k \cos \theta z \right)} + \frac{m T_0}{2} e^{i \left( k \sin \theta x - k \cos \theta z \right)}
\end{equation}

\noindent
Помимо записанной волны мы также получили восстанавливающую волну и сопряжённую волну.

Рассмотрим теперь голограмму с наклонным пучком. Пусть опорная волна имеет вид $E = E_0 e^{i \left( k \cos \theta z + k \sin \theta x \right)}$, а предметная в плоскости голограммы -- $a(x) e^{i \psi(x)}$ Введём обозначение $u = k \sin \theta$. Тогда интенсивность имеет вид

\begin{equation}
    I(x) = E_0^2 + a^2 + E_0 a \left( e^{i \left( u x - \psi(x) \right)} + e^{i \left( - u x + \psi(x) \right)} \right)
\end{equation}

\noindent
Пусть функция пропускания полученной фотопластинки имеет вид $T = T_0 * I(x)$. Тогда при просвечивании волной, аналогичной опорной, мы получим следующее распределение волнового фронта в плоскости $z = 0$:

\begin{equation}
    E = E_0 T_0 (E_0^2 + a^2) e^{i u x} + T_0 E_0^2 a e^{i \psi(x)} + T_0 E_0^2 a e^{i \left( 2 u x - \psi(x) \right)}
\end{equation}

\noindent
Первое слагаемое соответствует амплитудно-модулированной восстанавливающей волне, второе -- изображению предмета, а третье -- сопряжённому изображению. На благодаря множителю $e^{i 2 u x}$ сопряжённая волна идёт в другом направлении, что позволяет выделить записанное изображение предмета.

До сих пор рассматривались голограммы, основанные на записи интерференционной картины в тонком слое светочувствительного материала. Возможно также создание \textit{объёмных} голограмм, в которых картина интерференция записывается в некотором объёме материала. Пусть предметная волна -- плоская, падающая на пластинку под углом $\theta / 2$, а опорная имеет ту же длину волны и падает симметрично предметной. Если амплитуды одинаковы, то распределение интенсивности будет иметь вид $I = 2 E_0^2 \left( 1 + \cos \delta \b k \b r \right)$, где $\delta \b k = \b k_2 - \b k 1$ -- вектор, направленный вдоль оси $x$. В результате голограмма будет представлять собой ряд зеркальных поверхностей.

При восстановлении голограммы происходит многолучевая интерференция, при которой волны, падающие под тем же углом, что и предметная, усиливают друг друга. Если $d$ -- полученный период интерференции, то для проходящей волны должно быть выполнено условие Брэгга-Вульфа: $2 d \sin \phi = m \lambda$. Такую голограмму можно восстанавливать естественным светом, поскольку длины волн, не удовлетворяющие этому условию, не будут проходить через голограмму.
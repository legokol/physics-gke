\section{Второе начало термодинамики. Цикл Карно. Энтропия и закон ее возрастания. Энтропия идеального газа. Статистический смысл энтропии.}

Существует множество различных формулировок второго начала термодинамики. Рассмотрим две классических.

\begin{itemize}
    \item Томсон: \textit{Невозможен круговой процесс, единственным результатом которого является совершение работ за счёт охлаждения теплового резервуара (получения тепла).}
    \item Клаузиус: \textit{Теплота не может самопроизвольно переходить от менее нагретого тела к более нагретому (нельзя передать теплоту так, чтобы в природе не произошло больше никаких изменений).}
\end{itemize}

Покажем эквивалентность этих определений. Пусть процесс Клаузиуса возможен. Тогда возьмём тепловую машину, которая отнимет у нагревателя теплоту $Q_1$ и отдаст холодильнику теплоту $Q_2$, совершив при этом работу $A = Q_1 - Q_2$. С помощью процесса Клаузиуса затем вернём нагревателю теплоту $Q_2$. Тогда мы получили процесс, единственным результатом которого является совершение работы за счёт охлаждения нагревателя, что противоречит невозможности процесса Томсона.

Предположим теперь, что возможен процесс Томсона. Тогда отнимем теплоту $Q$ у менее нагретого тела и с помощью работы в процессе Томсона (например, за счёт трения) передадим её более горячему. Мы получили процесс Клаузиуса. Эквивалентность формулировок доказана.

Рассмотрим процесс Карно, в котором система приводится в тепловой контакт с тепловыми резервуарами, температуры которых равны $T_1$ и $T_2$, $T_1 > T_2$. Более горячий резервуар -- нагреватель, более холодный -- холодильник (охладитель). Цикл Карно состоит в следующем:

\begin{enumerate}
    \item Система, имея температуру $T_1$ приводится в тепловой контакт с нагревателем. При квазистатическом уменьшении давления система изотермически расширяется, забирая теплоту у нагревателя и совершая работу против внешнего давления.
    \item После система адиабатически изолируется и расширяется по адиабате, пока не достигнет температуры $T_2$, также совершая работу.
    \item Система приводится в тепловой контакт с холодильником и изотермически сжимается, отдавая теплоту холодильнику, при этом система совершает отрицательную работу.
    \item Система адиабатически сжимается до температуры $T_1$.
\end{enumerate}

В результате цикла внутренняя энергия системы не меняется, работа есть $A = Q_1 - Q_2$, а КПД соответственно

\begin{equation}
    \eta = \frac{Q_1 - Q_2}{Q_1}
\end{equation}

\begin{theorem}[Первая теорема Карно]
    КПД тепловой машины, работающей по циклу Карно, зависит только от температур нагревателя и холодильника и не зависит от устройства машины или вида рабочего вещества.
\end{theorem}

\begin{proof}
    Рассмотрим две машины Карно, имеющие общие нагреватель и холодильник. Пусть КПД первой -- $\eta$, а второй -- $\eta'$. Предположим, что $\eta > \eta'$.

    Цикл Карно -- квазистатический, а значит его возможно проводить в обратном направлении. Пусть в результате $m$ циклов первая машина Карно совершит работу $A = Q_1 - Q_2$, равную разности теплоты, отнятой у нагревателя и отданной холодильнику. Будем использовать вторую машину для передачи теплоты от холодильника к нагревателю с помощью обратного цикла, совершая над ней работу за счёт первой машины. В результате $m'$ циклов она отберёт у холодильника теплоту $Q_2'$ и передаст нагревателю $Q_1'$, над ней будет совершена работа $A' = Q_1' - Q_2'$.

    В результате нагреватель отдал теплоту $Q_1 - Q_1'$, а холодильник получил теплоту $Q_2-Q_2'$. Таким образом машина совершила работу

    \begin{equation*}
        A - A' = (Q_1 - Q_2) - (Q_1' - Q_2') = \eta Q_1 - \eta' Q_1'
    \end{equation*}

    Будем использовать формулировку Томсона для доказательства. Подберём целые числа $m$ и $m'$ так, чтобы $Q_1 - Q_1' = 0$. Тогда результат процесса будет следующим: состояние нагревателя не изменилось, холодильник отдал теплоту $Q_2' - Q_2 = (\eta - \eta') Q_1 > $, а машина совершила работу $A - A' = (\eta - \eta') Q_1 > 0$. Таким образом мы получили процесс, единственным результатом которого является совершение работы за счёт теплоты, полученной от холодильника, что противоречит второму началу термодинамики. Аналогично доказывается невозможность случая $\eta < \eta'$, и равенство КПД.
\end{proof}

\begin{theorem} [Вторая теорема Карно]
    КПД любой тепловой машины, работающей между двумя резервуарами не превышает КПД машины Карно для резервуаров с такими же температурами.
\end{theorem}

\noindent
При этом КПД машины Карно $\eta = 1 - \frac{T_2}{T_1}$. В силу второй теоремы Карно

\begin{equation}
    1 - \frac{Q_2}{Q_1} \leq 1 - \frac{T_2}{T_1} \Rightarrow \frac{Q_1}{T_1} + \frac{Q_2}{T_2} \leq 0
\end{equation}

Полученное неравенство называется \textit{неравенством Клаузиуса}. Оно обращается в равенство при использовании цикла Карно. При обобщении можно получить неравенство Клаузиуса в интегральной форме

\begin{equation} \label{eq:неравенство клаузиуса}
    \oint \frac{\delta Q}{T} \leq 0
\end{equation}

\noindent
В общем случае в \eqref{eq:неравенство клаузиуса} подразумевается температура окружающей среды, с которой система обменивается теплотой. В квазистатическом процессе эти температуры равны. Поскольку квазистатический процесс обратим, то для обратного процесса \eqref{eq:неравенство клаузиуса} тоже должно быть верно. Это возможно, только если оно обращается в равенство. Таким образом для квазистатических процессов верно равенство Клаузиуса

\begin{equation} \label{eq:равенство клаузиуса}
    \oint_\text{квст} \frac{\delta Q}{T} = 0
\end{equation}

\noindent
В силу \eqref{eq:равенство клаузиуса} можно ввести новую функцию состояния \textit{энтропию} S:

\begin{equation}
    dS = \left( \frac{\delta Q}{T} \right)_\text{квст}
\end{equation}

\noindent
Получим выражение для энтропии идеального газа из первого начала термодинамики:

\begin{equation*}
    dS = \frac{\delta Q}{T} = \frac{\nu C_V dT}{T} + \frac{P dV}{T} = \frac{\nu C_V dT}{T} + \frac{\nu R dV}{V}
\end{equation*}

\noindent
Если $C_V$ не зависит от температуры, то получим

\begin{equation}
    S = \nu C_V \ln T + \nu R \ln V + \text{const}
\end{equation}

Рассмотрим переход системы из состояния 1 в состояние 2 в неравновесном процессе $I$, а затем её возвращение в исходной состояние в равновесном $II$. Тогда согласно неравенству Клаузиуса

\begin{equation*}
    \oint \frac{\delta Q}{T} = \int_I \frac{\delta Q}{T} + \int_{II} \frac{\delta Q}{T} \leq 0
\end{equation*}

\noindent
Поскольку процесс $II$ -- равновесный, то $\int_{II} \frac{\delta Q}{T} = S_1 - S_2$. Тогда имеем

\begin{equation*}
    S_2 - S_1 \geq \int_I \frac{\delta Q}{T}
\end{equation*}

\noindent
В адиабатически изолированной системе этот интеграл равен нулю, а значит

\begin{equation}
    S_2 \geq S_1
\end{equation}

Таким образом \textit{энтропия адиабатически изолированной системы не может убывать}. Это утверждение есть закон возрастания энтропии.

В статистической физике энтропия определяется формулой Больцмана:

\begin{equation}
    S = k \ln G
\end{equation}

\noindent
где $k$ -- постоянная Больцмана, а $G$ -- статистический вес данного макросостояния (число реализующих его микросостояний).
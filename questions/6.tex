\section{Закон сохранения момента импульса. Уравнение моментов. Вращение твердого тела вокруг неподвижной оси. Гироскопы.}

\begin{definition}
    Моментом силы $\b F$ относительно точки (полюса) $O$ называется векторное произведение $\b M = \left[ \b r, \b F \right]$, где $\b r$ -- радиус-вектор точки приложения силы.
\end{definition}

\begin{definition}
    Моментом импульса материальной точки с импульсом $\b p$ относительно точки (полюса) $O$ называется векторное произведение $\b L = \left[ \b r, \b p \right]$.
\end{definition}

Для производной момента импульса имеем:

\begin{equation*}
    \b{\dot L} = \left[ \b{\dot r}, \b p \right] + \left[ \b r, \b{\dot p} \right]
\end{equation*}

Поскольку при неподвижном начале координат скорость тела параллельна импульсу, то первое слагаемое обращается в ноль. С учётом второго закона Ньютона получаем уравнение моментов:

\begin{equation} \label{eq:уравнение моментов}
    \b{\dot L} = \b M
\end{equation}

Для замкнутой механической системы по аналогии с законом сохранения импульса имеет место закон сохранения момента импульса, поскольку сумма моментов противоположных сил так же равна нулю.

Рассмотрим вращение вокруг неподвижной оси. Если материальная точка вращается вокруг оси $O$ по окружности радиуса $r$ со скоростью $v$, то её момент импульса равен $L = mvr$. Пусть $\omega$ -- угловая скорость точки. Тогда для момента импульса имеет место выражение $L = m r^2 \omega = I \omega$, где величина $I$ называется моментом инерции относительно оси $O$. В случае системы материальных точек (твёрдого тела) момент инерции вычисляется как сумма (интеграл) по всем точкам системы (по всему телу). Отсюда получим основное уравнение динамики вращательного движения относительно неподвижной оси:

\begin{equation}
    \frac{d}{dt}\left( I \omega \right) = M
\end{equation}

Более общим понятием, чем момент инерции, является тензор инерции, в вывод которого углубляться не будем. Заметим лишь, что это симметричный тензор второго ранга, который можно привести к диагональному виду в ОНБ из собственных векторов, причём для твёрдого тела верно

\begin{equation}
    \b L = \hat I \boldsymbol \omega
\end{equation}

\noindent
То есть во общем случае момент импульса не сонаправлен с угловой скоростью.

Рассмотрим динамически-симметричное тело: тело, у которого два главных момента инерции совпадают. В таком теле момент импульса можно представить в следующем виде:

\begin{equation}
    \b L = I_\parallel \boldsymbol \omega_\parallel + I_\perp \boldsymbol \omega_\perp
\end{equation}

\noindent
Пусть составляющая угловой скорости вокруг оси симметрии намного больше, чем другая её составляющая. В таком случае при вычислении момента импульса можно пренебречь вторым слагаемым. Если приложить силу к оси гироскопа, то момент сил будет перпендикулярен моменту импульса, а значит может изменить лишь его направление, а не скорость. При постоянной силе это явление называется вынужденной регулярной прецессией гироскопа. Пусть $\boldsymbol \Omega$ -- скорость прецессии. Тогда имеем

\begin{equation}
    \b{\dot L} = \left[ \boldsymbol \Omega, \b L \right] = \b M = \left[ \b a, \b F \right]
\end{equation}

\noindent
Основная формула гироскопии:

\begin{equation}
    \Omega = \frac{M}{L}
\end{equation}
\noindent
Если же ось гироскопа наклонена относительно оси прецессии, то формула примет вид

\begin{equation}
    \boldsymbol \Omega = - \frac{a}{L} \b F
\end{equation}
\section{Тонкая и сверхтонкая структуры оптических спектров. Правила отбора при поглощении и испускании фотонов атомами.}

В случае атома водорода и водородоподобных атомов энергия состояния определяется главным квантовым числом по формуле

\begin{equation}
    \mathcal{E}_n = - \frac{Ry}{n^2}
\end{equation}

\noindent
Соответственно, спектр должен состоять из частот, удовлетворяющих переходам между этими уровнями. Исследование спектральных линий с помощью приборов с высокой разрешающей способностью показывает расщепление этих линий. Основное взаимодействие, которым непосредственно определяются энергетические уровни -- кулоновское. Однако из-за движения электрона вокруг ядра и наличия у электрона спина возникает дополнительное \textit{спин-орбитальное взаимодействие}.

Рассмотрим модель атома водорода, в которой электрон вращается вокруг ядра. В системе отсчёта, связанной с электроном, протон движется по круговой орбите, создавая магнитное поле $\b B$. Электрон в свою очередь обладает спиновым магнитным моментом $\mathfrak{m}$, который имеет две допустимых проекции на направление магнитного поля. В результате происходит расщепление с энергией $\pm \left( \b B, \mathfrak{m} \right)$, называемое \textit{тонкой структурой}.

Оказывается, что в ряде случаев происходит дополнительное расщепление уровней энергии, уже не связанное со спин-орбитальным взаимодействием, называемое \textit{сверхтонкой структурой}. Её появление связано со \textit{спиновым моментом ядра}, складывающимся из спинов и орбитальных моментов протонов и нейтронов, составляющих ядро. При движении электронов вокруг ядра создаётся магнитное поле, энергия взаимодействия с которым и объясняет расщепление уровней.

Как было сказано ранее, переходы между энергетическими уровнями возможны при поглощении или испускании фотона, энергия которого точно равна разности энергий между уровнями. На практике не все такие переходы оказываются возможными и существуют \textit{правила отбора}, связанные с законами сохранения. Мы рассмотрим только однофотонные процессы,, поскольку другие крайне маловероятны.

\textit{Закон сохранения момента импульса} можно записать в виде

\begin{equation}
    \b J = \b J' + \b s
\end{equation}

\noindent
где $\b s$ -- спин фотона. Абсолютно невозможны переходы, при которых $J =  J' = 0$.

Для длин векторов можно воспользоваться неравенством треугольника, учитывая только, что $\left| \b J \right| = J (J + 1)$. Для фотона $s = 1$, соответственно $s (s + 1) = 2$. Из неравенств следуют возможные изменения полного момента:

\begin{equation}
    \Delta J = 0, \pm 1
\end{equation}

Для проекций моментов $m_J$ аналогично

\begin{equation}
    \Delta m_J = 0, \pm 1
\end{equation}
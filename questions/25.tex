\section{Закон Ома в цепи постоянного тока. Переходные процессы в электрических цепях.}

Рассмотрим упорядоченное движение электрических зарядов (например электронов). Пусть  $\b u$ -- их дрейфовая скорость. Тогда плотность тока -- величина заряда, протекающая через единицу площади за единицу времени, равна

\begin{equation}
    \b j = n e \b u
\end{equation}

\noindent здесь $e$ -- заряд частиц (предполагаем их одинаковыми), а $n$ -- их концентрация.

Токи в веществе возникают под действием внешних электрических полей. Они связаны между собой законом Ома:

\begin{equation}
    \b j = \lambda \b E
\end{equation}

\noindent где величина $\lambda$ называется \textit{удельной проводимостью} и является свойством материала. Величина, обратная удельной проводимости, называется \textit{удельным сопротивлением}:

\begin{equation}
    \rho = \frac{1}{\lambda}
\end{equation}

Если бы единственными источниками поля $\b E$ в проводниках являлись заряды, то со временем поле бы пропадало и, соответственно, прекращалось бы течение тока. Для его поддержания необходимы дополнительные силы $\b F^\text{стор}$. Определим их напряжённость $\b E^\text{стор}$ как отношение силы к заряду. Тогда закон Ома примет вид:

\begin{equation}
    \b j = \lambda \left( \b E + \b E^\text{стор} \right)
\end{equation}

Рассмотрим теперь случай течения тока по тонким проводам. При малой толщине плотность тока можно считать постоянной вдоль всего сечения и силу тока:

\begin{equation}
    I = j S
\end{equation}

\noindent если ток постоянен, то величина $I$ будет постоянна вдоль всего провода. Из закона Ома в дифференциальной форме:

\begin{equation*}
    E + E^\text{стор} = \frac{I}{\lambda S}
\end{equation*}

Умножим это соотношение на элемент длины $dl$ и проинтегрируем по участку провода от некоторой точки 1 до точки 2, на части которого $E^\text{стор} \neq 0$. Имеем

\begin{equation*}
    \int_{12} E dl + \int_{12} E^\text{стор} dl = \int_{12} \frac{I}{\lambda S} dl
\end{equation*}

\noindent
Первый интеграл в силу потенциальности электрического поля равен просто разности потенциалов $\phi_1 - \phi_2$.

Пусть сторонние силы действуют на участке $34$. Величина

\begin{equation}
    \mathcal{E} = \int_{12} E^\text{стор} dl = \int_{34} E^\text{стор} dl
\end{equation}

\noindent
называется электродвижущей силой и характеризует работу сторонних сил на единицу заряда. Наконец, третий интеграл

\begin{equation}
    R = int_{12} \frac{dl}{\lambda S} = \int_{12} \rho \frac{dl}{S}
\end{equation}

\noindent
характеризует свойства проводника и называется сопротивлением. Таким образом имеем интегральный закон Ома для участка цепи:

\begin{equation}
    \phi_1 - \phi_2 + \mathcal{E} = I R
\end{equation}

\noindent
Если на участке нет источника тока (не действуют сторонние силы), то

\begin{equation}
    \phi_1 - \phi_2 = I R
\end{equation}

\noindent
Если же рассмотреть замкнутый участок цепи, то $\phi_1 = \phi_2$ и

\begin{equation}
    \mathcal{E} = I R
\end{equation}

Рассмотрим задачи о зарядке и разрядке конденсатора в предположении, что в каждый момент времени величина тока постоянна во всём проводнике. Пусть $Q$ -- заряд конденсатора, $\phi$ -- разность потенциалов между обкладками, а $C$ -- его ёмкость. Имеем следующие уравнения:

\begin{align*}
    I &= - \frac{d Q}{d t}, & R I &= \phi, & Q &= C \phi
\end{align*}

\noindent
Исключая из этих уравнений разность потенциалов и силу тока, получим

\begin{equation*}
    \frac{d Q}{d t} + \frac{Q}{R C} = 0
\end{equation*}

\noindent
После интегрирования придём к соотношению, описывающему зависимость зарядка конденсатора от времени:

\begin{equation}
    Q = Q_0 e^{-t / \tau}
\end{equation}

\noindent
где $Q_0$ -- начальная величина заряда, а $\tau = R C$ -- время релаксации. Аналогично можно получить решение задачи о зарядке конденсатора, подключенного к источнику тока с ЭДС $\mathcal{E}$:

\begin{equation}
    Q = \mathcal{E} C \left( 1 - e^{-t / \tau} \right)
\end{equation}
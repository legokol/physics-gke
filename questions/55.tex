\section{Ядерный и электронный магнитный резонансы.}

Эффект Зеемана наблюдается при переходах между подуровнями \textbf{различных уровней}. В случае магнитного резонанса происходит переход между расщеплёнными состояниями одного уровня. В случае \textit{электронного парамагнитного резонанса} (ЭПР) состояния электрона с разным направлением спина имеют разную энергию, причём $\Delta E = 2 \mathfrak{m} B$, где $\mathfrak{m}$ -- магнитный момент электрона. Переход возможен при поглощении фотона с резонансной частотой:

\begin{equation}
    \h \omega_0 = 2 \mu_\text{Б} B
\end{equation}

\noindent
Слово <<парамагнитный>> в названии связано с тем, что переход осуществляется при наличии неспаренного электрона во внешней оболочке атома, что является причиной парамагнетизма вещества.

Ядро так же может обладать кинетическим моментом, с которым связан магнитный момент. Характерный масштаб ядерных магнитных моментов есть \textit{ядерный магнетон}, имеющий величину почти примерно в 2000 раз меньшую величины магнетона Бора. В случае, когда происходит поглощение излучения, связанное с переходом между состояниями ядра, говорят о \textit{ядерном магнитной резонансе} (ЯМР).
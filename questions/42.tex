\subsection{Поляризация света. Угол Брюстера. Оптические явления в одноосных кристаллах.}

\begin{definition}
    Поляризация -- характеристика векторных волновых полей, описывающая поведение вектора колеблющейся величины в плоскости, перпендикулярной направлению распространения.
\end{definition}

В изотропных средах свет представляет собой поперечные электромагнитные волны. Плоскостью поляризации называется плоскость, построенная на векторах $\b E$ и $\b k$. Если ориентация этой плоскости (и, соответственно, вектора $\b E$) меняется случайным образом, то свет \textit{естественный} или \textit{неполяризованный}.

Пусть волновой вектор $\b k$ направлен вдоль оси $z$, а векторы $\b E$ и $\b H$ лежат в плоскости $x y$. Для компонент поля с частотой колебаний $\omega$ можно записать

\begin{align}
    E_x &= E_{x 0} \cos \left( \omega t + \phi_x \right) \\
    E_y &= E_{y 0} \cos \left( \omega t + \phi_y \right)
\end{align}

Если разность фаз в колебаниях этих величин равна

\begin{equation}
    \delta = \phi_x - \phi_y = 0, \pm \pi
\end{equation}

\noindent
то вектор $\b E$ совершает колебания вдоль фиксированной прямой, образующей с осью $x$ угол $\theta$: $\tg \theta = \pm E_{y 0} / E_{x 0}$. Это случай линейно поляризованной волны.

В общем случае, вектор $\b E$ вращается в плоскости $x y$, и говорят об \textit{эллиптической поляризации} света. В случае, когда $\delta = \pm \pi / 2$ и $E_{x 0} = E_{y 0}$ имеет место \textit{круговая поляризация}. Поляризация света определяется только сдвигом фаз между компонентами поля и соотношением их амплитуд.

Рассмотрим один из способов получения поляризованного света: поляризация при отражении от границы раздела сред с разными оптическими свойствами. Пусть их показатели преломления равны $n_1$ и $n_2$ соответственно. Согласно законам отражения и преломления возникают отражённая и преломлённая волна. Падающую волну рассмотрим как суперпозицию $s$--поляризованной ($\b E$ параллелен границе раздела) и $p$-поляризованной ($\b H$ параллелен границе раздела). Амплитуды отражённой и преломлённой волн определяются формулами Френеля. Для $s$-поляризации коэффициент отражения не обращается в $0$ ни при каких значениях угла падения, а для $p$-поляризации при угле падения $\theta = \theta_\text{Б}$, где

\begin{equation}
    \tg \theta_\text{Б} = n = \frac{n_2}{n_1}
\end{equation}

\noindent
Таким образом при падении света под \textit{углом Брюстера} отражённая волна имеет $s$-поляризацию.

Ранее все оптические явления рассматривались в изотропных средах, где $\mu = 1$, а $\epsilon$ -- скалярная величина. Рассмотрим уравнения Максвелла в общем случае:

\begin{align*}
    \div \b D &= 0 & \rot \b E &= - \frac{1}{c} \frac{\partial \b B}{\partial t} \\
    \div \b B &= 0 & \rot \b B &= \frac{1}{c} \frac{\partial \b D}{\partial t}
\end{align*}

\noindent
Для плоской волны, то есть решения вида $\b f = \b f_0 e^{i \left( \b k \b r - \omega t \right)}$, следствия из уравнений Максвелла для среды с $\mu = 1$ выглядят следующим образом:

\begin{align*}
    \b k & \perp \b D & \left[ \b k, \b E \right] &= \frac{\omega}{c} \b B \\
    \b k & \perp \b B & \left[ \b k, \b B \right] &= - \frac{\omega}{c} \b D \\
\end{align*}

\noindent
То есть векторы $\b D$ и $\b B$ лежат в плоскости, перпендикулярной $\b k$ и перпендикулярны друг другу, а вектор $\b E$ перпендикулярен $\b B$. В общем случае вектор $\b E$ не сонаправлен с $\b D$.

\begin{equation}
    \b D = \b E + 4 \pi \b P = \b E + 4 \pi \hat{\alpha} \b E = \hat{\epsilon} \b E
\end{equation}

\noindent
Где $\hat{\alpha}$ и $\hat{\epsilon}$ -- тензоры второго ранга. Рассмотрим случай одноосного кристалла, когда в главных осях тензор диэлектрической проницаемости имеет вид

\begin{equation}
    \begin{pmatrix}
        \epsilon_1 & 0 & 0 \\
        0 & \epsilon_2 & 0 \\
        0 & 0 & \epsilon_3
    \end{pmatrix}
\end{equation}

\noindent
Причём $\epsilon_1 = \epsilon_2 = \epsilon_o$, $\epsilon_3 = \epsilon_e$. Таким образом коэффициент преломления зависит от поляризации падающей волны. В случае падения волны на кристалл под углом образуется два луча, направленных под разными углами, это явление называется \textit{двойным лучепреломлением}. Для \textit{обыкновенного луча} показатель преломления не зависит от направление распространения, для \textit{необыкновенного} -- зависит.

С помощью одноосных кристаллов можно менять поляризацию падающей волны, если направить волновой вектор $\b k$ вдоль одной из двух главных осей, где диэлектрическая проницаемость равна $\epsilon_o$. Представим падающую волну как суперпозицию двух плоских волн. Показатели преломления для этих волн будут различны: $n_o$ для одной и $n_e$ для другой. Таким образом, при прохождении пластинки толщиной $h$ возникает разность фаз $\Delta \phi k \left( n_e - n_o \right) h$. Регулируя длину пластинки, можно, например, менять поляризацию с линейной на круговую или на эллиптическую. Если получаемая разность фаз $\Delta \phi = \pi + 2 \pi m$, то пластинки называют $\lambda / 2$, если $\Delta \phi = \pi / 2 + 2 \pi m$, то $\lambda / 4$, если $\Delta \phi = 2 \pi m$, то $\lambda$. Если исходная волна имеет линейную поляризацию, то при прохождении $\lambda / 4$ пластинки поляризацией будет круговой, при прохождении $\lambda / 2$ отразится симметрично относительно одной из осей, а при прохождении $\lambda$ поляризация не изменится.
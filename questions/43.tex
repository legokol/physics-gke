\section{Дифракция рентгеновских лучей. Формула Брэгга-Вульфа. Показатель преломления вещества для
рентгеновских лучей.}

Рассмотрим явление дифракции рентгеновских лучей, то есть излучения с длинами волн в диапазоне $10^{-3} - 50$ нм. Из-за малой длины волны наблюдение интерференционных и дифракционных явлений заметно сложнее, чем в других диапазонах.

Для наблюдения дифракции лучей с длинами волн $\sim 1$ \AA используются кристаллы, поскольку с одной стороны межатомные расстояния имеют именно такой порядок, а с другой кристаллы обладают периодической структурой, что позволяет наблюдать дифракцию.

Рассмотрим дифракцию на решётке с расстоянием $d$ между атомными плоскостями, на которую падает излучение с углом скольжения $\theta$ (рис. \ref{fig:формула-Брэгга-Вульфа}). Разность хода между лучами, отражёнными от двух соседних плоскостей, равна $\Delta = 2 d \sin \theta$. Тогда для интерференционного усиления необходимо

\begin{equation}
    2 d \sin \theta = m \lambda
\end{equation}

\noindent
Это соотношение называется \textit{формулой Брэгга-Вульфа}.

\begin{figure}[htbp]
    \centering
    \def\svgwidth{0.4\linewidth}
    \input{images/43_формула_Брэгга-Вульфа.pdf_tex}
    \caption{Дифракция на двумерной решётке}
    \label{fig:формула-Брэгга-Вульфа}
\end{figure}

Стоит отметить, что помимо дифракционных максимумов, направление на которые задаётся формулой Брэгга-Вульфа, возникают другие. Они связаны с отражением от других атомных плоскостей (структура периодична во всех направлениях). При этом наиболее сильно отражение происходит от плоскостей с наибольшей густотой атомов.
\section{Квазистатические процессы. Первое начало термодинамики. Количество теплоты и работа. Внутренняя энергия. Энтальпия.}

\begin{definition}
    Термодинамическое равновесие -- состояние, в которое изолированная система приходит с течением времени и не выходит из него.
\end{definition}

\begin{definition}
    Квазистатические процесс -- процесс, который состоит из непрерывно следующих друг за другом равновесных состояний.
\end{definition}

При расширении или сжатии газ способен совершать работу. Рассмотрим, например, газ под поршнем. Пусть поршень сместился на малое расстояние $dx$, так что давление газа практически не изменилось. Тогда работа, совершённая газом, будет равна $\delta A = F dx = P S dx = P dV$, где $S$ -- площадь поршня. При квазистатическим процессах каждое состояние является равновесным, а значит работа внешних сил равна работе газа, взятой со знаком <<->> : $A_\text{внеш} = - A$.

Введём понятие адиабатической оболочки:

\begin{definition}
    Оболочка называется адиабатической, если состояние системы внутри неё остаётся неизменным при любых изменениях температуры тел вне оболочки.
\end{definition}

Сформулируем первое начало термодинамики для систем в адиабатических оболочках: \textit{Если система тел адиабатически изолирована, то работа внешних сил над этой системой зависит только от конечного и начального состояний этой системы, но не от пути, по которому осуществляется переход.} В рамках этого утверждения можно определить понятие внутренней энергии системы:

\begin{definition}
    Внутренней энергией системы $U$ называется функция состояния, приращение которой равно работе внешних сил, совершаемых над системой в адиабатической оболочке при переходе от одного равновесного состояния к другому.
\end{definition}

\begin{note}
    Внутренняя энергия определена с точностью до аддитивной постоянной.
\end{note}

Если система помещена в адиабатическую оболочку, то единственным способом изменения её внутренней энергии является совершение работы (путём изменения внешних параметров). Однако если адиабатической изоляции нет, то изменение внутренней энергии возможно, например, с помощью теплообмена.

\begin{definition}
    Энергия, полученная телом в результате теплообмена с окружающей средой, называется количеством теплоты или теплотой.
\end{definition}

Теперь можно дать математическую формулировку первого начала термодинамики: \textit{Теплота, полученная системой, идёт на приращение её внутренней энергии и совершение работы.}

\begin{equation}
    Q = \Delta U + A
\end{equation}

\noindent
Для квазистатических процессов

\begin{equation}
    \delta Q = dU + \delta A = dU + P dV
\end{equation}

Далее будет введено понятие энтропии $dS = \delta Q / T$, причём для квазистатических процессов имеет место равенство $dU = T dS - PdV$.

Помимо давления, объёма, температуры, внутренней энергии и т.п. есть ещё множество функций, которые можно использовать для описания термодинамической системы. Введём понятие \textit{энтальпии} $H$ (или $I$):

\begin{equation}
    H = U + PV
\end{equation}

\noindent
Для дифференциала энтальпии имеем:

\begin{equation}
    dH = T dS + V dP
\end{equation}

\textit{Энтальпия есть такая функция состояния, приращение которой в процессе при постоянном давлении равно количеству теплоты, полученному системой.}
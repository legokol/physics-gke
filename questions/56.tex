\section{Виды распадов. Закон радиоактивного распада. Период полураспада и время жизни.}

\textit{Радиоактивность} -- произвольное изменение состава ядра. Встречаются следующие виды радиоактивных распадов:

\begin{enumerate}
    \item $\alpha$-распад -- испускание альфа частицы (ядро гелия)
    \item $\beta$-распад -- распад, связанный с изменением заряда ядра при сохранении массового числа
    \item Протонная или нейтронная эмиссия в нестабильных ядрах
    \item Спонтанное деление неустойчивых ядер
\end{enumerate}

Радиоактивный распад -- статистическое явление, все предсказания, основанные на законах радиоактивного распада носят статистический характер.

Вероятность распада ядра за единицу времени называется постоянной распада $\lambda$, для общего числа распадов имеем:

\begin{equation}
    \frac{d N}{d t} = - \lambda N
\end{equation}

\noindent
Интегрируя это уравнение, получим

\begin{equation}
    N (t) = N_0 e^{- \lambda t} = N_0 2^{- t / T_{1 / 2}}
\end{equation}

Где величина $T_{1 / 2} = \frac{\ln 2}{\lambda}$ называется \textit{периодом полураспада}. За время, равное периоду полураспада, распадается половина от изначального числа ядер. Величина $\lambda N$ называется \textit{активностью}.

Постоянная $\lambda$ связана со \textit{временем жизни} радиоактивного ядра:

\begin{equation}
    \tau = \frac{1}{\lambda}
\end{equation}
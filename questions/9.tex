\section{Упругие деформации. Модуль Юнга и коэффициент Пуассона. Энергия упругой деформации.}

Все тела деформируемы, то есть меняют форму и объём при приложении внешних сил. Если при прекращении действия приложенных сил деформации исчезают, они называются \textit{упругими}. Иначе -- \textit{пластическими}. Тип деформации зависит не только от материала, но и от прикладываемой силы. Если напряжение (сила на единицу площади) превзойдёт предел упругости, то деформация будет пластичной.

Для определения понятия напряжения рассмотрим деформированное тело или среду. Мысленно разделим его на два и посмотрим на границу раздела. Полученные тела будут взаимодействовать с некоторой силой. Пусть на элемент поверхности одного из них $dS$ действует сила $d \b F$. Тогда напряжением назовём отношение силы к площади:

\begin{equation}
    \boldsymbol \sigma = \frac{d \b F}{dS}
\end{equation}

\noindent
В общем случае напряжённое состояние характеризуется \textit{тензором напряжений}.

Рассмотрим наиболее простой случай: одномерное растяжение/сжатие стержня. Если он растягивается под действием силы $F$, то напряжение будет равно

\begin{equation*}
    \sigma = \frac{F}{S}
\end{equation*}

\noindent
где $S$ -- площадь поперечного сечения стержня. Пусть $l_0$ -- длина недеформированного стержня, а $\Delta l$ -- приращение длины. Тогда вводится понятия относительного удлинения/сжатия (\textit{относительная деформация}):

\begin{equation}
    \epsilon = \frac{\Delta l}{l_0}
\end{equation}

Опыт показывает, что при малых деформациях относительная деформация пропорциональна напряжению

\begin{equation} \label{eq:закон гука}
    \sigma = E \epsilon
\end{equation}

\noindent
Коэффициент пропорциональности $E$ зависит от вещества и называется \textit{модулем Юнга}.

Закон Гука также представим в виде $F = k \Delta x$, где $k = E \frac{S}{l}$. Тогда работа, совершаемая при деформации, равна

\begin{equation}
    U = \int_0^{\Delta l} k x dx = \frac{k \Delta l^2}{2}
\end{equation}

\noindent
В терминах напряжений и относительных деформаций имеем

\begin{equation}
    U = \frac{E S l^2 \epsilon^2}{2 l} = \frac{E \epsilon^2}{2} V
\end{equation}

\noindent
Таким образом, упругая энергия единицы объёма

\begin{equation}
    u = \frac{E \epsilon^2}{2}
\end{equation}

Опыт показывает, что под действием растягивающей или сжимающей силы изменяются не только продольные, но и поперечные размеры. Если сила растягивающая, то поперечные размеры уменьшаются, если сжимающая -- увеличиваются. Коэффициент пропорциональности между поперечной и продольной деформацией называют \textit{коэффициентом Пуассона}:

\begin{equation}
    \mu = - \frac{\epsilon_\perp}{\epsilon_\parallel} = - \frac{\Delta d / d_0}{\Delta l / l_0}
\end{equation}

\noindent
С учётом поперечных деформаций и закона Гука можно получить следующую формулу:

\begin{equation}
    \epsilon_x = \frac{\sigma_x}{E} - \mu \frac{\sigma_y + \sigma_z}{E}
\end{equation}

Модуль Юнга и коэффициент Пуассона полностью характеризуют упругие свойства изотропного материала.
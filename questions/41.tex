\section{Волновой пакет. Фазовая и групповая скорости. Формула Рэлея. Классическая теория дисперсии.
Нормальная и аномальная дисперсии.}

Ранее предполагалось, что связь между волновым числом и частотой линейна: $\omega = k v$, где величина $v$ постоянна и зависит только от свойств среды. эта величина называется \textit{фазовой скоростью} волны:

\begin{equation}
    v_\text{ф} = \frac{\omega}{k}
\end{equation}

\noindent
Если рассмотреть уравнение $u(x, t) = f(k x - \omega t)$, то видно что $v_\text{ф}$ -- скорость перемещения точки, в которой фаза имеет фиксированное значение.

В общем случае связь между волновым числом и частотой нелинейна и определяется \textit{законом дисперсии} $\omega = \omega(k)$. Соответственно фазовая скорость зависит от частоты (или длины волны). \textit{Дисперсией} называется зависимость фазовой скорости от длины волны.

Рассмотрим две плоские волны с близкими частотам $\omega_1$ и $\omega_2$ и соответствующими им волновыми числами $k_1$ и $k_2$. Для суммы волн имеем

\begin{equation}
    E = E_1 + E_2 = E_0 \left( \cos \left( \omega_1 t - k_1 x \right) + \cos \left( \omega_2 t - k_2 x \right) \right) = 2 E_0 \cos \left( \omega t - k x \right) \cos \left( \frac{\Delta \omega}{2} t - \frac{\Delta k}{2} x \right)
\end{equation}

\noindent
Здесь $\omega = (\omega_1 + \omega_2) / 2$, $k = (k_1 + k_2) / 2$. Первый косинус меняется в пространстве и времени намного быстрее, чем второй. Вторым описывается огибающая, распространяющаяся со скоростью

\begin{equation}
    v_\text{гр} = \frac{d \omega}{d k}
\end{equation}

\noindent
Эта величина называется групповой скоростью, причём выражение для неё остаётся верным для суперпозиции произвольного числа волн.

Найдём связь между фазовой и групповой скоростями. Учитывая, что $\omega = v_\text{ф} k$, получим

\begin{equation}
    v_\text{гр} = v_\text{ф} + k \frac{d v_\text{ф}}{d k}
\end{equation}

\noindent
Учитывая, что $\lambda = 2 \pi / k$:

\begin{equation}
    v_\text{гр} = v_\text{ф} - \lambda \frac{d v_\text{ф}}{d \lambda}
\end{equation}

\noindent
Это выражение называется \textit{формулой Рэлея}.

Для электромагнитных волн $v_\text{ф} = c / n$. Если показатель преломления зависит от длины волны света $n = n(\lambda)$, то $\frac{d v_\text{ф}}{d \lambda} = \left( - v_\text{ф} / n \right) \frac{d n}{d \lambda}$. Подставляя это в формулу Рэлея, получим

\begin{equation}
    v_\text{гр} = v_\text{ф} \left( 1 + \frac{\lambda}{n} \frac{d n}{d \lambda} \right)
\end{equation}

Классическая теория дисперсии основана на модели гармонического осциллятора, когда движение электрона в поле ядра описывается уравнением

\begin{equation}
    m \ddot{\b r} + \beta \dot{\b r} + \kappa \b r = 0
\end{equation}

Пусть теперь на электрон действует электрическое поле $\b E = \b E_0 e^{- i \omega t}$. Мы не будем учитывать зависимость поля от координаты, считая, что амплитуда колебания мала. Введём обозначения $\omega_0^2 = \kappa / m$, $2 \gamma = \beta / m$. Тогда уравнение движения электрона примет вид

\begin{equation}
    \ddot{\b r} + 2 \gamma \dot{\b r} + \omega_0^2 \b r = - \frac{e \b E_0}{m} e^{ i \omega t}
\end{equation}

\noindent
Частное решение этого уравнение имеет вид

\begin{equation}
    \b r = - \frac{e \b E_0 / m}{\omega_0^2 - \omega^2 + 2 i \gamma \omega}
\end{equation}

Дипольный момент молекулы равен $\b p = - e \b r$, вектор поляризации $\b P = N \b p = \alpha \b E$, где $N$ -- концентрация. Тогда можно получить выражение для диэлектрической проницаемости $\epsilon = 1 + 4 \pi \alpha$:

\begin{equation}
    \epsilon (\omega) = 1 + \frac{4 \pi N e^2 / m}{\omega_0^2 - \omega^2 + 2 i \gamma \omega}
\end{equation}

В оптике рассматриваются среды, где $\mu = 1$, то есть показатель преломления $n = \sqrt{\epsilon} = n_0 + i \delta$, где $n_0$ -- непосредственно показатель преломления, а величина $\delta$ отвечает за затухание волн. Проанализируем полученную величину:

\begin{itemize}
    \item при $\omega \to \infty$
    \begin{equation}
        n \approx \sqrt{1 - \frac{\omega_p^2}{\omega^2}}
    \end{equation}
    \item в случае слабого затухания $| \omega_0^2 - \omega^2 | \gg \gamma \omega$:
    \begin{equation}
        n \approx \sqrt{1 + \frac{\omega_p^2}{\omega_0^2 - \omega^2}}
    \end{equation}
\end{itemize}

Области частот, где показатель преломления возрастает $d n / d \omega > 0$ или $d n / d \lambda < 0$ называется \textit{областью нормальной дисперсии}. В противном случае, когда $d n / d \omega < 0$ или $d n / d \lambda > 0$ -- \textit{областью аномальной дисперсии}.
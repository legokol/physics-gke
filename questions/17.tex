\section{Явления переноса: диффузия, теплопроводность, вязкость. Коэффициенты переноса в газах. Уравнение стационарной теплопроводности.}

С тепловым движением газа связаны различные \textit{явления переноса}, такие как \textit{диффузия}, \textit{вязкость} и \textit{теплопроводность}. Их особенность состоит в медленной скорости протекания, поскольку несмотря на высокую скорость движения молекул они довольно часто сталкиваются. Частоту столкновений характеризует характерная длина $\lambda$, называемая \textit{длиной свободного пробега} -- среднее расстояние, которое частицы проходят без столкновения.

Начнём рассмотрение этих процессов с диффузии лёгкой примеси. Пусть её концентрация меняется с изменением координаты $n = n(x)$, $\lambda$ -- длина свободного пробега а $v$ -- средняя скорость молекул. В единицу времени на площадку $dS$ между слоями газа попадает $\frac{1}{6} n(x - \lambda) v$ молекул с одной стороны и $\frac{1}{6} n(x + \lambda) v$ с другой. Таким образом поток частиц

\begin{equation*}
    j = \frac{1}{6} v \left( n(x - \lambda) - n(x + \lambda)\right) \approx - \frac{1}{3} \lambda v \frac{\partial n}{\partial x}
\end{equation*}

\noindent
Или в общем случае

\begin{equation}
    j = - D \nabla n
\end{equation}

\noindent
Где $D = \frac{1}{3} \lambda v$ -- \textit{коэффициент диффузии}. Полученное выражение называется законом Фика.

Аналогично рассматриваются явления теплопроводности, как переноса энергии, и вязкости, как переноса импульса. Для теплопроводности имеем

\begin{equation}
    q \approx - \frac{1}{3} \lambda v n \cdot c_V k \frac{\partial T}{\partial x}
\end{equation}

\noindent
Здесь $c_V$ -- теплоёмкость одной молекулы. Соответственно \textit{коэффициент теплопроводности}

\begin{equation}
    \varkappa = \frac{1}{3} n \lambda v c_V
\end{equation}

Вязкость имеет место, если при направленном движении газа скорость $u$ является функцией координаты $u(x)$. Тогда при смещении слоёв переносится импульс и его поток равен

\begin{equation}
    j_p \approx - \frac{1}{3} \lambda v n m \frac{\partial u}{\partial x}
\end{equation}

\noindent
Отсюда можно получить приближённое выражение для вязкости:

\begin{equation}
    \eta = \frac{1}{3} n m \lambda v
\end{equation}

Рассмотрим перенос тепловой энергии вдоль оси $x$. В объёме $dV = dS dx$ содержится $dU = u dV$ энергии ($u$ -- объёмная плотность энергии). Пусть $q$ -- поток тепла. Тогда количество энергии, поступившей в единицу времени, равно

\begin{equation*}
    q(x, t) - q(x + dx, t) dS \approx - \frac{\partial q}{\partial x} dV
\end{equation*}

\noindent
Это выражение для изменения энергии в выделенном объёме. Тогда

\begin{equation*}
    \frac{\partial u}{\partial t} + \frac{\partial q}{\partial x}
\end{equation*}

\noindent
В случае идеального газа $u = n c_v T$. С учётом выражения для потока тепла получим уравнение теплопроводности:

\begin{equation}
    n c_V \frac{\partial T}{\partial t} = \frac{\partial}{\partial x} \left( \varkappa \frac{\partial T}{\partial x} \right)
\end{equation}
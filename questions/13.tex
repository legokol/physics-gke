\section{Термодинамические потенциалы. Условия равновесия термодинамических систем.}

Ранее уже были введены понятия таких функций состояния как внутренняя энергия $U$, энтропия $S$ и энтальпия $H = U + PV$.

\begin{align}
    dU &= T dS - P dV \\
    dH &= T dS + V dP
\end{align}

Рассмотрим ещё несколько функций состояния:

\begin{itemize}
    \item \textit{Свободная энергия Гельмгольца} $\Psi$ ($F$)
    \begin{align}
        \Psi &= U - TS \\
        d \Psi &= -S dT - PdV
    \end{align}
    \item \textit{Термодинамический потенциал (Гиббса)} $\Phi$ ($G$)
    \begin{align}
        \Phi &= \Psi + PV = U - TS + PV \\
        d \Phi &= -S dT + V dP
    \end{align}
\end{itemize}

Эти выражения вместе с выражениями дифференциалов внутренней энергии и энтальпии позволяют рассматривать все термодинамические потенциалы как функции двух переменных. Для получения различных термодинамических соотношений часто используется равенство смешанных производных. Полученные таким образом выражения называются соотношениями Максвелла.

Введём понятие \textit{химического потенциала} $\phi$ ($\mu$, $g$). Он характеризует изменение энергии при изменении числа частиц

\begin{equation}
    dU = T dS - P dV + \phi dN
\end{equation}

\noindent
Этот член необходимо, соответственно, добавить ко всем остальным потенциалам. Так как термодинамический потенциал является функцией только интенсивных величин (не зависящих от массы системы), то 

\begin{equation}
    \phi = \frac{\Phi}{N}
\end{equation}

\noindent
то есть химический потенциал является удельным термодинамическим потенциалом. Соответственно

\begin{equation}
    d \phi = - s dT + v dP
\end{equation}

Равновесие термодинамических систем (фазовое равновесие) требует, во-первых, теплового равновесия, то есть $T_1 = T_2 = T$, во-вторых, механического равновесия $P_1 = P_2 =P$. Покажем, что также требуется равенство химических потенциалов. Произведём некоторый процесс, тогда внутренние энергии фаз изменятся:

\begin{align*}
    dU_1 &= T dS_1 - P dV_1 + \phi_1 dN_1 \\
    dU_2 &= T dS_2 - P dV_2 + \phi_2 dN_2
\end{align*}

\noindent
Пусть $S = S_1 + S_2$ -- полная энтропия системы. Ввиду изолированности энергия не будет меняться $dU_1 + dU_2 = 0$, как и объём $dV_1 + dV_2 = 0$. Таким образом

\begin{equation*}
    T dS + \phi_1 dN_1 + \phi_2 dN_2 = 0
\end{equation*}

\noindent
Так как число частиц не меняется, то $dN_1 = - dN_2$. Получим

\begin{equation*}
    T dS = \left( \phi_1 - \phi_2 \right) dN_2
\end{equation*}

\noindent
В равновесии энтропия максимальна а значит $dS = 0$, что и доказывает равенство химических потенциалов. 
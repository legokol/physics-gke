\section{Течение идеальной жидкости. Уравнение непрерывности. Уравнение Бернулли.}

Движение жидкостей или газов можно рассматривать различными способами. Один из способов -- описание движения каждой отдельно взятой частицы. Другой -- описание параметров течения (скорость, плотность, температура) в каждой точке пространства. При таком подходе параметры жидкости являются функциями координат и времени.

\begin{definition}
    Течение называется стационарным, если его параметры не меняются со временем.
\end{definition}

\begin{definition}
    Жидкость называется идеальной, если при её движении не учитывается трение о стенки и слои жидкости между собой.
\end{definition}

Рассмотрим элемент объема жидкости, не перемещающийся в пространстве. Масса в нём может меняться только за счёт "утекания через стенки":

\begin{equation*}
    \frac{\partial m}{\partial t} = \frac{\partial}{\partial t} \int_V \rho dV = - \oint_{\partial V} \rho \b v \cdot d \b A
\end{equation*}

\noindent
Пользуясь теоремой Гаусса-Остроградского для перехода от поверхностного интеграла к объёмному и равенства нулю этого интеграла по произвольному объёму получим уравнение непрерывности:

\begin{equation} \label{eq:уравнение непрерывности}
    \frac{\partial \rho}{\partial t} + \left( \nabla, \rho \b v \right) = 0
\end{equation}

Аналогом закона Ньютона для идеальной жидкости является уравнение Эйлера. Сила, действующая на малый объём со стороны давления выражается следующим интегралом:

\begin{equation*}
    - \oint_{\partial V} p d \b A
\end{equation*}

\noindent
Или, переходя к объёмному интегралу:

\begin{equation*}
    - \int_V \nabla p dV
\end{equation*}

\noindent
Таким образом для элемента объёма жидкости имеем:

\begin{equation*}
    \rho \frac{d \b v}{dt} = - \nabla p
\end{equation*}

\noindent
Если так же учесть объёмные силы $\b f$ и раскрыть полную производную скорости, получим уравнение Эйлера:

\begin{equation}
    \rho \left( \frac{\partial \b v}{\partial t} + \left( \b v, \nabla \right) \b v \right) = - \nabla p + \rho \b f
\end{equation}

Рассмотрим теперь стационарное течение идеальной жидкости в поле тяжести вдоль линии тока. Градиент в таком случае есть производная по направлению кривой $\frac{\partial}{\partial l}$, а сила тяжести потенциальна и равна градиенту потенциала соответственно. Тогда имеем

\begin{equation*}
    \rho \b v \frac{\partial \b v}{\partial l} = - \frac{\partial p}{\partial l} - \frac{\partial\left( \rho g h \right)}{\partial l}
\end{equation*}

\noindent
Проинтегрировав, получим уравнение Бернулли для идеальной жидкости:

\begin{equation}
    \frac{\rho v^2}{2} + p + \rho g h = \text{const}
\end{equation}
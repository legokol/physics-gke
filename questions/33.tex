\section{Электромагнитные волны в волноводах. Критическая частота. Объемные резонаторы.}

Рассмотрим распространение электромагнитной волны в волноводе. \textit{Волноводом} называется канал, способный поддерживать распространение в нём электромагнитных волн. Мы рассмотрим прямоугольный волновод с проводящими стенками, в котором волна распространяется вдоль оси $z$. Для простоты так же будем считать, что в среде волновода $\epsilon = \mu = 1$. Из уравнений Максвелла и условия отсутствия поля в стенках имеем граничные условия: $\b E_\tau = 0$, $B_n = 0$.

Выделают два основных типа волн: $E$-волны или ТМ-волны, в в которых $\b H \perp z$, $E_Z \neq 0$, и $H$-волны или ТЕ-волны, в которых соответственно $\b E \perp z$, $H_z \neq 0$. Рассмотрим случай ТЕ-волны. Будем искать решение в виде $\b E(\b r, t) = \b E_0 (x, y) e^{i k_z z - i \omega t}$. Подставляя это выражение в уравнение Гельмгольца, получим

\begin{equation}
    \frac{\partial^2 \b E_0}{\partial x^2} + \frac{\partial^2 \b E_0}{\partial y^2} + \left( \frac{\omega^2}{c^2} - k_z^2 \right) \b E_0 = 0
\end{equation}

Пусть вектор $\b E$ параллелен оси $y$. Тогда из теоремы Гаусса ($\div \b E = 0$) поле не зависит от координаты $y$ и уравнения приходит к виду

\begin{equation}
    \frac{d^2 E_{0 y}}{d x^2} + \kappa^2 E_{0 y} = 0
\end{equation}

\noindent
Решение этого уравнения имеет вид $E_{0 y} = c_1 \sin \kappa x + c_2 \cos \kappa x$. С учётом граничных условий $E_y |_{x = 0} = 0$, $E_y |_{x = a} = 0$, где $a$ -- ширина волновода вдоль оси $x$, получим $c_1 = 0$, $\kappa = \pi n / a$, где $n$ -- натуральное число. Таким образом поле имеет вид

\begin{equation}
    E_y = E_0 \sin \left( \frac{\pi n}{a} x \right) e^{i k_z z - i \omega t}
\end{equation}

\noindent
С учётом полученного значения $\kappa$ имеем

\begin{equation}
    \frac{\omega^2}{c^2} = \frac{\pi^2 n^2}{a^2} + k_z^2
\end{equation}

\noindent
Таким образом, существует минимальная частота волны, которая может распространяться в волноводе, называемая критической:

\begin{equation}
    \omega_\text{кр} = \frac{\pi c}{a}
\end{equation}

\noindent
В общем случае для ТЕ-волны добавляется гармоническая зависимость от второй координаты, аналогичный член добавляется и в выражение для частот.

Следует заметить, что фазовая скорость волны в волноводе $v_\text{ф} = \omega / k_z$ превышает скорость света. Однако групповая скорость, с которой распространяется информация и энергия, меньше.

Рассмотрим теперь \textit{объёмный резонатор} -- устройство, в котором накапливается ЭМ энергия от внешнего источника. Мы будем рассматривать резонатор в форме параллелепипеда с размерами $L_x$, $L_y$ и $L_z$. Граничные условия остаются теми же, что и при рассмотрении прямоугольного волновода, с добавлением граничных условий вдоль оси $z$. В итоге получим зависимость вида

\begin{align}
    E_x &= E_{0 x} \cos k_x x \sin k_y y \sin k_z z e^{i \omega t} \\
    E_y &= E_{0 y} \sin k_x x \cos k_y y \sin k_z z e^{i \omega t} \\
    E_z &= E_{0 z} \sin k_x x \sin k_y y \cos k_z z e^{i \omega t}
\end{align}

\noindent
где в силу граничных условий $k_x L_x = \pi n$, $k_y L_y = \pi m$, $k_z L_z = \pi l$. Для допустимых частот имеем:

\begin{equation}
    \frac{\omega^2}{c^2} = \frac{\pi^2 n^2}{L_x^2} + \frac{\pi^2 m^2}{L_y^2} + \frac{\pi^2 l^2}{L_z^2}
\end{equation}

Минимальная допустимая частота называется основной модой и определяется из этой формулы с учётом размеров волновода (среди чисел $n$, $m$, $l$ допустим только один ноль, иначе поля в резонаторе нет).